% ----------------------------------------------------------------------
%
%                            TFMTesis.tex
%
%----------------------------------------------------------------------
%
% Este fichero contiene el "documento maestro" del documento. Lo único
% que hace es configurar el entorno LaTeX e incluir los ficheros .tex
% que contienen cada sección.
%
%----------------------------------------------------------------------
%
% Los ficheros necesarios para este documento son:
%
%       TeXiS/* : ficheros de la plantilla TeXiS.
%       Cascaras/* : ficheros con las partes del documento que no
%          son capítulos ni apéndices (portada, agradecimientos, etc.)
%       Capitulos/*.tex : capítulos de la tesis
%       Apendices/*.tex: apéndices de la tesis
%       constantes.tex: constantes LaTeX
%       config.tex : configuración de la "compilación" del documento
%       guionado.tex : palabras con guiones
%
% Para la bibliografía, además, se necesitan:
%
%       *.bib : ficheros con la información de las referencias
%
% ---------------------------------------------------------------------

\documentclass[11pt,a4paper,twoside]{book}

%
% Definimos  el   comando  \compilaCapitulo,  que   luego  se  utiliza
% (opcionalmente) en config.tex. Quedaría  mejor si también se definiera
% en  ese fichero,  pero por  el modo  en el  que funciona  eso  no es
% posible. Puedes consultar la documentación de ese fichero para tener
% más  información. Definimos también  \compilaApendice, que  tiene el
% mismo  cometido, pero  que se  utiliza para  compilar  únicamente un
% apéndice.
%
%
% Si  queremos   compilar  solo   una  parte  del   documento  podemos
% especificar mediante  \includeonly{...} qué ficheros  son los únicos
% que queremos  que se incluyan.  Esto  es útil por  ejemplo para sólo
% compilar un capítulo.
%
% El problema es que todos aquellos  ficheros que NO estén en la lista
% NO   se  incluirán...  y   eso  también   afecta  a   ficheros  de
% la plantilla...
%
% Total,  que definimos  una constante  con los  ficheros  que siempre
% vamos a querer compilar  (aquellos relacionados con configuración) y
% luego definimos \compilaCapitulo.
\newcommand{\ficherosBasicosTeXiS}{%
TeXiS/TeXiS_pream,TeXiS/TeXiS_cab,TeXiS/TeXiS_bib,TeXiS/TeXiS_cover%
}
\newcommand{\ficherosBasicosTexto}{%
constantes,guionado,Cascaras/bibliografia,config%
}
\newcommand{\compilaCapitulo}[1]{%
\includeonly{\ficherosBasicosTeXiS,\ficherosBasicosTexto,Capitulos/#1}%
}

\newcommand{\compilaApendice}[1]{%
\includeonly{\ficherosBasicosTeXiS,\ficherosBasicosTexto,Apendices/#1}%
}

%- - - - - - - - - - - - - - - - - - - - - - - - - - - - - - - - - - -
%            Preámbulo del documento. Configuraciones varias
%- - - - - - - - - - - - - - - - - - - - - - - - - - - - - - - - - - -

% Define  el  tipo  de  compilación que  estamos  haciendo.   Contiene
% definiciones  de  constantes que  cambian  el  comportamiento de  la
% compilación. Debe incluirse antes del paquete TeXiS/TeXiS.sty
%---------------------------------------------------------------------
%
%                          config.tex
%
%---------------------------------------------------------------------
%
% Contiene la  definici\'on de constantes  que determinan el modo  en el
% que se compilar\'a el documento.
%
%---------------------------------------------------------------------
%
% En concreto, podemos  indicar si queremos "modo release",  en el que
% no  aparecer\'an  los  comentarios  (creados  mediante  \com{Texto}  o
% \comp{Texto}) ni los "por  hacer" (creados mediante \todo{Texto}), y
% s\'i aparecer\'an los \'indices. El modo "debug" (o mejor dicho en modo no
% "release" muestra los \'indices  (construirlos lleva tiempo y son poco
% \'utiles  salvo  para   la  versi\'on  final),  pero  s\'i   el  resto  de
% anotaciones.
%
% Si se compila con LaTeX (no  con pdflatex) en modo Debug, tambi\'en se
% muestran en una esquina de cada p\'agina las entradas (en el \'indice de
% palabras) que referencian  a dicha p\'agina (consulta TeXiS_pream.tex,
% en la parte referente a show).
%
% El soporte para  el \'indice de palabras en  TeXiS es embrionario, por
% lo  que no  asumas que  esto funcionar\'a  correctamente.  Consulta la
% documentaci\'on al respecto en TeXiS_pream.tex.
%
%
% Tambi\'en  aqu\'i configuramos  si queremos  o  no que  se incluyan  los
% acr\'onimos  en el  documento final  en la  versi\'on release.  Para eso
% define (o no) la constante \acronimosEnRelease.
%
% Utilizando \compilaCapitulo{nombre}  podemos tambi\'en especificar qu\'e
% cap\'itulo(s) queremos que se compilen. Si no se pone nada, se compila
% el documento  completo.  Si se pone, por  ejemplo, 01Introduccion se
% compilar\'a \'unicamente el fichero Capitulos/01Introduccion.tex
%
% Para compilar varios  cap\'itulos, se separan sus nombres  con comas y
% no se ponen espacios de separaci\'on.
%
% En realidad  la macro \compilaCapitulo  est\'a definida en  el fichero
% principal tesis.tex.
%
%---------------------------------------------------------------------


% Comentar la l\'inea si no se compila en modo release.
% TeXiS har\'a el resto.
% ���Si cambias esto, haz un make clean antes de recompilar!!!
\def\release{1}


% Descomentar la linea si se quieren incluir los
% acr\'onimos en modo release (en modo debug
% no se incluir\'an nunca).
% ���Si cambias esto, haz un make clean antes de recompilar!!!
%\def\acronimosEnRelease{1}


% Descomentar la l\'inea para establecer el cap\'itulo que queremos
% compilar

% \compilaCapitulo{01Introduccion}
% \compilaCapitulo{02EstructuraYGeneracion}
% \compilaCapitulo{03Edicion}
% \compilaCapitulo{04Imagenes}
% \compilaCapitulo{05Bibliografia}
% \compilaCapitulo{06Makefile}

% \compilaApendice{01AsiSeHizo}

% Variable local para emacs, para  que encuentre el fichero maestro de
% compilaci\'on y funcionen mejor algunas teclas r\'apidas de AucTeX
%%%
%%% Local Variables:
%%% mode: latex
%%% TeX-master: "./Tesis.tex"
%%% End:


% Paquete de la plantilla
\usepackage{TeXiS/TeXiS}

% Incluimos el fichero con comandos de constantes
%---------------------------------------------------------------------
%
%                          constantes.tex
%
%---------------------------------------------------------------------
%
% Fichero que  declara nuevos comandos LaTeX  sencillos realizados por
% comodidad en la escritura de determinadas palabras
%
%---------------------------------------------------------------------

%%%%%%%%%%%%%%%%%%%%%%%%%%%%%%%%%%%%%%%%%%%%%%%%%%%%%%%%%%%%%%%%%%%%%%
% Comando: 
%
%       \titulo
%
% Resultado: 
%
% Escribe el título del documento.
%%%%%%%%%%%%%%%%%%%%%%%%%%%%%%%%%%%%%%%%%%%%%%%%%%%%%%%%%%%%%%%%%%%%%%
\def\titulo{AdaptaMaterialEscolar2.0: Mejorando una herramienta para la adaptación de asignaturas a necesidades educativas especiales \vfill 
AdaptaMaterialEscolar2.0: Improving a tool for adapting subjects to special educational needs}

%%%%%%%%%%%%%%%%%%%%%%%%%%%%%%%%%%%%%%%%%%%%%%%%%%%%%%%%%%%%%%%%%%%%%%
% Comando: 
%
%       \autor
%
% Resultado: 
%
% Escribe el autor del documento.
%%%%%%%%%%%%%%%%%%%%%%%%%%%%%%%%%%%%%%%%%%%%%%%%%%%%%%%%%%%%%%%%%%%%%%
\def\autor{Álvaro Gómez Sittima, Dunia Namour Doughani, Alberto Alejandro Rivas Fernández, Johan Sebastian Salvatierra Gutierrez}

% Variable local para emacs, para  que encuentre el fichero maestro de
% compilación y funcionen mejor algunas teclas rápidas de AucTeX

%%%
%%% Local Variables:
%%% mode: latex
%%% TeX-master: "tesis.tex"
%%% End:

% Sacamos en el log de la compilación el copyright
%\typeout{Copyright Marco Antonio and Pedro Pablo Gomez Martin}

%
% "Metadatos" para el PDF
%
\ifpdf\hypersetup{%
    pdftitle = {\titulo},
    pdfsubject = {Plantilla de Tesis},
    pdfkeywords = {Plantilla, LaTeX, tesis, trabajo de
      investigación, trabajo de Master},
    pdfauthor = {\textcopyright\ \autor},
    pdfcreator = {\LaTeX\ con el paquete \flqq hyperref\frqq},
    pdfproducer = {pdfeTeX-0.\the\pdftexversion\pdftexrevision},
    }
    \pdfinfo{/CreationDate (\today)}
\fi


%- - - - - - - - - - - - - - - - - - - - - - - - - - - - - - - - - - -
%                        Documento
%- - - - - - - - - - - - - - - - - - - - - - - - - - - - - - - - - - -
\begin{document}

% Incluimos el  fichero de definición de guionado  de algunas palabras
% que LaTeX no ha dividido como debería
%----------------------------------------------------------------
%
%                          guionado.tex
%
%----------------------------------------------------------------
%
% Fichero con algunas divisiones de palabras que LaTeX no
% hace correctamente si no se le da alguna ayuda.
%
%----------------------------------------------------------------

\hyphenation{
% a
abs-trac-to
abs-trac-tos
abs-trac-ta
abs-trac-tas
ac-tua-do-res
A-dap-ta-Ma-te-rial-Es-co-lar
a-gra-de-ci-mien-tos
ana-li-za-dor
an-te-rio-res
an-te-rior-men-te
apa-rien-cia
a-pro-pia-do
a-pro-pia-dos
a-pro-pia-da
a-pro-pia-das
a-pro-ve-cha-mien-to
a-que-llo
a-que-llos
a-que-lla
a-que-llas
a-sig-na-tu-ra
a-sig-na-tu-ras
a-so-cia-da
a-so-cia-das
a-so-cia-do
a-so-cia-dos
au-to-ma-ti-za-do
% b
batch
bi-blio-gra-f�a
bi-blio-gr�-fi-cas
bien
bo-rra-dor
boo-l-ean-expr
% c
ca-be-ce-ra
call-me-thod-ins-truc-tion
cas-te-lla-no
cir-cuns-tan-cia
cir-cuns-tan-cias
co-he-ren-te
co-he-ren-tes
co-he-ren-cia
co-li-bri
co-men-ta-rio
co-mer-cia-les
co-no-ci-mien-to
cons-cien-te
con-si-de-ra-ba
con-si-de-ra-mos
con-si-de-rar-se
cons-tan-te
cons-trucci�n
cons-tru-ye
cons-tru-ir-se
con-tro-le
co-rrec-ta-men-te
co-rres-pon-den
co-rres-pon-dien-te
co-rres-pon-dien-tes
co-ti-dia-na
co-ti-dia-no
crean
cris-ta-li-zan
cu-rri-cu-la
cu-rri-cu-lum
cu-rri-cu-lar
cu-rri-cu-la-res
% d
de-di-ca-do
de-di-ca-dos
de-di-ca-da
de-di-ca-das
de-rro-te-ro
de-rro-te-ros
de-sa-rro-llo
de-sa-rro-llos
de-sa-rro-lla-do
de-sa-rro-lla-dos
de-sa-rro-lla-da
de-sa-rro-lla-das
de-sa-rro-lla-dor
de-sa-rro-llar
des-cri-bi-re-mos
des-crip-ci�n
des-crip-cio-nes
des-cri-to
des-pu�s
de-ta-lla-do
de-ta-lla-dos
de-ta-lla-da
de-ta-lla-das
di-a-gra-ma
di-a-gra-mas
di-se-�os
dis-po-ner
dis-po-ni-bi-li-dad
do-cu-men-ta-da
do-cu-men-to
do-cu-men-tos
% e
edi-ta-do
e-du-ca-ti-vo
e-du-ca-ti-vos
e-du-ca-ti-va
e-du-ca-ti-vas
e-la-bo-ra-do
e-la-bo-ra-dos
e-la-bo-ra-da
e-la-bo-ra-das
es-co-llo
es-co-llos
es-tu-dia-do
es-tu-dia-dos
es-tu-dia-da
es-tu-dia-das
es-tu-dian-te
e-va-lua-cio-nes
e-va-lua-do-res
exis-ten-tes
exhaus-ti-va
ex-pe-rien-cia
ex-pe-rien-cias
% f
for-ma-li-za-do
% g
ge-ne-ra-ci�n
ge-ne-ra-dor
ge-ne-ra-do-res
ge-ne-ran
% h
he-rra-mien-ta
he-rra-mien-tas
% i
i-dio-ma
i-dio-mas
im-pres-cin-di-ble
im-pres-cin-di-bles
in-de-xa-do
in-de-xa-dos
in-de-xa-da
in-de-xa-das
in-di-vi-dual
in-fe-ren-cia
in-fe-ren-cias
in-for-ma-ti-ca
in-gre-dien-te
in-gre-dien-tes
in-me-dia-ta-men-te
ins-ta-la-do
ins-tan-cias
% j
% k
% l
len-gua-je
li-be-ra-to-rio
li-be-ra-to-rios
li-be-ra-to-ria
li-be-ra-to-rias
li-mi-ta-do
li-te-ra-rio
li-te-ra-rios
li-te-ra-ria
li-te-ra-rias
lo-tes
% m
ma-ne-ra
ma-nual
mas-que-ra-de
ma-yor
me-mo-ria
mi-nis-te-rio
mi-nis-te-rios
mo-de-lo
mo-de-los
mo-de-la-do
mo-du-la-ri-dad
mo-vi-mien-to
% n
na-tu-ral
ni-vel
nues-tro
% o
obs-tan-te
o-rien-ta-do
o-rien-ta-dos
o-rien-ta-da
o-rien-ta-das
% p
pa-ra-le-lo
pa-ra-le-la
par-ti-cu-lar
par-ti-cu-lar-men-te
pe-da-g�-gi-ca
pe-da-g�-gi-cas
pe-da-g�-gi-co
pe-da-g�-gi-cos
pe-rio-di-ci-dad
per-so-na-je
plan-te-a-mien-to
plan-te-a-mien-tos
po-si-ci�n
pre-fe-ren-cia
pre-fe-ren-cias
pres-cin-di-ble
pres-cin-di-bles
pri-me-ra
pro-ble-ma
pro-ble-mas
pr�-xi-mo
pu-bli-ca-cio-nes
pu-bli-ca-do
% q
% r
r�-pi-da
r�-pi-do
ra-zo-na-mien-to
ra-zo-na-mien-tos
re-a-li-zan-do
re-fe-ren-cia
re-fe-ren-cias
re-fe-ren-cia-da
re-fe-ren-cian
re-le-van-tes
re-pre-sen-ta-do
re-pre-sen-ta-dos
re-pre-sen-ta-da
re-pre-sen-ta-das
re-pre-sen-tar-lo
re-qui-si-to
re-qui-si-tos
res-pon-der
res-pon-sa-ble
% s
se-pa-ra-do
si-guien-do
si-guien-te
si-guien-tes
si-guie-ron
si-mi-lar
si-mi-la-res
si-tua-ci�n
% t
tem-pe-ra-ments
te-ner
trans-fe-ren-cia
trans-fe-ren-cias
% u
u-sua-rio
Unreal-Ed
% v
va-lor
va-lo-res
va-rian-te
ver-da-de-ro
ver-da-de-ros
ver-da-de-ra
ver-da-de-ras
ver-da-de-ra-men-te
ve-ri-fi-ca
% w
% x
% y
% z
}
% Variable local para emacs, para que encuentre el fichero
% maestro de compilaci�n
%%%
%%% Local Variables:
%%% mode: latex
%%% TeX-master: "./Tesis.tex"
%%% End:


% Marcamos  el inicio  del  documento para  la  numeración de  páginas
% (usando números romanos para esta primera fase).
\frontmatter
\pagestyle{empty}

%---------------------------------------------------------------------
%
%                          configCover.tex
%
%---------------------------------------------------------------------
%
% cover.tex
% Copyright 2009 Marco Antonio Gomez-Martin, Pedro Pablo Gomez-Martin
%
% This file belongs to the TeXiS manual, a LaTeX template for writting
% Thesis and other documents. The complete last TeXiS package can
% be obtained from http://gaia.fdi.ucm.es/projects/texis/
%
% Although the TeXiS template itself is distributed under the 
% conditions of the LaTeX Project Public License
% (http://www.latex-project.org/lppl.txt), the manual content
% uses the CC-BY-SA license that stays that you are free:
%
%    - to share & to copy, distribute and transmit the work
%    - to remix and to adapt the work
%
% under the following conditions:
%
%    - Attribution: you must attribute the work in the manner
%      specified by the author or licensor (but not in any way that
%      suggests that they endorse you or your use of the work).
%    - Share Alike: if you alter, transform, or build upon this
%      work, you may distribute the resulting work only under the
%      same, similar or a compatible license.
%
% The complete license is available in
% http://creativecommons.org/licenses/by-sa/3.0/legalcode
%
%---------------------------------------------------------------------
%
% Fichero que contiene la configuración de la portada y de la 
% primera hoja del documento.
%
%---------------------------------------------------------------------


% Pueden configurarse todos los elementos del contenido de la portada
% utilizando comandos.

%%%%%%%%%%%%%%%%%%%%%%%%%%%%%%%%%%%%%%%%%%%%%%%%%%%%%%%%%%%%%%%%%%%%%%
% Título del documento:
% \tituloPortada{titulo}
% Nota:
% Si no se define se utiliza el del \titulo. Este comando permite
% cambiar el título de forma que se especifiquen dónde se quieren
% los retornos de carro cuando se utilizan fuentes grandes.
%%%%%%%%%%%%%%%%%%%%%%%%%%%%%%%%%%%%%%%%%%%%%%%%%%%%%%%%%%%%%%%%%%%%%%

%USAMOS LA VARIABLE DE TITULO

%%%%%%%%%%%%%%%%%%%%%%%%%%%%%%%%%%%%%%%%%%%%%%%%%%%%%%%%%%%%%%%%%%%%%%
% Autor del documento:
% \autorPortada{Nombre}
% Se utiliza en la portada y en el valor por defecto del
% primer subtítulo de la segunda portada.
%%%%%%%%%%%%%%%%%%%%%%%%%%%%%%%%%%%%%%%%%%%%%%%%%%%%%%%%%%%%%%%%%%%%%%
\autorPortada{\autor}

%%%%%%%%%%%%%%%%%%%%%%%%%%%%%%%%%%%%%%%%%%%%%%%%%%%%%%%%%%%%%%%%%%%%%%
% Fecha de publicación:
% \fechaPublicacion{Fecha}
% Puede ser vacío. Aparece en la última línea de ambas portadas
%%%%%%%%%%%%%%%%%%%%%%%%%%%%%%%%%%%%%%%%%%%%%%%%%%%%%%%%%%%%%%%%%%%%%%
\fechaPublicacion{\today}

%%%%%%%%%%%%%%%%%%%%%%%%%%%%%%%%%%%%%%%%%%%%%%%%%%%%%%%%%%%%%%%%%%%%%%
% Imagen de la portada (y escala)
% \imagenPortada{Fichero}
% \escalaImagenPortada{Numero}
% Si no se especifica, se utiliza la imagen TODO.pdf
%%%%%%%%%%%%%%%%%%%%%%%%%%%%%%%%%%%%%%%%%%%%%%%%%%%%%%%%%%%%%%%%%%%%%%
\imagenPortada{Imagenes/Vectorial/escudoUCM}
\escalaImagenPortada{.2}

%%%%%%%%%%%%%%%%%%%%%%%%%%%%%%%%%%%%%%%%%%%%%%%%%%%%%%%%%%%%%%%%%%%%%%
% Tipo de documento.
% \tipoDocumento{Tipo}
% Para el texto justo debajo del escudo.
% Si no se indica, se utiliza "TESIS DOCTORAL".
%%%%%%%%%%%%%%%%%%%%%%%%%%%%%%%%%%%%%%%%%%%%%%%%%%%%%%%%%%%%%%%%%%%%%%
\tipoDocumento{Trabajo de Fin de Grado}

%%%%%%%%%%%%%%%%%%%%%%%%%%%%%%%%%%%%%%%%%%%%%%%%%%%%%%%%%%%%%%%%%%%%%%
% Institución/departamento asociado al documento.
% \institucion{Nombre}
% Puede tener varias líneas. Se utiliza en las dos portadas.
% Si no se indica aparecerá vacío.
%%%%%%%%%%%%%%%%%%%%%%%%%%%%%%%%%%%%%%%%%%%%%%%%%%%%%%%%%%%%%%%%%%%%%%
\institucion{%
Trabajo de Fin de Grado en Ingeniería del Software\\[0.2em]
Facultad de Informática\\[0.2em]
Universidad Complutense de Madrid
}

%%%%%%%%%%%%%%%%%%%%%%%%%%%%%%%%%%%%%%%%%%%%%%%%%%%%%%%%%%%%%%%%%%%%%%
% Director del trabajo.
% \directorPortada{Nombre}
% Se utiliza para el valor por defecto del segundo subtítulo, donde
% se indica quién es el director del trabajo.
% Si se fuerza un subtítulo distinto, no hace falta definirlo.
%%%%%%%%%%%%%%%%%%%%%%%%%%%%%%%%%%%%%%%%%%%%%%%%%%%%%%%%%%%%%%%%%%%%%%
\directorPortada{Virginia Francisco Gilmartín\\Raquel Hervás Ballesteros}


%%%%%%%%%%%%%%%%%%%%%%%%%%%%%%%%%%%%%%%%%%%%%%%%%%%%%%%%%%%%%%%%%%%%%%
% Texto del primer subtítulo de la segunda portada.
% \textoPrimerSubtituloPortada{Texto}
% Para configurar el primer "texto libre" de la segunda portada.
% Si no se especifica se indica "Memoria que presenta para optar al
% título de Doctor en Informática" seguido del \autorPortada.
%%%%%%%%%%%%%%%%%%%%%%%%%%%%%%%%%%%%%%%%%%%%%%%%%%%%%%%%%%%%%%%%%%%%%%
\textoPrimerSubtituloPortada{%
\textbf{Trabajo de Fin de Grado en Ingeniería del Software}  \\ [0.3em]
% \textbf{Departamento de XXXXXXXXXXXXX} \\ [0.3em]
}

%%%%%%%%%%%%%%%%%%%%%%%%%%%%%%%%%%%%%%%%%%%%%%%%%%%%%%%%%%%%%%%%%%%%%%
% Texto del segundo subtítulo de la segunda portada.
% \textoSegundoSubtituloPortada{Texto}
% Para configurar el segundo "texto libre" de la segunda portada.
% Si no se especifica se indica "Dirigida por el Doctor" seguido
% del \directorPortada.
%%%%%%%%%%%%%%%%%%%%%%%%%%%%%%%%%%%%%%%%%%%%%%%%%%%%%%%%%%%%%%%%%%%%%%
% \textoSegundoSubtituloPortada{%
% \textbf{Convocatoria: }\textit{Febrero/Junio/Septiembre \the\year} \\ [0.2em]
% \textbf{Calificación: }\textit{Nota}
% }

%%%%%%%%%%%%%%%%%%%%%%%%%%%%%%%%%%%%%%%%%%%%%%%%%%%%%%%%%%%%%%%%%%%%%%
% \explicacionDobleCara
% Si se utiliza, se aclara que el documento está preparado para la
% impresión a doble cara.
%%%%%%%%%%%%%%%%%%%%%%%%%%%%%%%%%%%%%%%%%%%%%%%%%%%%%%%%%%%%%%%%%%%%%%
\explicacionDobleCara

%%%%%%%%%%%%%%%%%%%%%%%%%%%%%%%%%%%%%%%%%%%%%%%%%%%%%%%%%%%%%%%%%%%%%%
% \isbn
% Si se utiliza, aparecerá el ISBN detrás de la segunda portada.
%%%%%%%%%%%%%%%%%%%%%%%%%%%%%%%%%%%%%%%%%%%%%%%%%%%%%%%%%%%%%%%%%%%%%%
%\isbn{978-84-692-7109-4}


%%%%%%%%%%%%%%%%%%%%%%%%%%%%%%%%%%%%%%%%%%%%%%%%%%%%%%%%%%%%%%%%%%%%%%
% \copyrightInfo
% Si se utiliza, aparecerá información de los derechos de copyright
% detrás de la segunda portada.
%%%%%%%%%%%%%%%%%%%%%%%%%%%%%%%%%%%%%%%%%%%%%%%%%%%%%%%%%%%%%%%%%%%%%%
\copyrightInfo{\autor}


%%
%% Creamos las portadas
%%
\makeCover

% Variable local para emacs, para que encuentre el fichero
% maestro de compilación
%%%
%%% Local Variables:
%%% mode: latex
%%% TeX-master: "../Tesis.tex"
%%% End:

\chapter*{Autorización de difusión}

   
El abajo firmante, matriculado en el Máster en Ingeniería en Informática de la Facultad de Informática, autoriza a la Universidad Complutense de Madrid (UCM) a difundir y utilizar con fines académicos, no comerciales y mencionando expresamente a su autor el presente Trabajo Fin de Máster: ``TITULO DEL TRABAJO'', realizado durante el curso académico CURSO bajo la dirección de DIRECTORES en el Departamento de XXXXXXXXXXXXXXXXXXXXXXXX, y a la Biblioteca de la UCM a depositarlo en el Archivo Institucional E-Prints Complutense con el objeto de incrementar la difusión, uso e impacto del trabajo en Internet y garantizar su preservación y acceso a largo plazo.

\vspace{5cm}

% +--------------------------------------------------------------------+
% | On the line below, replace "Enter Your Name" with your name
% | Use the same form of your name as it appears on your title page.
% | Use mixed case, for example, Lori Goetsch.
% +--------------------------------------------------------------------+
\begin{center}
	\large Nombre Del Alumno\\
	
	\vspace{0.5cm}
	
	% +--------------------------------------------------------------------+
	% | On the line below, replace Fecha
	% |
	% +--------------------------------------------------------------------+
	
	\today\\
	
\end{center}

% +--------------------------------------------------------------------+
% | Dedication Page (Optional)
% +--------------------------------------------------------------------+

\chapter*{Dedicatoria}

\begin{flushright}
\begin{minipage}[c]{8.5cm}
\flushright{\textit{A Pedro Pablo y Marco Antonio, por crear TeXiS e iluminar nuestro camino}}
\end{minipage}
\end{flushright}
% +--------------------------------------------------------------------+
% | Acknowledgements Page (Optional)                                   |
% +--------------------------------------------------------------------+

\chapter*{Agradecimientos}
Nos complace expresar nuestro sincero agradecimiento a nuestras directoras, Virginia y Raquel, por toda la dedicación y apoyo en todo el proyecto. Su compromiso y asesoramiento han sido fundamentales para el desarrollo de AdaptaMaterialEscolar 2.0. También agradecemos a Antonio Navarro por sus consejos y orientación en relación con los diagramas.











%---------------------------------------------------------------------
%
%                      resumen.tex
%
%---------------------------------------------------------------------
%
% Contiene el cap�tulo del resumen.
%
% Se crea como un cap�tulo sin numeraci�n.
%
%---------------------------------------------------------------------

\chapter{Resumen}
\cabeceraEspecial{Resumen}

\begin{FraseCelebre}
\begin{Frase}

\end{Frase}
\begin{Fuente}

\end{Fuente}
\end{FraseCelebre}

Por ley, el curr�culum educativo trata de garantizar que el proceso de formaci�n del alumnado cuente con los materiales, recursos y contenidos necesarios durante su recorrido acad�mico, en igualdad de condiciones. Sin embargo, debido a que no todos los estudiantes cuentan con las mismas herramientas de aprendizaje, existe la posibilidad de realizar adaptaciones curriculares.

Actualmente no existen herramientas gratuitas para realizar adaptaciones curriculares, y los docentes dedican demasiado tiempo en crear material acad�mico para estudiantes que lo necesiten: tienen que ajustar la fuente de texto, el tama�o, buscar im�genes en Internet o escanearlas de libros, etc., e incluso algunos tienen que hacerlo a mano.

Este proyecto consiste en el desarrollo de una aplicaci�n web, AdaptaMaterialEscolar, creada espec�ficamente para facilitar a los docentes la adaptaci�n curricular no significativa de ex�menes, actividades y unidades did�cticas personalizadas para cada alumno. Para ello, hemos implementado un editor de texto que cuenta con los formatos y las funcionalidades m�s utilizadas por los profesores, con el fin de que puedan realizar las adaptaciones de manera m�s r�pida, siguiendo un Dise�o Centrado en el Usuario. Las funcionalidades que se han implementado en AdaptaMaterialEscolar son: un buscador de pictogramas, ejercicios de completar huecos en blanco, de generaci�n de sopas de letras, de verdadero o falso, de definir conceptos, y de pregunta para desarrollar con un espacio limitado.

La aplicaci�n web ha sido probada y evaluada por usuarios finales que pretenden usarla en el per�odo escolar. Los resultados obtenidos demuestran que se cubre una necesidad real del profesorado y su inter�s por seguir desarrollando y mejorando AdaptaMaterialEscolar.

\endinput
% Variable local para emacs, para  que encuentre el fichero maestro de
% compilaci�n y funcionen mejor algunas teclas r�pidas de AucTeX
%%%
%%% Local Variables:
%%% mode: latex
%%% TeX-master: "../Tesis.tex"
%%% End:

\begin{otherlanguage}{english}
\chapter{Abstract}

The educational curriculum aims to ensure equal opportunities in the students' learning process by providing the necessary materials, resources, and content throughout their academic career. However, due to the diverse educational needs of students, sometimes curricular adaptations are necessary. Unfortunately, currently available tools for curricular adaptation mostly focus on specific types of adaptations. This means that teachers have to invest a significant amount of time in creating adapted academic materials for students who need them. They have to make textual adjustments, search for images, create exercises, and so on.

This project aims to provide a centralization of curricular adaptation tools. It involves the improvement and updating of a web application created in a Degree Final Proyect (DFP) during the 20/21 academic year, called AdaptaMaterialEscolar 1.0. It was specifically developed to simplify the process of non-significant curricular adaptation for teachers. We have redesigned the original application, improved some of the existing functionalities, and added new ones. The updated features in AdaptaMaterialEscolar include a pictogram search engine, fill-in-the-blank exercises, crossword puzzle generation, true or false questions, concept definitions, PDF export, and questions requiring developed answers within a limited space. Additionally, we have developed new features that expand the capabilities of the application. These include definition exercises, concept matching, integration of gapped mathematical formulas, a drawing space, color legend, summary generation, and a pictotranslator.

The web application AdaptaMaterialEscolar 2.0 has been evaluated by end-users who intend to use it during the school period. The obtained results demonstrate that the application meets a real need of teachers and show that its usability level is highly satisfactory.

The application can be accessed at \url{http://143.47.43.25/}, and the source code can be found at \url{https://github.com/NILGroup/TFG-22-23-AdaptaMaterialEscolar2.0}.


\chapter{KeyWords}
\cabeceraEspecial{KeyWords}

Non-significant curricular adaptation

Accessibility

Teaching tool

Text editor

Web application

Academic material

Special educational needs

Educational technology

Educational innovation



% Si el trabajo se escribe en inglés, comentar esta línea y descomentar
% otra igual que hay justo antes de \end{document}
\end{otherlanguage}

\ifx\generatoc\undefined
\else
%---------------------------------------------------------------------
%
%                          TeXiS_toc.tex
%
%---------------------------------------------------------------------
%
% TeXiS_toc.tex
% Copyright 2009 Marco Antonio Gomez-Martin, Pedro Pablo Gomez-Martin
%
% This file belongs to TeXiS, a LaTeX template for writting
% Thesis and other documents. The complete last TeXiS package can
% be obtained from http://gaia.fdi.ucm.es/projects/texis/
%
% This work may be distributed and/or modified under the
% conditions of the LaTeX Project Public License, either version 1.3
% of this license or (at your option) any later version.
% The latest version of this license is in
%   http://www.latex-project.org/lppl.txt
% and version 1.3 or later is part of all distributions of LaTeX
% version 2005/12/01 or later.
%
% This work has the LPPL maintenance status `maintained'.
% 
% The Current Maintainers of this work are Marco Antonio Gomez-Martin
% and Pedro Pablo Gomez-Martin
%
%---------------------------------------------------------------------
%
% Contiene  los  comandos  para  generar los  �ndices  del  documento,
% entendiendo por �ndices las tablas de contenidos.
%
% Genera  el  �ndice normal  ("tabla  de  contenidos"),  el �ndice  de
% figuras y el de tablas. Tambi�n  crea "marcadores" en el caso de que
% se est� compilando con pdflatex para que aparezcan en el PDF.
%
%---------------------------------------------------------------------


% Primero un poquito de configuraci�n...


% Pedimos que inserte todos los ep�grafes hasta el nivel \subsection en
% la tabla de contenidos.
\setcounter{tocdepth}{2} 

% Le  pedimos  que nos  numere  todos  los  ep�grafes hasta  el  nivel
% \subsubsection en el cuerpo del documento.
\setcounter{secnumdepth}{3} 


% Creamos los diferentes �ndices.

% Lo primero un  poco de trabajo en los marcadores  del PDF. No quiero
% que  salga una  entrada  por cada  �ndice  a nivel  0...  si no  que
% aparezca un marcador "�ndices", que  tenga dentro los otros tipos de
% �ndices.  Total, que creamos el marcador "�ndices".
% Antes de  la creaci�n  de los �ndices,  se a�aden los  marcadores de
% nivel 1.

\ifpdf
   \pdfbookmark{�ndices}{indices}
\fi

% Tabla de contenidos.
%
% La  inclusi�n  de '\tableofcontents'  significa  que  en la  primera
% pasada  de  LaTeX  se  crea   un  fichero  con  extensi�n  .toc  con
% informaci�n sobre la tabla de contenidos (es conceptualmente similar
% al  .bbl de  BibTeX, creo).  En la  segunda ejecuci�n  de  LaTeX ese
% documento se utiliza para  generar la verdadera p�gina de contenidos
% usando la  informaci�n sobre los  cap�tulos y dem�s guardadas  en el
% .toc
\ifpdf
   \pdfbookmark[1]{Tabla de contenidos}{tabla de contenidos}
\fi

\cabeceraEspecial{\'Indice}

\tableofcontents

\newpage 

% �ndice de figuras
%
% La idea es semejante que para  el .toc del �ndice, pero ahora se usa
% extensi�n .lof (List Of Figures) con la informaci�n de las figuras.

\cabeceraEspecial{\'Indice de figuras}

\ifpdf
   \pdfbookmark[1]{�ndice de figuras}{indice de figuras}
\fi

\listoffigures

\newpage

% �ndice de tablas
% Como antes, pero ahora .lot (List Of Tables)

\ifpdf
   \pdfbookmark[1]{�ndice de tablas}{indice de tablas}
\fi

\cabeceraEspecial{\'Indice de tablas}

\listoftables

\newpage

% Variable local para emacs, para  que encuentre el fichero maestro de
% compilaci�n y funcionen mejor algunas teclas r�pidas de AucTeX

%%%
%%% Local Variables:
%%% mode: latex
%%% TeX-master: "../Tesis.tex"
%%% End:

\fi

% Marcamos el  comienzo de  los capítulos (para  la numeración  de las
% páginas) y ponemos la cabecera normal
\mainmatter

\pagestyle{fancy}
\restauraCabecera

%%%%%%%%%%%%%%%%%%%%%%%%%%%%%%%%%%%%%%%%%%%%%%%%%%%%%%%%%%%%%%%%%%%%%%%%%%%
% Si el TFM se escribe en ingles, comentar las siguientes líneas 
% porque no hace falta incluir nuevamente la Introducción en inglés
\begin{otherlanguage}{english}
\chapter{Introduction}
\label{cap:introduction}

Introduction to the subject area.










\end{otherlanguage}
\addtocounter{chapter}{-1} 
%%%%%%%%%%%%%%%%%%%%%%%%%%%%%%%%%%%%%%%%%%%%%%%%%%%%%%%%%%%%%%%%%%%%%%%%%%%

\chapter{Introducción}
\label{ch:introduccion}

En este capítulo se hace una introducción al Trabajo de Fin de Grado que va a ser presentado en este documento. Primero en la Sección \ref{cap:motivacio} se explicará la motivación que ha dado lugar al trabajo. A continuacción, en la Sección \ref{cap:objetivos} se presentan los objetivos que se pretende alcanzar. Por último, la estuctura del proyecto final se detalla en la Sección \ref{cap:estructura}.

\section{Motivación}\label{cap:motivacio}
La educación escolar tiene como objetivo promover el desarrollo de ciertas habilidades y el aprendizaje de varios contenidos necesarios para que los estudiantes se conviertan en miembros activos de la sociedad. Para ello, la escuela debe dar respuestas educativas que eviten la discriminación y promuevan la igualdad de oportunidades. Los docentes permiten lograr estos objetivos empleando recursos pedagógicos como el currículo educativo, en el cual se incluyen los planes de estudio, los fundamentos, la metodología y los programas para facilitar a los alumnos una formación integral y completa.

Además, en el curriculum escolar todos los alumnos tienen necesidades educativas comunes. Sin embargo, no todos los estudiantes se enfrentan al mismo con las mismas capacidades de aprendizaje, sino que cada alumno tiene necesidades individuales. La mayoría de estas se abordan a través de acciones simples: dar a los alumnos más tiempo para aprender determinados contenidos, diseñar actividades complementarias, etc.  Sin embargo, también existen necesidades individuales que no se pueden atacar por estos medios, y que precisan una serie de medidas didácticas especiales, diferentes de las normalmente requeridas para la mayoría de los estudiantes. Dichas necesidades se pueden satisfacer con las adaptaciones curriculares. Existen dos tipos de adaptaciones curriculares:
\begin{itemize}
    \item Adaptación no significativa: Adaptaciones en la metodología, las actividades, los tiempos, las técnicas e instrumentos de evaluación. No modifican los contenidos del curriculum.  
    \item Adaptación significativa: Ajustes significativos en el curriculum, es decir se eliminan apartados del curriculum oficial. 
\end{itemize}
Las adaptaciones curriculares no significativas para los Alumnos con Necesidades Educativas Especiales (ACNEE) deben ser realizadas por los profesores. Sin embargo, no se les facilita una herramienta para ello a pesar de ser un trabajo muy costoso, ya que requiere la adaptación personalizada de materiales, pruebas de evaluación, etc.

\section{Objetivos}\label{cap:objetivos}
La finalidad de este TFG es proporcionar una herramienta al profesorado que permita la adaptación curricular no significativa de los contenidos de las asignaturas de forma intuitiva, rápida y simple, con el objetivo de hacer materiales que se trabajen en el aula, adaptados a las distintas necesidades de los alumnos.

Para crear nuestra herramienta partiremos de la aplicación AdaptaMaterialEscolar 1.0  que permite la creación de diferentes tipos de ejercicios (por ejemplo, sopas de letras, rellenar espacios en blanco, etc). Analizaremos en detalle la herramienta creada, los requisitos cubiertos por ella y los que quedaron por incorporar, rediseñaremos tanto la interfaz de la aplicación como la arquitectura, seguiremos una metodología de Diseño Centrado en el Usuario (DCU) para encontrar cuales son las necesidades que quedan por cubrir y finalmente evaluaremos el resultado de nuestro TFG con usuarios finales.
 
En relación con los objetivos académicos, aspiramos a emplear los conocimientos adquiridos durante el Grado de Ingeniería del Software en un proyecto real y adquirir nuevos conocimientos.



\section{Estructura del proyecto}\label{cap:estructura}
La memoria se encuentra organizada en ocho capítulos. A continuación, se realiza un pequeño resumen de cada uno, exceptuando el actual y la traducción al inglés.
\begin{itemize}
    \item \textbf{\hyperref[cap:estadoDelArte]{Capítulo 3}}: Se presenta el estado del arte, en el que se define qué es la adaptación curricular y los tipos posibles. Además se incluyen las herramientas existentes y una mención a la aplicación de  AdaptaMaterialEscolar 1.0.
    \item \textbf{\hyperref[cap:metodologia]{Capítulo 4}}: Se presenta la metodología usada junto a sus reglas, políticas y el tablero. También se encuentra explicado el plan de pruebas.
    \item \textbf{\hyperref[cap:AdaptaMaterialEscolar2.0]{Capítulo 5}}: En este capítulo se explica todo lo relacionado con la segunda versión de AdaptaMaterialEscolar.
    \item \textbf{\hyperref[cap:conclusiones]{Capítulo 6}}: En este capítulo se presentan las conclusiones y el trabajo futuro a realizar.
    \item \textbf{\hyperref[cap:conclusions]{Capítulo 7}}: Se presenta las conclusiones y el trabajo futuro a realizar en inglés.
    \item \textbf{\hyperref[cap:TrabajoIndividual]{Capítulo 8}}: Se presenta el trabajo individual realizado por cada integrante del grupo.
\end{itemize}
\chapter{Estado del arte}
\label{cap:estadoDelArte}

\section{Adaptación curricular}
\nocite{adaptacionCurricular}
\nocite{adaptacionCurricular2}
En nuestro sistema educativo, aceptar la diversidad del alumnado y la individualidad de cada uno de ellos, constituye la base del quehacer de los docentes. Los profesores deben modificar el currículum y el programa de aula con el fin de adaptar el desarrollo del aprendizaje. Para poder realizar esta actividad, el profesorado deberá detectar, evaluar y valorar al alumno y a los elementos curriculares y del entorno.

Una vez detectado esto, el profesorado se encuentra preparado para amoldarse a las circunstancias del alumno, gracias a esto el estudiante adquiere la atención educativa que requiere, logrando una mejora en su desarrollo personal y social.

Con esto, determinamos que la adaptación curricular es un mecanismo para particularizar el currículum, ayudando a efectuar las labores docentes con el fin de apoyar al alumno a conseguir el nivel exigido por el Currículum Oficial. \nocite{adaptacionIntro} 

\subsection{Concepto de la adaptación curricular}
El BOE \citep{BOE} define la adaptación curricular como ``la medida de modificación de los elementos del currículo a fin de dar respuesta a las necesidades del alumnado. En todo caso, la adaptación tendrá como referente los objetivos y las competencias básicas del currículo que corresponda.''

Es decir, la adaptación curricular es cualquier adaptación personal para estudiantes, cuyas necesidades no pueden satisfacer el currículo ordinario y por tanto, no pueden acceder a él de la misma manera que sus compañeros.

En definitiva, son planes de acción y estrategias didácticas que incluye las modificaciones del currículo, asegurándose de que los estudiantes tengan éxito en el proceso de aprendizaje y alcancen los objetivos generales definidos.

Para poder aplicar la adaptación curricular a un alumno se debe precisar el tipo de adaptación en función de sus necesidades, para que pueda alcanzar los objetivos propuestos. Para ello, definimos los tipos de adaptaciones que más uso tienen, enfocándonos tanto en el acceso como en el currículum. 

\subsection{Adaptaciones de acceso}
Las adaptaciones de acceso permiten al alumno acceder a los diferentes componentes del currículum. No implica una adaptación del currículum sino únicamente un acceso a él. 
Dicha adaptación puede tomar a su vez diferentes tipos:

\subsubsection{Acceso espacial}
Hacen referencia a las adaptaciones en relación con el espacio. Destacan las siguientes:
\begin{itemize}
    \item \textbf{Adaptación en la sonorización del aula:} Significa que las aulas deben tener un cierto nivel de volumen adecuado, sin que haya ruidos contínuos, sin eco,etc. Dichas clases son especialmente adecuadas para estudiantes que tienen impedimentos auditivos o visuales o ,que requieren, por su propia condición especial, un entorno con pocos estímulos auditivos que les distraiga.
    \item \textbf{Adaptación en la iluminación del aula:} Las aulas bien iluminadas requieren que estas no tengan sombras, deben poseer ventanales que suministren luz natural o en su lugar luz artificial. Estos requisitos son necesarios para los estudiantes con discapacidades sensoriales.
    \item \textbf{Adaptación en el espacio físico:} Es todo aquello relacionado con la ausencia de barreras arquitectónicas:  braille en las puertas,ascensores, pasamanos, remate de las escaleras rugoso, etc. En esta sección también se encuentran los aspectos relacionados con la ubicación del aula(sin escaleras, lugares poco ruidosos) y con la disposición del estudiante dentro del aula ( al lado de un enchufe, del profesor, de la puerta,etc.)
\end{itemize}

\subsubsection{Acceso material}
\begin{itemize}
    \item \textbf{Material adaptado:} Se refieren a materiales que se usan habitualmente, se adaptan para un uso apropiado por parte de los alumnos con necesidades especiales, ejemplo de ello es plastificar un libro o hacer dibujos con relieve.
    \item \textbf{Materiales específicos:} Los materiales específicos deben superar las dificultades de los niños, por ejemplo el mobiliario, las sillas y mesas deben de estar adaptadas, los comunicadores electrónicos con salidas de voz o escrita, etc.
\end{itemize}

\subsubsection{Acceso de comunicación}
Algunos alumnos son incapaces de comprender o expresarse por medio del lenguaje oral, o su nivel no es suficiente para poder comunicarse correctamente. Dichos alumnos requieren estudiar y usar códigos de comunicación suplementarios al lenguaje oral, o alguna alternativa al mismo. Aprender dichos códigos de comunicación facilitarán el acceso a los elementos curriculares ordinarios, les concederá una herramienta imprescindible tanto para el desarrollo de algunas competencias y aprendizajes de diferentes contenidos, como para relacionarse y comunicarse con el resto de personas.

\subsection{Adaptaciones del currículo}

Como se ha mencionado anteriormente, la adaptación curricular es un conjunto de refinamientos y cambios en los elementos del currículum para adaptar la respuesta educativa a las necesidades formativas de los estudiantes especiales. 

Las adaptaciones curriculares se dividen en dos tipos, las adaptaciones curriculares significativas y las adaptaciones curriculares no significativas. 
\subsubsection{Adaptaciones curriculares significativas}
Según el BOE \citep{BOE}, ``una adaptación curricular será significativa cuando la modificación de los elementos del currículo afecten al grado de consecución de los objetivos, los contenidos y los aprendizajes imprescindibles que determinan las competencias básicas en la etapa, ciclo, grado, curso o nivel correspondiente.''

Es decir, la adaptación curricular significativa son los ajustes que se realizan en el currículum. Para su elaboración e implementación se deben seguir el criterio de menor a mayor significatividad, el enfoque sería el siguiente:
\begin{itemize}
    \item Inclusión. Se introducen elementos curriculares no presentes en el currículo. Puede incorporar objetivos, contenidos, criterios de evaluación, etc conforme a las necesidades del alumno.
    \item Reformulación. Esta adaptación conlleva la amplia modificación de los elementos del currículo.
    \item Temporalización fuera de ciclo. Los alumnos con ritmo de aprendizaje lento con respecto a sus compañeros, tendrán la oportunidad de conseguir los objetivos en otro ciclo posterior, posponiendo a otras etapas algunos elementos curriculares.
    \item Eliminación. Este tipo de adaptación es la más significativa. Inicialmente se deben eliminar contenidos, a continuación, criterios de evaluación y objetivos, finalmente se propondrá quitar material.
\end{itemize}

\subsubsection{Adaptaciones curriculares no significativas}
Son adaptaciones que no modifican sustancialmente el contenido del currículo oficial, es decir, se adapta la metodología, las actividades, los tiempos, las técnicas e instrumentos de evaluación. Para su elaboración se debe  seguir el criterio de menor a mayor significatividad, los aspectos serían los siguientes:
\begin{itemize}
    \item Cómo evaluar: Se ajusta la manera de evaluar a las necesidades del alumno, ejemplo de ello es cuando un alumno con escayola no puede realizar un examen escrito por lo que se le adapta la forma de evaluar realizando un examen oral.
    \item Metodología: Aquí se hace mención a cómo se enseña, por ejemplo, la manera de explicar a algunos alumnos que a otros, es decir, se puede dar la situación de que un alumno requiera que le expliquen las cosas más lentamente que a otro estudiante.
    \item Priorización de objetivos o contenidos: Dentro de la planificación se podría dar más valor a unos contenidos que a otros.
    \item Temporalización de contenidos u objetivos: Permitir más tiempo para alcanzar algunos de los contenidos pero respetando el ciclo, ejemplo de esto es trabajar elementos de segundo en tercero sin que sea muy significativo.
\end{itemize}

\subsection{Ejemplificación de algunas adaptaciones curriculares asociadas a diferentes necesidades }

\begin{itemize}
    \item Discapacidad motora: Es un grupo de alteraciones que se producen como consecuencia de diversas anomalías en los Sistemas que forman el movimiento. Este tipo de discapacidad requiere  adaptaciones de acceso tales como rampas, pasamanos, suelos antideslizantes, etc. 
    
    En relación a las adaptaciones significativas atañen sobre todo al área de Educación Física, Música o Plástica ya que en estas modalidades se precisa del manejo de instrumentos. 

    Con respecto a las adaptaciones no significativas se debe adaptar la forma de evaluar ya que se debe tener en cuenta su movilidad. 
    
    \item Discapacidad auditiva: Es la pérdida parcial o completa de la audición, supone la obtención del lenguaje oral por otras vías como por la visual. 
    
    Con respecto a las adaptaciones de acceso, el alumno se debe encontrar en una zona del aula en la que no haya muchas sombras ya que la adquisición de conocimientos se realiza por vía visual. Por otro lado, los alumnos con una pérdida parcial de la audición necesitan de un ambiente poco ruidoso. 
    
    En relación a las adaptaciones curriculares significativas, los profesores deberán trabajar de forma conjunta con especialistas en audición y lenguaje para que el alumno logre alcanzar los objetivos conectados con el lenguaje oral.

    En cuanto a las adaptaciones curriculares no significativas, hay que tener en cuenta la manera de evaluar, además de la forma de hablar al alumno, esta debe ser de un modo claro, sin gesticular excesivamente, etc.
    
    \item Discapacidad intelectual: El funcionamiento intelectual es significativamente más bajo que la media. En este tipo de discapacidad no es muy relevante las adaptaciones de acceso, pero podemos destacar el posicionamiento del alumno en el aula, de manera que se encuentre en una zona donde no tengas muchas distracciones.
    
    Las adaptaciones curriculares significativas se aplicarán en función de su nivel de competencia curricular. 

    En relación a las adaptaciones curriculares no significativas se centrarán en la metodología como por ejemplo, se incentivará la motivación y el refuerzo positivo.
    
    \item Espectro autista: Discapacidades del desarrollo en el cerebro. 
    En relación a las adaptaciones de acceso al espacio se precisa no realizar grandes cambios en la disposición del mobiliario. También destacamos las adaptaciones de comunicación, ya que las personas autistas se caracterizan por la ausencia de comunicación , para ayudar a romper la barrera de la comunicación lo que realizan es asociar palabras con gestos e impulsar un refuerzo positivo.

    En cuanto a las adaptaciones curriculares significativas, se debe introducir o priorizar el contenido en lo que respecta a la  comunicación o rediseñar los objetivos o elementos que no alcancen.

   Las adaptaciones curriculares no significativas se centran en la metodología. Las actividades deben de ser consistentes, con una estructura y organización claras.

    \item Altas capacidades intelectuales: En este caso no se requiere de adaptaciones de acceso ya que  este tipo de alumnos no tienen dificultades para acceder al currículum. 
    
    Con respecto a las adaptaciones curriculares significativas lo que deberían realizar los profesores es ampliar el currículum añadiendo objetivos y contenidos.

   Las adaptaciones curriculares no significativas hacen hincapié en la metodología por ejemplo, haciendo actividades de ampliación.

\end{itemize}
\nocite{adaptacionUNED}

\subsection{Conclusión}
Tras una exhaustiva documentación acerca de las adaptaciones y aportar ejemplos de ello, determinamos que las adaptaciones han de ser individualizadas ya que cada alumno tiene sus propias dificultades. El profesorado ha de ser el encargado de la adaptación, pero su realización es algo tediosa y no disponen de herramientas que faciliten la modificación del material. Por ello, en este TFG nos centraremos exclusivamente en las adaptaciones curriculares no significativas, realizando una aplicación para que el profesorado pueda adaptar el material. Por otra parte, nuestro enfoque está en las adaptaciones mencionadas anteriormente, ya que son las únicas que podemos tratar, puesto que no tenemos la potestad de modificar y/o acceder al currículum. 


\section{Herramientas existentes para adaptaciones curriculares}
Widgit Symbols\footnote{\url{https://www.widgit.com/about-symbols/widgit_symbol_set.htm}} es un conjunto de símbolos coloridos y sencillos que cubren un amplio vocabulario de palabras y admite 17 idiomas, principalmente el inglés. Los propios creadores de Widgit también ofrece herramientas\footnote{\url{https://www.widgit.com/products/index.htm}} que permiten combinar la escritura de texto con pictogramas utilizando los Widgit Symbols. Por ejemplo, InPrint 3\footnote{\url{https://www.widgit.com/products/inprint/index.htm}} permite utilizar plantillas y editarlas para relacionar texto con pictogramas facilitando la adaptación de cualquier información, documento o recurso didáctico.

EducaPlay\footnote{\url{https://es.educaplay.com/?lang=es}} es una web que permite crear actividades interactivas o juegos didácticos. Entre las actividades que se pueden realizar en EducaPlay hay sopas de letras, pruebas, juegos de memoria, mapas interactivos, ejercicios de relacionar columnas, etc. El problema de esta web es que está diseñada para realizar las actividades en la web y muchas de las actividades no se pueden imprimir en papel, por lo que requieren un ordenador con acceso a internet. 

\section{AdaptaMaterialEscolar1.0}

Una vez introducida la adaptación curricular, pasamos a describir AdaptaMateriaEscolar1.0 con el fin de exponer la aplicación anterior y de esta forma poder comprender la nueva versión de esta. 


\subsection{Captura de requisitos}
La captura de requisitos se realizó hablando directamente con el usuario final para poder conocer sus necesidades reales. Para esto se hicieron varias reuniones con 2 profesoras del Aula TEA (Trastornos del Espectro Autista) del IES Maestro Juan de Ávila de Ciudad Real.

También, se presentó la aplicación en un workshop del grupo de investigación (Natural Interaction based on Language). Los profesores pidieron a los alumnos del TFG añadir una funcionalidad que permita introducir en el papel el espacio suficiente para aquellos los alumnos con un amplio tamaño de letra.

A partir de estas reuniones se decidió que la aplicación fuera web, además de no ser necesario un sistema de iniciar sesión, y se capturó una lista de funcionalidades que se deberían implementar. Se llegó a la conclusión de que la aplicación fuera web ya que no todos los profesores tienen un ordenador personal en el que puedan instalar aplicaciones. Por otro lado, se decidió que no hacía falta iniciar sesión en la aplicación, para que los profesores no tengan que usar su información personal. Para poder priorizar las funcionalidades, se le envió una lista con todos los requisitos a las profesoras, para que ellas les dieran una importancia del 1 al 3, siendo el 1 la menor importancia y el 3 la mayor importancia. Luego, el equipo de desarrollo les dio una dificultad del 1 al 3, siendo el 1 la mayor dificultad y el 3 la menor dificultad. A continuación, se hizo una tabla con las importancias y dificultades medias de cada requisito. Estos 2 valores se multiplicaron para obtener un resultado, según el cual se ordenaron las funcionalidades de mayor prioridad a menor.


\subsection{Diseño de la aplicación}
Para realizar el diseño de la aplicación, el equipo empezó haciendo un boceto en papel entre todos. Luego, llevaron a cabo un diseño en Moqups,  una aplicación que permite crear prototipos, diagramas, etc. 

Para conseguir mejorar el diseño, decidieron que cada miembro del equipo debería realizar un prototipo por su cuenta para luego ponerlos todos en común, y poder ver las diferentes perspectivas. Se hizo así para que cada persona pudiera plasmar sus ideas sin ser influenciada por las ideas de los demás.

Finalmente, se realizó el diseño final de la aplicación después de poner en común los prototipos de todo el equipo. Una vez terminado, se le envió a las profesoras del Aula TEA para poder recibir feedback, el cual fue positivo.


\subsection{Implementación}
Las funcionalidades implementadas fueron las siguientes:

\begin{itemize}
    \item Subir un documento fuente PDF, a partir del cual se puede realizar las adaptaciones, ejercicios, etc.
    \item Editor, en el que se pueden añadir y modificar las adaptaciones. También sirve como editor de texto, en el que se puede cambiar la fuente de letra, el color, posicionamiento del texto, etc.
    \item Buscador de pictogramas que permite añadirlos al editor.
    \item Ejercicio de completar huecos, en el que a partir de un texto puedes seleccionar las palabras que deben ser completadas.
    \item Ejercicio de definiciones.
    \item Ejercicio de desarrollo, en el que se puede crear un enunciado y añadir un cierto número de líneas para la respuesta.
    \item Sopa de letras, con ciertas palabras dadas por el usuario y un cierto tamaño.
    \item Ejercicio de verdadero o falso.
\end{itemize}

Para desarrollar AdaptaMaterialEscolar1.0, se utilizó React, una biblioteca de JavaScript de código abierto mantenida por Meta para crear interfaces de usuario. React nos permite crear un front-end interactivo y complejo de una forma mucho más fácil que utilizando Javascript puro. Se basa en una arquitectura de componentes reutilizables que tienen su propio estado y que juntos forman una aplicación completa.

Para simplificar la gestión del estado de la aplicación, se utilizó Redux, un patrón de arquitectura de datos que permite controlar el estado de la aplicación de manera predecible, reduciendo el número de relaciones entre componentes de la aplicación, manteniendo un flujo de información sencillo.

Para las pruebas de unidad se utilizó Jest, una librería de JavaScript para testeo mantenida por Meta.

En vez de implementar un editor de texto, se decidió utilizar una librería externa, ya que desarrollar un editor desde cero llevaría demasiado trabajo y sería ineficiente porque ya existen editores que se pueden reutilizar. En este caso se utilizó CKEditor.

Por último, para el buscador de pictogramas, se empleó la API de ARASAAC, la cual fue creada por el Centro Aragonés para la comunicación Aumentativa y Alternativa. Esta API nos permite hacer una petición con un término de búsqueda y nos devuelve una serie de pictogramas relacionados.


\subsection{Evaluación}

El objetivo de la evaluación fue descubrir si la aplicación es realmente útil para los profesores y si les ayuda a resolver sus problemas. Para esto se creó un exámen de Ciencias Naturales adaptado usando AdaptaMaterialEscolar1.0. Luego, se replicó este exámen con los profesores para mostrarles cómo se usaría la herramienta en situaciones reales.

Después de esta demostración, se les hizo una encuesta a los profesores para que pudieran dar su opinión y feedback. Este cuestionario tenía preguntas sobre usabilidad, diseño, funcionalidades y utilidad real de la aplicación. En el cuestionario se usó una serie de preguntas llamadas Escala de Usabilidad de un Sistema o SUS (System Usability Scale), que sirven para medir que tan buena es la usabilidad de un sistema. 

Esta evaluación de la aplicación se realizó con varios profesores en diferentes días. Primero se hizo con las 2 profesoras del aula TEA, la orientadora y otra profesora del IES Maestro Juan de Ávila, de Ciudad Real. Luego 3 docentes de Biología, Historia y Geografía se pusieron en contacto con el equipo ya que también querían probar la aplicación. Por último, la jefa de estudios y profesora del IES Pedro Álvarez de Sotomayor, de Manzanares, también se vio interesada ya que en su instituto había un alumno con sordera total.

Como resultado de la evaluación se llegó a la conclusión de que la aplicación sí resuelve problemas reales que tienen los profesores y que se debería seguir desarrollando.

En la encuesta se obtuvo un 99 sobre 100 en la escala SUS. También se obtuvo un 4,6 sobre 5 en estética y se recomendó que cada pestaña de las funcionalidades fuese de un color diferente para diferenciarlas mejor. También se observó que los pictogramas a veces eran muy pequeños y debían poder aumentarse de tamaño.

Se obtuvo una nueva lista de requisitos a implementar para mejorar la aplicación:

\begin{itemize}
    \item Traducir pictogramas a lenguaje natural y viceversa.
    \item Cuadrícula para ejercicios de matemática.
    \item Poder añadir doble pauta, en vez de renglones de una línea para determinar el tamaño de letra del alumno.
    \item Poder recortar imágenes.
    \item Añadir encabezado con el nombre del centro educativo, asignatura y nombre del alumno.
    \item Añadir espacio para dibujar.
    \item Fórmulas con huecos que puedan ser rellenados por el alumno.
    \item Enumerar ejercicios automáticamente.
    \item Fuente de letra escolar.
\end{itemize}
%\include{Capitulos/Capitulo3}
%\include{Capitulos/Capitulo4}
%\include{Capitulos/Capitulo5}
\chapter{Conclusiones y Trabajo Futuro}
\label{cap:conclusiones}

En la Sección \ref{sec:conclusiones} se explicará las conclusiones a las que se han llegado tras realizar el proyecto y en la Sección \ref{sec:TrabajoFuturo} se describirá las posibles mejoras que se podrían llevar a cabo en la aplicación.

\section{Conclusiones}
\label{sec:conclusiones}
En este Trabajo de Fin de Grado (TFG), se planteó como objetivo principal el desarrollo de una aplicación web que facilite la adaptación curricular no significativa para los docentes. Para lograr dicho objetivo, se investigaron los distintos tipos de adaptaciones y herramientas disponibles. La aplicación fue diseñada con el propósito de ofrecer una amplia variedad de adaptaciones. Para ello, se inició un proceso de rediseño de AdaptaMaterialEscolar 1.0, a través de una iteración competitiva con todos los miembros del equipo, siguiendo una metodología de Diseño Centrado en el Usuario (DCU). Además, se realizaron cambios en la arquitectura del proyecto y las tecnologías utilizadas. Entre los cambios destacados, se migró de una arquitectura \textit{serverless} a una arquitectura cliente-servidor. También se decidió no utilizar Redux ni Sass y se empezó a utilizar Tailwind CSS. Asimismo, debido a que la licencia de CKEditor que se utilizaba en AdaptaMaterialEscolar 1.0 expiró y no fue posible renovarla, se implementó el editor de texto utilizando Slate. Se tuvo que implementar completamente los requisitos mínimos del editor que se solicitaron para AdaptaMaterialEscolar 1.0, ya que Slate no ofrece funcionalidades básicas de editor, sino utilidades para implementarlas. En el desarrollo de la aplicación se incluyeron la mayoría de los ejercicios y herramientas propuestos, como la funcionalidad de nuevo archivo, exportar a PDF, relacionar conceptos, ejercicios de matemáticas con huecos, ejercicio con espacio para dibujar, leyenda de colores, generar resumen y pictotraductor. Las funcionalidades implementadas en AdaptaMaterialEscolar 1.0 tuvieron que ser creadas desde cero debido a los cambios en la arquitectura y el rediseño, así como a ciertos cambios en los requisitos. Una vez desarrollada la aplicación, se alojó en un servidor de Oracle para llevar a cabo una evaluación con usuarios. Como resultado de dicha evaluación, se recopilaron datos sobre la usabilidad y utilidad de la aplicación.

Además, otro objetivo del TFG fue aplicar y ampliar los conocimientos adquiridos durante la carrera. Las asignaturas más influyentes para el desarrollo del TFG fueron:

\begin{itemize}
    \item \textbf{Gestión de proyectos Software}: Se centra en la planificación, organización, seguimiento y control de todos los aspectos de un proyecto de software, desde la concepción hasta la entrega final del producto. En esta asignatura hemos aprendido cómo administrar los recursos y el tiempo para garantizar que los proyectos se completen dentro de los plazos establecidos. También hemos aprendido la importancia de establecer objetivos claros, crear un plan de proyecto sólido y hacer un seguimiento regular del progreso para garantizar que el proyecto esté en camino.
    \item \textbf{Aplicaciones Web}: Enseña a cómo diseñar y desarrollar aplicaciones web eficientes y escalables. Además, cubre una amplia variedad de tecnologías, desde el diseño y la creación de interfaces de usuario hasta la gestión de bases de datos y la seguridad de aplicaciones. En concreto hemos utilizado los conocimientos adquiridos sobre \textit{HTML}, \textit{CSS}, \textit{JavaScript} y \textit{Node.js}.
    \item \textbf{Ingeniería del Software, Modelado de Software}: Las asignaturas de Ingeniería de Software y Modelado de Software son importantes para el desarrollo de software de alta calidad. La primera se enfoca en los principios, prácticas y técnicas necesarias para crear software eficiente y funcional. La segunda se enfoca en crear modelos precisos y detallados antes de la implementación para reducir errores y permitir una implementación más rápida y efectiva. En ambas asignaturas se cubren técnicas y habilidades esenciales para el proceso de desarrollo de software. En concreto hemos aplicado los conocimientos adquiridos sobre la gestión de errores y los patrones de diseño.
    \item \textbf{Estructura de Datos}: La comprensión de las estructuras de datos y las técnicas de manipulación de datos son esenciales para el desarrollo de software de alta calidad y eficiente. Esta asignatura proporciona una base sólida para construir soluciones de software más efectivas y escalables en el futuro. En concreto hemos aplicado los conocimientos adquiridos para entender mejor la estructura de Slate.
    \item \textbf{Ética, legislación y profesión}: Se centra sobre los aspectos éticos y legales de la ingeniería de software, como la privacidad de los datos, la propiedad intelectual, la responsabilidad social y profesional, y la seguridad del software. También se enseñan las leyes y regulaciones relevantes, como la Ley de Protección de Datos Personales y la Ley de Propiedad Intelectual.
    \item \textbf{Administración de Sistemas y Redes}: Se centra en la administración de sistemas operativos, incluyendo la instalación, configuración y mantenimiento de servidores y clientes. También se enseña la administración de redes, incluyendo la configuración de routers, switches y firewalls, la gestión de direcciones IP y el monitoreo del tráfico de la red. En concreto hemos aplicado los conocimientos adquiridos para montar el servidor en el que se ha alojado la aplicación.
\end{itemize}

Durante el desarrollo de la aplicación AdaptaMaterialEscolar, hemos adquirido nuevos conocimientos sobre el uso de tecnologías, como React, ampliamente utilizada en el mundo laboral. Además, hemos aprendido y aplicado eficazmente Tailwind CSS y Slate, herramientas que han contribuido a una experiencia de desarrollo más eficiente y flexible. También hemos con usuarios finales, identificando y satisfaciendo sus necesidades.

\section{Trabajo Futuro}
\label{sec:TrabajoFuturo}
Después de desarrollar el proyecto y cumplir con la mayoría de los requisitos fijados al principio del trabajo, es inevitable que queden algunas tareas pendientes para posible trabajo futuro.

Desde el punto de vista de los requisitos, ya en la Sección \ref{cap:requisitos} se comentó las funcionalidades descartadas por falta de información. En concreto estos requisitos son:
\begin{itemize}
    \item Añadir imágenes buscando una palabra.
    \item Sustituir una palabra por una imagen.
    \item Crear una herramienta de recorte de imágenes para el texto original.
    \item Crear tablas que organicen el temario y/o las actividades, seleccionando contenido.
    \item Crear esquemas.
    \item Ejercicios de completar los espacios en blanco en tablas y esquemas.
\end{itemize}

Desde la perspectiva de los requisitos, se han dado requisitos que se han postergado por la prioridad establecida de los requisitos. Específicamente, estos requisitos son:
\begin{itemize}
    \item Importar a Word: no lo hemos realizado porque requeriría cambiar el comportamiento de los elementos tratados por Slate. Condideramos que era muy complejo y había otras funcionalidades con igual o mayor prioridad.
    \item Exportar a Word: No se ha realizado porque su implementación es similar a importar a Word.
    \item Sección de ayuda: Aunque consideramos que es importante, no se ha realizado ya que hemos priorizado la calidad del editor y de las adaptaciones, además de la cantidad de adaptaciones.
    \item Configuración general: Para implementar la configuración se necesitaría utilizar cookies de navegador para guardar la información aumentando considerablemente la complejidad de la aplicación. Para cada adaptación habría que guardar información sobre todas sus posibles opciones de configuración. Consideramos que nos iba a llevar mucho tiempo conseguir que funcionara correctamente y preferimos implementar más adaptaciones y mejorar el funcionamiento tanto del editor como de las adaptaciones.
\end{itemize}

La evaluación de la aplicación ha generado nuevas ideas y requisitos para realizar mejoras, basadas en las peticiones y propuestas recibidas por los docentes evaluadores. Estas son algunas de las sugerencias:

\begin{itemize}
    \item Capacidad de reorganizar los ejercicios en el documento de trabajo: Se requiere la opción de mover los ejercicios dentro del documento de manera intuitiva y sencilla.
    \item Funcionalidad de deshacer y rehacer en el documento de trabajo: Es necesario implementar una función que permita al usuario deshacer y rehacer acciones previas, proporcionando una forma de restaurar cambios no deseados.
    \item Añadir botón de ``ayuda'' para realizar las adaptaciones: Se sugiere añadir un botón de ayuda en todas las ventanas modales de las adaptaciones, el cual proporcionará información sobre el uso de la funcionalidad en cuestión.
\end{itemize}

%%%%%%%%%%%%%%%%%%%%%%%%%%%%%%%%%%%%%%%%%%%%%%%%%%%%%%%%%%%%%%%%%%%%%%%%%%%
% Si el TFM se escribe en inglés, comentar las siguientes líneas 
% porque no es necesario incluir nuevamente las Conclusiones en inglés
\setcounter{chapter}{\thechapter-1} 
\begin{otherlanguage}{english}
\chapter{Conclusions and Future Work}
\label{cap:conclusions}


In Section \ref{sec:conclusions}, the conclusions reached after completing the project are explained, and in Section \ref{sec:FutureWork}, possible improvements that could be implemented in the application are described.

\section{Conclusions}
\label{sec:conclusions}
The main objective of this Degree Final Project (TFG) was to develop a web application that facilitates non-significant curricular adaptation for teachers. To achieve this general objective, we set several specific objectives. Next we will see if we have achieved each of these specific objectives:

\begin{itemize}
    \item Analyze AdaptaMaterial 1.0 to identify the requirements that were not covered and the improvements that could be made. As shown in Section \ref{cap:requisitos}, we conducted this analysis and identified the requirements for our application.
    \item Redesign the application. This objective has also been achieved as we completely redesigned the interface, as presented in Section \ref{disenyoDeLaAplicacion}. Additionally, we refactored the application: we transitioned from a serverless architecture to a client-server architecture, migrated from class-based components to functional components, updated the React router to the latest available version, decided not to use Redux or Sass, and started using Tailwind CSS. Lastly, due to the expiration of the CKEditor license used in AdaptaMaterialEscolar 1.0 and the impossibility of renewing it, we implemented the text editor using Slate.
    \item Improvement of existing features and addition of new features. Regarding the development and definitions features present in the previous version (AdaptaMaterialEscolar 1.0), we have incorporated the ability to choose the type of ruling (double, single, grid), and we have also added the option to select a school font type. Additionally, we have introduced several new features, including: generating a summary, exporting the working document to Word and PDF, importing to Word, pictotranslator, content matching exercises, color legend, exercises with space for drawing, and mathematical formulas. All of these features have been implemented except for the import and export to Word.
    \item We conducted an evaluation with end users who provided feedback through a Google Form survey. The survey consisted of three parts: creating an exam model, preparing adapted notes, and a general evaluation of the application. Each task was accompanied by an explanation and an image, followed by three evaluation questions. After the first two parts, a ten-question questionnaire on ease of use was conducted. Open-ended responses were also requested to gather opinions about the tool.

    The results obtained demonstrate that we have exceeded expectations, achieving a notable score of 78.5 on the System Usability Scale (SUS). Furthermore, analyzing Figure \ref{fig:graficaComparativaEjerciciosApuntes}, we can observe that users were more satisfied and comfortable using the features for generating exercises compared to the features for creating notes.
\end{itemize}



In addition, at the beginning of the Degree Final Project (TFG), we set ourselves two academic objectives that we have also achieved:
\begin{itemize}
    \item Apply the knowledge acquired during our degree to this project.
    \begin{itemize}
        \item \textbf{Software Project Management}: This subject focuses on the planning, organization, monitoring, and control of all aspects of a software project, from conception to the final product delivery. In this course, we have learned how to manage resources and time to ensure that projects are completed within established deadlines. Specifically, in this project, we have used the Kanban methodology, which we learned in Software Project Management. This allowed us to create a solid project plan and regularly track progress to ensure the project stays on track.
        \item \textbf{Web Applications}: Teaches how to design and develop efficient and scalable web applications. It covers a wide variety of technologies, from user interface design and creation to database management and application security. Specifically, in this project, we have utilized the knowledge acquired in HTML, CSS, JavaScript, and Node.js.
        \item \textbf{Software Engineering, Software Modeling}: Software Engineering and Software Modeling subjects are important for the development of high-quality software. The former focuses on the principles, practices, and techniques necessary to create efficient and functional software, while the latter focuses on creating accurate and detailed models before implementation to reduce errors and enable faster and more effective implementation. Both subjects cover essential techniques and skills for the software development process. Specifically, in this project, we have applied the acquired knowledge regarding error management and design patterns.
        \item \textbf{Data Structures}: Understanding data structures and data manipulation techniques is essential for the development of high-quality and efficient software. This subject provides a solid foundation for building more effective and scalable software solutions in the future. Specifically, we have applied the acquired knowledge to better understand the structure of Slate.
        \item \textbf{Ethics, Legislation, and Profession}: This subject focuses on the ethical and legal aspects of software engineering, such as data privacy, intellectual property, social and professional responsibility, and software security. It also covers relevant laws and regulations, such as the Personal Data Protection Law and the Intellectual Property Law. Specifically, we have applied the acquired knowledge to understand how to use and manage third-party code, as well as managing the license for our project.
        \item \textbf{Systems and Networks Administration}: This subject focuses on the administration of operating systems, including server and client installation, configuration, and maintenance. It also covers network administration, including router, switch, and firewall configuration, IP address management, and network traffic monitoring. Specifically, we have applied the acquired knowledge to set up the server where the application has been hosted.
    \end{itemize}

    \item Acquire new knowledge.

    During the development of the AdaptaMaterialEscolar 2.0 application, we have acquired new knowledge about the use of technologies such as React, widely used in the professional world. Additionally, we have effectively learned and applied Tailwind CSS and Slate, tools that have contributed to a more efficient and flexible development experience. We have also worked with end-users, identifying and meeting their needs.
\end{itemize}


\section{Future Work}
\label{sec:FutureWork}
After developing the project and accomplishing most of the objectives set initially, it is inevitable that some tasks remain pending for possible future work.

From a requirements perspective, in Section \ref{cap:requisitos}, we already discussed the functionalities that were discarded due to lack of information.

\begin{itemize}
    \item Add image search by keyword.
    \item Substitute a word with an image.
    \item Create an image cropping tool for the original text.
    \item Create tables to organize the syllabus and/or activities, selecting content.
    \item Create diagrams.
    \item Exercises to fill in the blanks in tables and diagrams.
\end{itemize}

On the other hand, there have been certain requirements that have not been carried out due to the priority established among them. Specifically, these requirements are:

\begin{itemize}
    \item Import to Word: We have not implemented this because it would require changing the behavior of the elements handled by Slate. We considered it too complex, and there were other functionalities with equal or higher priority.
    \item Export to Word: The implementation of this functionality has not been carried out due to its similarity to the import to Word function. Just like in the case of importing to Word, we considered that this implementation would be too complex, and there were other functionalities with equal or higher priority.
    \item Help section: Although we consider it important, it has not been implemented as we prioritized the quality of the editor and the adaptations, as well as the quantity of adaptations.
    \item General configuration: To implement the general configuration, it would be necessary to use browser cookies to store the information, significantly increasing the complexity of the application. For each adaptation, information about all its possible configuration options would need to be stored. Additionally, it would involve a significant amount of data management since each adaptation can have multiple configuration options, requiring the storage and management of a large amount of related data. Therefore, we believed it would take a long time to make it work correctly, and we preferred to implement more adaptations and improve the functionality of both the editor and the adaptations.
\end{itemize}

The evaluation of the application has generated improvement ideas and new requirements:

\begin{itemize}
    \item Ability to rearrange exercises: The evaluators missed being able to change the order of the exercises.
    \item Allow undo and redo actions for all operations to make the user feel in control of the application and explore it without fear of making mistakes. This new functionality would also enable the user to easily recover from errors.
    \item Add help to all the features.
    \item Allow the selection of specific words to appear in the summary.
    \item Improve wording regarding punctuation and connectors to provide cohesion in the text.
    \item When translating text into pictograms, articles should be omitted as they are complex to represent through drawings.
    \item Improve exercise editing in the working document.
    \item Allow the generation of non-numbered exercises.
    \item Enhance advanced text editing features.
    \item Automatically assign colors to text based on selected categories in the color legend.
    \item Improve the functionality of math exercises because evaluators found it confusing.
\end{itemize}
\end{otherlanguage}
%%%%%%%%%%%%%%%%%%%%%%%%%%%%%%%%%%%%%%%%%%%%%%%%%%%%%%%%%%%%%%%%%%%%%%%%%%%


% Apéndices
\appendix
\chapter{Título del Apéndice A}
\label{Appendix:Key1}

Los apéndices son secciones al final del documento en las que se agrega texto con el objetivo de ampliar los contenidos del documento principal.
\chapter{Título del Apéndice B}
\label{Appendix:Key2}

Se pueden añadir los apéndices que se consideren oportunos.
%\include{Apendices/appendixC}
%\include{...}
%\include{...}
%\include{...}
\backmatter

%
% Bibliografía
%
% Si el TFM se escribe en inglés, editar TeXiS/TeXiS_bib para cambiar el
% estilo de las referencias
%---------------------------------------------------------------------
%
%                      configBibliografia.tex
%
%---------------------------------------------------------------------
%
% bibliografia.tex
% Copyright 2009 Marco Antonio Gomez-Martin, Pedro Pablo Gomez-Martin
%
% This file belongs to the TeXiS manual, a LaTeX template for writting
% Thesis and other documents. The complete last TeXiS package can
% be obtained from http://gaia.fdi.ucm.es/projects/texis/
%
% Although the TeXiS template itself is distributed under the 
% conditions of the LaTeX Project Public License
% (http://www.latex-project.org/lppl.txt), the manual content
% uses the CC-BY-SA license that stays that you are free:
%
%    - to share & to copy, distribute and transmit the work
%    - to remix and to adapt the work
%
% under the following conditions:
%
%    - Attribution: you must attribute the work in the manner
%      specified by the author or licensor (but not in any way that
%      suggests that they endorse you or your use of the work).
%    - Share Alike: if you alter, transform, or build upon this
%      work, you may distribute the resulting work only under the
%      same, similar or a compatible license.
%
% The complete license is available in
% http://creativecommons.org/licenses/by-sa/3.0/legalcode
%
%---------------------------------------------------------------------
%
% Fichero  que  configura  los  parámetros  de  la  generación  de  la
% bibliografía.  Existen dos  parámetros configurables:  los ficheros
% .bib que se utilizan y la frase célebre que aparece justo antes de la
% primera referencia.
%
%---------------------------------------------------------------------


%%%%%%%%%%%%%%%%%%%%%%%%%%%%%%%%%%%%%%%%%%%%%%%%%%%%%%%%%%%%%%%%%%%%%%
% Definición de los ficheros .bib utilizados:
% \setBibFiles{<lista ficheros sin extension, separados por comas>}
% Nota:
% Es IMPORTANTE que los ficheros estén en la misma línea que
% el comando \setBibFiles. Si se desea utilizar varias líneas,
% terminarlas con una apertura de comentario.
%%%%%%%%%%%%%%%%%%%%%%%%%%%%%%%%%%%%%%%%%%%%%%%%%%%%%%%%%%%%%%%%%%%%%%
\setBibFiles{%
nuestros%
}

%%%%%%%%%%%%%%%%%%%%%%%%%%%%%%%%%%%%%%%%%%%%%%%%%%%%%%%%%%%%%%%%%%%%%%
% Definición de la frase célebre para el capítulo de la
% bibliografía. Dentro normalmente se querrá hacer uso del entorno
% \begin{FraseCelebre}, que contendrá a su vez otros dos entornos,
% un \begin{Frase} y un \begin{Fuente}.
%
% Nota:
% Si no se quiere cita, se puede eliminar su definición (en la
% macro setCitaBibliografia{} ).
%%%%%%%%%%%%%%%%%%%%%%%%%%%%%%%%%%%%%%%%%%%%%%%%%%%%%%%%%%%%%%%%%%%%%%
% \setCitaBibliografia{
% \begin{FraseCelebre}
% \begin{Frase}
%   Y así, del mucho leer y del poco dormir, se le secó el celebro de
%   manera que vino a perder el juicio.\\ 
%   % \textcolor{red}{(modificar en Cascaras$\backslash$bibliografia.tex)}
% \end{Frase}
% \begin{Fuente}
%   Miguel de Cervantes Saavedra
% \end{Fuente}
% \end{FraseCelebre}
% }

%%
%% Creamos la bibliografia
%%
\makeBib

% Variable local para emacs, para  que encuentre el fichero maestro de
% compilación y funcionen mejor algunas teclas rápidas de AucTeX

%%%
%%% Local Variables:
%%% mode: latex
%%% TeX-master: "../Tesis.tex"
%%% End:

%
% Índice de palabras
%

% Sólo  la   generamos  si  está   declarada  \generaindice.  Consulta
% TeXiS.sty para más información.

% En realidad, el soporte para la generación de índices de palabras
% en TeXiS no está documentada en el manual, porque no ha sido usada
% "en producción". Por tanto, el fichero que genera el índice
% *no* se incluye aquí (está comentado). Consulta la documentación
% en TeXiS_pream.tex para más información.
\ifx\generaindice\undefined
\else
%%---------------------------------------------------------------------
%
%                        TeXiS_indice.tex
%
%---------------------------------------------------------------------
%
% TeXiS_indice.tex
% Copyright 2009 Marco Antonio Gomez-Martin, Pedro Pablo Gomez-Martin
%
% This file belongs to TeXiS, a LaTeX template for writting
% Thesis and other documents. The complete last TeXiS package can
% be obtained from http://gaia.fdi.ucm.es/projects/texis/
%
% This work may be distributed and/or modified under the
% conditions of the LaTeX Project Public License, either version 1.3
% of this license or (at your option) any later version.
% The latest version of this license is in
%   http://www.latex-project.org/lppl.txt
% and version 1.3 or later is part of all distributions of LaTeX
% version 2005/12/01 or later.
%
% This work has the LPPL maintenance status `maintained'.
% 
% The Current Maintainers of this work are Marco Antonio Gomez-Martin
% and Pedro Pablo Gomez-Martin
%
%---------------------------------------------------------------------
%
% Contiene  los  comandos  para  generar  el índice  de  palabras  del
% documento.
%
%---------------------------------------------------------------------
%
% NOTA IMPORTANTE: el  soporte en TeXiS para el  índice de palabras es
% embrionario, y  de hecho  ni siquiera se  describe en el  manual. Se
% proporciona  una infraestructura  básica (sin  terminar)  para ello,
% pero  no ha  sido usada  "en producción".  De hecho,  a pesar  de la
% existencia de  este fichero, *no* se incluye  en Tesis.tex. Consulta
% la documentación en TeXiS_pream.tex para más información.
%
%---------------------------------------------------------------------


% Si se  va a generar  la tabla de  contenidos (el índice  habitual) y
% también vamos a  generar el índice de palabras  (ambas decisiones se
% toman en  función de  la definición  o no de  un par  de constantes,
% puedes consultar modo.tex para más información), entonces metemos en
% la tabla de contenidos una  entrada para marcar la página donde está
% el índice de palabras.

\ifx\generatoc\undefined
\else
   \addcontentsline{toc}{chapter}{\indexname}
\fi


% Generamos el índice
\printindex

% Variable local para emacs, para  que encuentre el fichero maestro de
% compilación y funcionen mejor algunas teclas rápidas de AucTeX

%%%
%%% Local Variables:
%%% mode: latex
%%% TeX-master: "./tesis.tex"
%%% End:

\fi

%
% Lista de acrónimos
%

% Sólo  lo  generamos  si  está declarada  \generaacronimos.  Consulta
% TeXiS.sty para más información.


\ifx\generaacronimos\undefined
\else
%---------------------------------------------------------------------
%
%                        TeXiS_acron.tex
%
%---------------------------------------------------------------------
%
% TeXiS_acron.tex
% Copyright 2009 Marco Antonio Gomez-Martin, Pedro Pablo Gomez-Martin
%
% This file belongs to TeXiS, a LaTeX template for writting
% Thesis and other documents. The complete last TeXiS package can
% be obtained from http://gaia.fdi.ucm.es/projects/texis/
%
% This work may be distributed and/or modified under the
% conditions of the LaTeX Project Public License, either version 1.3
% of this license or (at your option) any later version.
% The latest version of this license is in
%   http://www.latex-project.org/lppl.txt
% and version 1.3 or later is part of all distributions of LaTeX
% version 2005/12/01 or later.
%
% This work has the LPPL maintenance status `maintained'.
% 
% The Current Maintainers of this work are Marco Antonio Gomez-Martin
% and Pedro Pablo Gomez-Martin
%
%---------------------------------------------------------------------
%
% Contiene  los  comandos  para  generar  el listado de acr�nimos
% documento.
%
%---------------------------------------------------------------------
%
% NOTA IMPORTANTE:  para que la  generaci�n de acr�nimos  funcione, al
% menos  debe  existir  un  acr�nimo   en  el  documento.  Si  no,  la
% compilaci�n  del   fichero  LaTeX  falla  con   un  error  "extra�o"
% (indicando  que  quiz�  falte  un \item).   Consulta  el  comentario
% referente al paquete glosstex en TeXiS_pream.tex.
%
%---------------------------------------------------------------------


% Redefinimos a espa�ol  el t�tulo de la lista  de acr�nimos (Babel no
% lo hace por nosotros esta vez)

\def\listacronymname{Lista de acr�nimos}

% Para el glosario:
% \def\glosarryname{Glosario}

% Si se  va a generar  la tabla de  contenidos (el �ndice  habitual) y
% tambi�n vamos a  generar la lista de acr�nimos  (ambas decisiones se
% toman en  funci�n de  la definici�n  o no de  un par  de constantes,
% puedes consultar config.tex  para m�s informaci�n), entonces metemos
% en la  tabla de contenidos una  entrada para marcar  la p�gina donde
% est� el �ndice de palabras.

\ifx\generatoc\undefined
\else
   \addcontentsline{toc}{chapter}{\listacronymname}
\fi


% Generamos la lista de acr�nimos (en realidad el �ndice asociado a la
% lista "acr" de GlossTeX)

\printglosstex(acr)

% Variable local para emacs, para  que encuentre el fichero maestro de
% compilaci�n y funcionen mejor algunas teclas r�pidas de AucTeX

%%%
%%% Local Variables:
%%% mode: latex
%%% TeX-master: "../Tesis.tex"
%%% End:

\fi

%
% Final
%
%---------------------------------------------------------------------
%
%                      fin.tex
%
%---------------------------------------------------------------------
%
% fin.tex
% Copyright 2009 Marco Antonio Gomez-Martin, Pedro Pablo Gomez-Martin
%
% This file belongs to the TeXiS manual, a LaTeX template for writting
% Thesis and other documents. The complete last TeXiS package can
% be obtained from http://gaia.fdi.ucm.es/projects/texis/
%
% Although the TeXiS template itself is distributed under the 
% conditions of the LaTeX Project Public License
% (http://www.latex-project.org/lppl.txt), the manual content
% uses the CC-BY-SA license that stays that you are free:
%
%    - to share & to copy, distribute and transmit the work
%    - to remix and to adapt the work
%
% under the following conditions:
%
%    - Attribution: you must attribute the work in the manner
%      specified by the author or licensor (but not in any way that
%      suggests that they endorse you or your use of the work).
%    - Share Alike: if you alter, transform, or build upon this
%      work, you may distribute the resulting work only under the
%      same, similar or a compatible license.
%
% The complete license is available in
% http://creativecommons.org/licenses/by-sa/3.0/legalcode
%
%---------------------------------------------------------------------
%
% Contiene la última página
%
%---------------------------------------------------------------------


% Ponemos el marcador en el PDF
\ifpdf
   \pdfbookmark{Fin}{fin}
\fi

\thispagestyle{empty}\mbox{}

\vspace*{4cm}

\small

% \hfill \emph{--¿Qué te parece desto, Sancho? -- Dijo Don Quijote --}

% \hfill \emph{Bien podrán los encantadores quitarme la ventura,}

% \hfill \emph{pero el esfuerzo y el ánimo, será imposible.}

% \hfill 

% \hfill \emph{Segunda parte del Ingenioso Caballero} 

% \hfill \emph{Don Quijote de la Mancha}

% \hfill \emph{Miguel de Cervantes}

% \vfill%space*{4cm}

% \hfill \emph{--Buena está -- dijo Sancho --; fírmela vuestra merced.}

% \hfill \emph{--No es menester firmarla -- dijo Don Quijote--,}

% \hfill \emph{sino solamente poner mi rúbrica.}

% \hfill 

% \hfill \emph{Primera parte del Ingenioso Caballero} 

% \hfill \emph{Don Quijote de la Mancha}

% \hfill \emph{Miguel de Cervantes}


\newpage
\thispagestyle{empty}\mbox{}

\newpage

% Variable local para emacs, para  que encuentre el fichero maestro de
% compilación y funcionen mejor algunas teclas rápidas de AucTeX

%%%
%%% Local Variables:
%%% mode: latex
%%% TeX-master: "../Tesis.tex"
%%% End:

%\end{otherlanguage}
\end{document}
