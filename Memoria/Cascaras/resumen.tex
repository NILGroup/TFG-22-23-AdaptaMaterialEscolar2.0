\chapter{Resumen}

El currículum educativo tiene como objetivo garantizar la igualdad de oportunidades en el proceso de formación del alumnado, proporcionando los materiales, recursos y contenidos necesarios a lo largo de su trayectoria académica. No obstante, debido a las distintas necesidades educativas de los alumnos, en ocasiones es necesario realizar adaptaciones curriculares. Lamentablemente, en la actualidad hay herramientas que ofrecen adaptaciones curriculares, pero, en su mayoría, se centran en un tipo de adaptación concreta. Esto implica que los docentes inviertan una cantidad significativa de tiempo en la creación de material académico adaptado para aquellos estudiantes que lo necesiten. Tienen que realizar ajustes textuales, buscar imágenes, crear ejercicios, etc.

Este proyecto trata de ofrecer una centralización de las herramientas de adaptación curricular. Consiste en la mejora y actualización de una aplicación web creada en un TFG del curso 20/21, AdaptaMaterialEscolar 1.0, desarrollada específicamente para simplificar el proceso de adaptación curricular no significativa para los docentes. Hemos rediseñado la aplicación original, hemos mejorado algunas de las funcionalidades existentes y añadido otras nuevas. Las funcionalidades actualizadas en AdaptaMaterialEscolar incluyen un buscador de pictogramas, ejercicios de completar huecos en blanco, generación de sopas de letras, preguntas de verdadero o falso, definición de conceptos, exportación a PDF y preguntas que requieren respuestas desarrolladas en un espacio limitado. Además, hemos desarrollado nuevas funcionalidades que amplían las capacidades de la aplicación. Estas incluyen ejercicios de definiciones, relacionar conceptos, integración de fórmulas matemáticas con huecos, espacio para dibujar, leyenda de colores, generación de resúmenes y un pictotraductor.

La aplicación web AdaptaMaterialEscolar 2.0 ha sido evaluada por usuarios finales que tienen la intención de utilizarla durante el período escolar. Los resultados obtenidos demuestran que la aplicación satisface una necesidad real del profesorado y demuestran que su nivel de usabilidad es altamente satisfactorio.

La aplicación puede usarse en \href{http://143.47.43.25/}{AdaptaMaterialEscolar 2.0} y el código se encuentra en \url{https://github.com/NILGroup/TFG-22-23-AdaptaMaterialEscolar2.0}.

\chapter{Palabras Clave}
\cabeceraEspecial{Palabras Clave}

Adaptación curricular no significativa

Accesibilidad

Herramienta docente

Editor de texto

Aplicación web

Material académico

Necesidades educativas especiales

Tecnología educativa

Innovación educativa



