\chapter{Plan de pruebas: Casos de prueba y guiones de prueba}\label{ape:pruebas}

En este Apéndice se exponen los distintos casos de prueba y guiones que se han definido para probar cada funcionalidad de AdaptaMaterialEscolar 2.0.

\section{Funcionalidad de verdadero/falso}
\label{planPruebas:v/f}
A continuación se expondrá una tabla con los casos de prueba, escenarios específicos para verificar si la funcionalidad cumple con los requisitos y funcionalidades establecidas, y el procedimiento para hacer dichos casos.
\subsection{Casos de Prueba}
\label{casosPruebas:v/f}

\begin{table}[H]
    \centering
    \scalebox{0.85}{
    \begin{tabular}{| c | c | p{4cm} |  c |c |}
    \hline
    \textbf{Precondición} &  \textbf{Campo} &  \textbf{Condición} &  \textbf{Datos de entrada} &  \textbf{\begin{tabular}[c]{@{}l@{}}salida esperada\\ (postconticion)\end{tabular}} \\ \hline
    Lista=[] & Nueva frase & Longitud de frase <= 0 & Frase = “” & Lista = []\\ \hline
    Lista=[] & Nueva frase & Longitud de frase > 0 & Frase = “hola” & Lista = [“hola”]\\ \hline
    Lista=[“hola”] & Editar frase &Longitud de frase <= 0 y la Frase existe en la lista & Frase = “” & Lista = [“hola”]\\ \hline
    Lista=[“hola”] & Editar frase & Longitud de frase > 0 y la Frase existe en la lista & Frase = “adios” & Lista = [“adios”]\\ \hline
    Lista=[“hola”] & Borrar frase & Existe la frase a borrar en la lista & Frase = “hola” & Lista = []\\ \hline
    Lista=[“hola”, “adios”] & Reordenar frases &  & Lista=[“hola”, “adios”] & Lista = [“adios”, “hola”]\\ \hline
    Lista=[] & Reordenar frases &  & Lista=[] & Lista=[] \\ \hline
    Lista=[“hola”] & Reordenar frases &  & Lista=[“hola”] & Lista=[“hola”]\\ \hline
    \end{tabular}
    }
   \caption{Tabla de ejemplo}
    \end{table}

\subsection{Procedimiento de prueba}
\label{procedimientoPruebas:v/f}
\begin{enumerate}
    \item Abrir el modal de Verdadero/Falso
    \item Cerrar el modal
    \item Abrir el modal de Verdadero/Falso
    \item Comprobar que no está habilitado el botón de \item “Reordenar” ni el botón de “Ok”
    \item No introducir datos al insertar frase y dar click a añadir
    \item Comprobar que no ha producido ningún cambio
    \item Introducir datos, ejemplo “hola” y dar click a añadir
    \item Comprobar que se ha introducido en la lista de frases
    \item Editar la palabra introducida introduciendo “”
    \item Comprobar que no se ha cambiado el valor
    \item Editar la palabra introducida introduciendo “adios”
    \item Comprobar que se ha cambiado el valor al esperado
    \item Darle a borrar la frase
    \item Comprobar que se ha borrado
    \item Introducir datos, ejemplo “pepe”y dar click a añadir
    \item Comprobar que se ha introducido en la lista de frases
    \item Comprobar que no está habilitado el botón de reordenar
    \item Introducir datos, ejemplo “gato”y dar click a añadir
    \item Comprobar que se ha introducido en la lista de frases
    \item Darle a reordenar 
    \item Comprobar que no tiene el orden anterior
    \item Darle a ok
    \item Comprobar que se ha cerrado el modal
    \item Abrir el modal de Verdadero/Falso
    \item Comprobar que los campos están vacíos
  \end{enumerate}
