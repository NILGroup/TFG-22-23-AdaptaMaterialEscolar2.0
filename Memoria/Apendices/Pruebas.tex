\addtocontents{toc}{\protect\setcounter{tocdepth}{0}}
\chapter{Plan de pruebas: Casos de prueba y guiones de prueba}\label{ape:pruebas}

En este Apéndice se exponen los distintos casos de prueba, que son escenarios específicos para verificar si la funcionalidad cumple con los requisitos y funcionalidades establecidas, y guiones que se han definido para probar cada funcionalidad de AdaptaMaterialEscolar 2.0.

\section{Funcionalidad de verdadero/falso}
\label{planPruebas:v/f}
A continuación se expone la tabla (Figura \ref{tab:v/f}) con los casos de prueba de la funcionalidad de verdadero/falso y el procedimiento para hacer dichos casos.

\begin{table}[H]
    \resizebox{1.1\textwidth}{!}{
    \begin{tabular}{|c|c|c|c|c|}
    \hline
    \textbf{Precondición} & \textbf{Campo} & \textbf{Condición} & \textbf{Datos de entrada} & \textbf{\begin{tabular}[c]{@{}c@{}}Salida esperada\\ (Postcondición)\end{tabular}} \\ \hline
    Lista={[} {]} & Nueva frase & Longitud de frase \textless{}= 0 & Frase = “” & Lista = {[} {]} \\ \hline
    Lista={[} {]} & Nueva frase & Longitud de frase \textgreater 0 & Frase = “hola” & Lista = {[}“hola” {]} \\ \hline
    Lista={[}“hola”{]} & Editar frase & Longitud de frase \textless{}= 0 y la Frase existe en la lista & Frase = “” & Lista = {[}“hola”{]} \\ \hline
    Lista={[}“hola”{]} & Editar frase & Longitud de frase \textgreater 0 y la Frase existe en la lista & Frase = “adios” & Lista = {[}“adios”{]} \\ \hline
    Lista={[}“hola”{]} & Borrar frase & Existe la frase a borrar en la lista & Frase = “hola” & Lista = {[} {]} \\ \hline
    Lista={[}“hola”, “adios”{]} & Reordenar frases &  & Lista={[}“hola”, “adios”{]} & Lista = {[}“adios”, “hola”{]} \\ \hline
    Lista={[}{]} & Reordenar frases &  & Lista={[}{]} & Lista={[}{]} \\ \hline
    Lista={[}“hola”{]} & Reordenar frases &  & Lista={[}“hola”{]} & Lista={[}“hola”{]} \\ \hline
    \end{tabular}
    }
    \caption{Casos de prueba de la funcionalidad de verdadero/falso.}
    \label{tab:v/f}
\end{table}

\subsection{Procedimiento de prueba}
\label{procedimientoPruebas:v/f}
\begin{enumerate}
    \item Abrir el modal de Verdadero/Falso.
    \item Cerrar el modal.
    \item Abrir el modal de Verdadero/Falso.
    \item Comprobar que no está habilitado el botón de ``Reordenar'' ni el botón de ``O.k''.
    \item No introducir datos al insertar frase y dar click a añadir.
    \item Comprobar que no ha producido ningún cambio.
    \item Introducir datos, ejemplo ``hola''  y dar click a añadir.
    \item Comprobar que se ha introducido en la lista de frases.
    \item Editar la palabra introducida introduciendo una cadena vacía.
    \item Comprobar que no se ha cambiado el valor.
    \item Editar la palabra introducida introduciendo .“adios”.
    \item Comprobar que se ha cambiado el valor al esperado
    \item Darle a borrar la frase.
    \item Comprobar que se ha borrado.
    \item Introducir datos, ejemplo ``pepe'' dar click a añadir.
    \item Comprobar que se ha introducido en la lista de frases.
    \item Comprobar que no está habilitado el botón de reordenar.
    \item Introducir datos, ejemplo ``gato'' y dar click a añadir.
    \item Comprobar que se ha introducido en la lista de frases.
    \item Darle a reordenar .
    \item Comprobar que no tiene el orden anterior.
    \item Darle a ``Ok''.
    \item Comprobar que se ha cerrado el modal.
    \item Abrir el modal de Verdadero/Falso.
    \item Comprobar que los campos están vacíos.
\end{enumerate}

\subsection{Errores encontrados}
\label{errores:v/f}
\begin{itemize}
    \item Editar una palabra hace que se duplique en la lista previa.
    \item Al pulsar el boton de reordenar, puede devolver el mismo orden.
    \item No se reinicia el estado cuando se cierra el modal.
    \item No se puedan editar frases en el documento de trabajo.
\end{itemize}

\section{Funcionalidad de buscar pictograma}
\label{planPruebas:busPicto}
A continuación se expone la tabla (Figura \ref{tab:busPicto}) con los casos de prueba de la funcionalidad de buscar pictograma y el procedimiento para hacer dichos casos.

% Please add the following required packages to your document preamble:
% \usepackage{graphicx}
\begin{table}[H]
    \resizebox{1.1\textwidth}{!}{
    \begin{tabular}{|c|c|c|c|c|}
    \hline
    \textbf{Precondición} & \textbf{Campo} & \textbf{Condición} & \textbf{Datos de entrada} & \textbf{\begin{tabular}[c]{@{}c@{}}Salida esperada\\ (Postcondición)\end{tabular}} \\ \hline
    ListaPicto={[}{]} & Nuevo concepto & Longitud del concepto \textless{}= 0 & concepto= “” & ListaPicto= {[} {]} \\ \hline
    ListaPicto={[}{]} & Nuevo concepto & Longitud del concepto \textgreater 0 & concepto= “hola” & ListaPicto= {[}pictograma1,pictograma2…{]} \\ \hline
    ListaPicto={[}{]} & Nuevo concepto & Longitud del concepto \textgreater 0 & concepto=”jhfueh” & ListaPicto={[}{]}, mensaje=”no se ha encontrado pictograma” \\ \hline
    \end{tabular}
    }
    \caption{Casos de prueba de la funcionalidad de buscar pictograma.}
    \label{tab:busPicto}
\end{table}

\subsection{Procedimiento de prueba}
\label{procedimientoPruebas:busPicto}
\begin{enumerate}
    \item Abrir el modal de Buscar pictograma.
    \item Cerrar el modal.
    \item Abrir el modal de Buscar pictograma.
    \item No introducir concepto y dar click a buscar.
    \item Comprobar que no ha producido ningún cambio.
    \item Introducir un concepto y darle a buscar.
    \item Comprobar que salen los pictogramas relacionados con el concepto.
    \item Darle a un pictograma y comprobar que se ha cerrado el modal.
    \item Abrir el modal de Buscar pictograma.
    \item Comprobar que los campos están vacíos.

\end{enumerate}

\section{Funcionalidad de desarrollo}
\label{planPruebas:desarrollo}
A continuación se expone la tabla (Figura \ref{tab:desarrollo}) con los casos de prueba de la funcionalidad de desarrollo y el procedimiento para hacer dichos casos.

\begin{table}[H]
    \resizebox{1.1\textwidth}{!}{
    \begin{tabular}{|c|c|c|c|c|}
    \hline
    \textbf{Precondición} & \textbf{Campo} & \textbf{Condición} & \textbf{Datos de entrada} & \textbf{\begin{tabular}[c]{@{}c@{}}Salida esperada\\ (Postcondición)\end{tabular}} \\ \hline
    Enunciado=”” & Insertar enunciado & Longitud de enunciado \textless{}= 0 & Enunciado= “” & Enunciado=”” \\ \hline
    Enunciado=”” & Insertar enunciado & Longitud de enunciado \textgreater 0 & Enunciado= “hola” & Enunciado=“hola” \\ \hline
    Enunciado=“hola” & Insertar enunciado & Longitud de enunciado \textless{}= 0 & Enunciado= “” & Enunciado=“” \\ \hline
    Enunciado=“hola” & Insertar enunciado & Longitud de enunciado \textgreater 0 & Enunciado= “adios” & Enunciado= “adios” \\ \hline
    NumeroFilas=1 & Insertar número de filas & Número de filas\textless{}= 0 y existe un enunciado & NumeroFilas= -1 & Enunciado=”” \\ \hline
    NumeroFilas=1 & Insertar número de filas & Número de filas \textgreater 0 y existe un enunciado & NumeroFilas= 2 & Enunciado != “” \\ \hline
    Opción pauta=”” & Elegir pauta & Número de filas \textgreater 0 y existe un enunciado & Opción pauta!= “” & Opción pauta!=”” \\ \hline
    \end{tabular}
    }
    \caption{Casos de prueba de la funcionalidad de desarrollo.}
    \label{tab:desarrollo}
\end{table}

\subsection{Procedimiento de prueba}
\label{procedimientoPruebas:desarrollo}
\begin{enumerate}
    \item Abrir el modal de desarrollo.
    \item Cerrar el modal.
    \item Abrir el modal de desarrollo.
    \item Comprobar que no está habilitado el botón de ``Ok''.
    \item No introducir datos al insertar enunciado y comprobar que no hay nada en la vista previa.
    \item Introducir número de filas mayor que cero y comprobar que no hay nada en la vista previa.
    \item Probar tipos de pauta y comprobar que no hay nada en la vista previa.
    \item Introducir número de filas menor que cero y comprobar que no hay nada en la vista previa.
    \item Probar tipos de pauta y comprobar que no hay nada en la vista previa.
    \item Introducir texto en el insertar enunciado, introducir número de filas mayor que cero y comprobar que se ve en la vista previa .
    \item Probar tipos de pauta y comprobar que se ve en la visa previa.
    \item Probar a cambiar enunciado y comprobar que el cambio se refleja en la vista previa.
    \item Darle a ``Ok''.
    \item Comprobar que se ha cerrado el modal.
    \item Abrir el modal de desarrollo.
    \item Comprobar que los campos están vacíos.
\end{enumerate}

\subsection{Errores encontrados}
\label{errores:desarrollo}
\begin{itemize}
    \item No se debe poder darle a ``Ok'' sin enunciado o con numero de filas $\leq$ 0.
    \item Enunciado con número de filas > 0 provoca un error en consola tras darle ``Ok''.
    \item No debe aparecer pauta si no hay enunciado.
    \item Si se da a la primera opción de pauta no va.
    \item Si el número de filas es un número muy alto como 1000 la página se bloquea.
    \item Al darle a la ``x'' no se resetea.
    \item Boton de ``Ok'' no está desactivado.
    \item  Al añadir la cuadrícula al editable con filas mayor que 1 en la última cuadrícula aparece sin el borde.

\end{itemize}

\section{Funcionalidad de sopa de letras}
\label{planPruebas:sopa}
A continuación se expone la tabla (Figura \ref{tab:sopa}) con los casos de prueba de la funcionalidad de sopa de letras y el procedimiento para hacer dichos casos.

\begin{table}[H]
    \resizebox{1.1\textwidth}{!}{%
    \begin{tabular}{|c|c|c|c|c|}
    \hline
    \textbf{Precondición} & \textbf{Campo} & \textbf{Condición} & \textbf{Datos de entrada} & \textbf{\begin{tabular}[c]{@{}c@{}}Salida esperada\\ (Postcondición)\end{tabular}} \\ \hline
    Listapalabra={[}{]} & Insertar palabra & Longitud de palabra \textless{}= 0 & Palabra= “” & Listapalabra={[}{]} \\ \hline
    Listapalabra={[}{]} & Insertar palabra & Longitud de palabra\textgreater 0 & Palabra= “hola” & Listapalabra={[}“hola”{]} \\ \hline
    Listapalabra={[}“Hola”{]} & Insertar palabra & Longitud de palabra\textless{}= 0 & Palabra= “” & Listapalabra={[}“Hola”{]} \\ \hline
    Listapalabra={[}“Hola”{]} & Insertar palabra & Longitud de palabra\textgreater 0 & palabra= “adios” & Listapalabra={[}“Hola”,”adios”{]} \\ \hline
    Listapalabra={[}“Hola”{]} & Editar palabra & \begin{tabular}[c]{@{}c@{}}Longitud de palabra\textless{}= 0 y \\ la palabra existe en la lista\end{tabular} & palabra= “” & Listapalabra={[}“Hola”{]} \\ \hline
    Listapalabra={[}“Hola”{]} & Editar palabra & \begin{tabular}[c]{@{}c@{}}Longitud de palabra\textgreater 0\\  y la palabra existe en la lista\end{tabular} & palabra= “adios” & Listapalabra= {[}“adios”{]} \\ \hline
    Listapalabra={[}“Hola”{]} & Borrar palabra & Existe la palabra borrar en la lista & palabra= “hola” & Listapalabra= {[} {]} \\ \hline
    NumFilas=1\&\& NumCols=1 & \begin{tabular}[c]{@{}c@{}}Insertar número de\\  filas y columnas\end{tabular} & NumFilas y numCols\textless{}= 0 & \begin{tabular}[c]{@{}c@{}}NumFilas= 0\\ NumCols=0\end{tabular} & NumFilas=1\&\& NumCols=1 \\ \hline
    NumFilas=1\&\& NumCols=1 & \begin{tabular}[c]{@{}c@{}}Insertar número de \\ filas y columnas\end{tabular} & NumFilas y numCols\textgreater 0 \textless 1000 & \begin{tabular}[c]{@{}c@{}}NumFilas= 3\\ NumCols=8\end{tabular} & NumFilas=3 NumCols=8 \\ \hline
    NumFilas=1\&\& NumCols=1 & \begin{tabular}[c]{@{}c@{}}Insertar número de \\ filas y columnas\end{tabular} & NumFilas =1000 numCols=1000 & \begin{tabular}[c]{@{}c@{}}NumFilas= 1000\\ NumCols=1000\end{tabular} & sopa de letras correctamente creada \\ \hline
    \begin{tabular}[c]{@{}c@{}}NumFilas=1\&\& NumCols=1 \&\& \\ Listapalabra={[}“Hola”,”adios”{]} \&\& \\ posicionamiento=horizontal\end{tabular} & \begin{tabular}[c]{@{}c@{}}Insertar número de \\ filas y columnas\end{tabular} & NumFilas\textgreater{}0 numCols\textgreater{}0 & Número de filas=2 numCols=4 & sopa de letras correctamente creada \\ \hline
    \begin{tabular}[c]{@{}c@{}}NumFilas=1\&\& NumCols=1 \&\& \\ Listapalabra={[}“Hola”,”adios”{]} \&\& \\ posicionamiento=horizontal\end{tabular} & \begin{tabular}[c]{@{}c@{}}Insertar número de \\ filas y columnas\end{tabular} & \begin{tabular}[c]{@{}c@{}}NumFilas\textgreater{}0 \&\& \\ 0\textless numCols \\ \textless{}longitud de la palabra más larga\end{tabular} & Número de filas=2 numCols=3 & Error \\ \hline
    \begin{tabular}[c]{@{}c@{}}NumFilas=1\&\& NumCols=1 \&\& \\ Listapalabra={[}“Hola”,”adios”{]} \&\& \\ posicionamiento=vertical\end{tabular} & \begin{tabular}[c]{@{}c@{}}Insertar número de \\ filas y columnas\end{tabular} & \begin{tabular}[c]{@{}c@{}}NumFilas=número de \\ palabras a insertar larga \&\& \\ numCols \textgreater{}0\end{tabular} & Número de filas=2 numCols=4 & sopa de letras correctamente creada \\ \hline
    \begin{tabular}[c]{@{}c@{}}NumFilas=1\&\& NumCols=1 \&\& \\ Listapalabra={[}“Hola”,”adios”{]} \&\& \\ posicionamiento=vertical\end{tabular} & \begin{tabular}[c]{@{}c@{}}Insertar número de \\ filas y columnas\end{tabular} & \begin{tabular}[c]{@{}c@{}}NumFilas\textless{}número de \\ palabras a insertar larga \\ \&\& numCols \textgreater{}0\end{tabular} & Número de filas=1 numCols=4 & Error \\ \hline
    \begin{tabular}[c]{@{}c@{}}NumFilas=1\&\& NumCols=1 \&\& \\ Listapalabra={[}“Hola”,”adios”{]} \&\& \\ posicionamiento=diagonal\end{tabular} & \begin{tabular}[c]{@{}c@{}}Insertar número de \\ filas y columnas\end{tabular} & \begin{tabular}[c]{@{}c@{}}NumFilas=longitud de la \\ palabra más larga\\  larga \&\& numCols=longitud de la \\ palabra más larga\end{tabular} & Número de filas=4 numCols=4 & sopa de letras correctamente creada \\ \hline
    \begin{tabular}[c]{@{}c@{}}NumFilas=1\&\& NumCols=1 \&\& \\ Listapalabra={[}“Hola”,”adios”{]} \&\& \\ posicionamiento=diagonal\end{tabular} & \begin{tabular}[c]{@{}c@{}}Insertar número de \\ filas y columnas\end{tabular} & \begin{tabular}[c]{@{}c@{}}NumFilas\textless{}longitud de la \\ palabra más larga\\  larga \&\& numCols\textless{}longitud de la \\ palabra más larga\end{tabular} & Número de filas=3 numCols=3 & Error \\ \hline
    \end{tabular}%
    }
    \caption{Casos de prueba de la funcionalidad de sopa de letras.}
    \label{tab:sopa}
\end{table}

\subsection{Procedimiento de prueba}
\label{procedimientoPruebas:sopa}
\begin{enumerate}
    \item Abrir el modal de sopa de letras.
    \item Cerrar el modal.
    \item Abrir el modal de sopa de letras.
    \item Comprobar que no está habilitado el botón de ``Ok''.
    \item No introducir datos al insertar las palabras y dar click a añadir.
    \item Comprobar que no ha producido ningún cambio.
    \item Introducir número de filas y columnas mayor que uno.
    \item Comprobar que no ha producido ningún cambio.
    \item Introducir uno o varios posicionamientos de la palabra en la sopa de letras.
    \item Comprobar que no ha producido ningún cambio.
    \item Introducir datos, ejemplo ``pepe''y dar click a añadir.
    \item Comprobar que se ha introducido en la lista de palabras.
    \item Editar palabra introduciendo una cadena vacía.
    \item Comprobar que no ha producido ningún cambio.
    \item Editar palabra introduciendo ``juan''.
    \item Comprobar que la palabra se ha modificado por ``juan''.
    \item Darle a borrar la palabra.
    \item Comprobar que se ha borrado.
    \item Introducir un posicionamiento horizontal e introducir tantas columnas como la longitud de la palabra.
    \item Comprobar que se ha creado la sopa de letras.
    \item Cambia a un posicionamiento vertical e introducir tantas filas como la longitud de la palabra.
    \item Comprobar que se ha creado la sopa de letras.
    \item Cambia a un posicionamiento diagonal e introducir tantas filas y columnas como la longitud de la palabra.
    \item Comprobar que se ha creado la sopa de letras.
    \item Cambia a un posicionamiento al revés e introducir tantas columnas como la longitud de la palabra.
    \item Comprobar que se ha creado la sopa de letras.
    \item Seleccionar todos los tipos de posicionamiento y adecuar el número de columnas y de filas.
    \item Comprobar que se ha creado la sopa de letras.
    \item Darle a ``Ok''.
    \item Comprobar que se ha cerrado el modal.
    \item Abrir el modal de sopa de letras.
    \item Comprobar que los campos están vacíos.
\end{enumerate}

\subsection{Errores encontrados}
\label{errores:sopa}
\begin{itemize}
    \item Si pones 1000 tanto en el número de columnas como en el de filas la pagina no responde.
    \item Al añadir la sopa de letras en el documento de trabajo, posicionar el cursor después del enunciado y dar enter, se crean tantas sopas de letras como veces se da al enter.
    \item Si se ponen 3 números para la filas o las columnas, por ejemplo 300, solo se puede ver los dos primeros números.
\end{itemize}

\section{Funcionalidad de completar huecos}
\label{planPruebas:huecos}
A continuación se expone la tabla (Figura \ref{tab:huecos}) con los casos de prueba de la funcionalidad de completar huecos y el procedimiento para hacer dichos casos.

\begin{table}[H]
    \resizebox{1.1\textwidth}{!}{
    \begin{tabular}{|c|c|c|c|c|}
    \hline
    \textbf{Precondición} & \textbf{Campo} & \textbf{Condición} & \textbf{Datos de entrada} & \textbf{\begin{tabular}[c]{@{}c@{}}Salida esperada\\ (Postcondición)\end{tabular}} \\ \hline
    Texto=”” & Nuevo texto & Longitud de texto\textless{}= 0 & Texto= “” & Texto= “” \\ \hline
    Texto=”” & Nuevo texto & Longitud de texto\textgreater 0 & Texto= “hola” & Texto=“hola” \\ \hline
    Texto =”” & Introducir huecos & Longitud de texto\textless{}= 0 & \begin{tabular}[c]{@{}c@{}}Tamaño de hueco \\ que pertenece \\ a algún tamaño de \\ hueco disponible\end{tabular} & Texto=”” \\ \hline
    Texto =”Hola esto es una prueba” & Introducir huecos & \begin{tabular}[c]{@{}c@{}}Longitud de texto\textgreater 0\\  y el hueco no \\ sea un espacio\end{tabular} & \begin{tabular}[c]{@{}c@{}}Tamaño de hueco \\ que pertenece \\ a algún tamaño de \\ hueco disponible\end{tabular} & Texto =”\_\_\_\_\_\_\_ esto es una \_\_\_\_\_\_\_” \\ \hline
    Texto =”Hola esto es una prueba” & Editar Texto & Longitud de texto\textgreater 0 & Texto =”Adiós esto es una prueba”” & Texto =”Adiós esto es una prueba”” \\ \hline
    \end{tabular}
    }
    \caption{Casos de prueba de la funcionalidad de completar huecos.}
    \label{tab:huecos}
\end{table}

\subsection{Procedimiento de prueba}
\label{procedimientoPruebas:huecos}
\begin{enumerate}
    \item Abrir el modal de completar huecos.
    \item Cerrar el modal.
    \item Abrir el modal de completar huecos.
    \item Comprobar que sin texto introducido no se puede darle a ``Ok''.
    \item Comprobar que sin texto introducido no se puede darle a añadir huecos.
    \item Insertar texto como por ejemplo ``Hola esto es una prueba'' comprobar que se puede dar click al botón de añadir hueco.
    \item Borrar texto y ver como no se puede darle a ok y no se puede darle a añadir huecos.
    \item Insertar texto como por ejemplo ``Hola esto es una prueba'' comprobar que se puede dar click al botón de añadir hueco.
    \item Dar click al botón de añadir hueco y probar todos los tipos de añadir hueco.
    \item Probar a dar click al boton de editar texto e introducir por ejemplo ``Adiós esto es una prueba''.
    \item Dar click al botón de añadir hueco y probar todos los tipos de añadir hueco.
    \item Comprobar que no puedes convertir los espacios en huecos.
    \item Dar click al ``Ok''.
    \item Comprobar que se ha cerrado el modal.
    \item Abrir el modal de completar huecos.
    \item Comprobar que los campos están vacíos.
\end{enumerate}

\subsection{Errores encontrados}
\label{errores:huecos}
\begin{itemize}
    \item Permite convertir los espacios en huecos.
    \item Si hay una palabra con tilde no te crea bien el hueco.
\end{itemize}

\section{Funcionalidad de definiciones}
\label{planPruebas:definiciones}
A continuación se expone la tabla (Figura \ref{tab:definiciones}) con los casos de prueba de la funcionalidad de definiciones y el procedimiento para hacer dichos casos.

\begin{table}[H]
    \resizebox{1.1\textwidth}{!}{
    \begin{tabular}{|c|c|c|c|c|}
    \hline
    \textbf{Precondición} & \textbf{Campo} & \textbf{Condición} & \textbf{Datos de entrada} & \textbf{\begin{tabular}[c]{@{}c@{}}Salida esperada\\ (Postcondición)\end{tabular}} \\ \hline
    Lista={[} {]} & Nueva definición & Longitud de frase \textless{}= 0 & Frase = “” & Lista = {[} {]} \\ \hline
    Lista={[} {]} & Nueva definición & Longitud de frase \textgreater 0 & Frase = “hola” & Lista = {[}“hola” {]} \\ \hline
    Lista={[}“hola”{]} & Editar definición & Longitud de frase \textless{}= 0 y la Frase existe en la lista & Frase = “” & Lista = {[}“hola”{]} \\ \hline
    Lista={[}“hola”{]} & Editar definición & \begin{tabular}[c]{@{}c@{}}Longitud de frase \textgreater 0\\  y la Frase existe en la lista\end{tabular} & Frase = “adios” & Lista = {[}“adios”{]} \\ \hline
    Lista={[}“hola”{]} & Borrar definición & Existe la frase a borrar en la lista & Frase = “hola” & Lista = {[} {]} \\ \hline
    NumeroFilas=1 & Insertar número de filas & Número de filas\textless{}= 0 y existe un enunciado & NumeroFilas= -1 & Enunciado=”” \\ \hline
    NumeroFilas=1 & Insertar número de filas & Número de filas \textgreater 0 y existe un enunciado & NumeroFilas= 2 & Enunciado != “” \\ \hline
    Opción pauta=”” & Elegir pauta & Número de filas \textgreater 0 y existe un enunciado & Opción pauta!= “” & Opción pauta!=”” \\ \hline
    \end{tabular}
    }
    \caption{Casos de prueba de la funcionalidad de definiciones.}
    \label{tab:definiciones}
\end{table}

\subsection{Procedimiento de prueba}
\label{procedimientoPruebas:definiciones}
\begin{enumerate}
    \item Abrir el modal de definiciones.
    \item Cerrar el modal.
    \item Abrir el modal de definiciones.
    \item Intentar darle al botón de ``Ok'' y comprobar que no está habilitado.
    \item No introducir datos al insertar definición y dar click a añadir.
    \item Comprobar que no se ha producido ningún cambio.
    \item Introducir datos, ejemplo ``hola'' y dar click a añadir.
    \item Comprobar que se ha introducido en la lista de frases.
    \item Editar palabra introducida escribiendo una cadena vacía.
    \item Comprobar que no se ha cambiado el valor.
    \item Editar la palabra introducida escribiendo ``adios''.
    \item Comprobar que se ha cambiado el valor al esperado.
    \item Darle a borrar la definición.
    \item Comprobar que se ha borrado.
    \item Introducir datos, ejemplo ``pepe'' y dar click a añadir.
    \item Comprobar que se ha introducido en la lista de definiciones.
    \item Introducir un número de filas menor o igual que cero y comprobar que no se dibujan en la vista previa.
    \item Introducir un número de filas mayor que 100 y comprobar que no se dibujan en la vista previa.
    \item Introducir un número de filas mayor que cero pero menor o igual que 100 y comprobar que se dibuja el número indicado de filas.
    \item Probar todos los tipos de pautas y comprobar que se dibuja el tipo de pauta indicado en cada caso.
    \item Darle a ``Ok''.
    \item Comprobar que se ha cerrado el modal y se ha añadido el ejercicio al editable.
    \item Abrir el modal de definiciones.
    \item Comprobar que los campos están vacíos.
\end{enumerate}

\subsection{Errores encontrados}
\label{errores:definicion}
\begin{itemize}
    \item Se puede insertar el ejercicio en el editor sin ninguna definición.
    \item Se puede insertar un número negativo de filas.
    \item Se pueden añadir definiciones que sean solamente espacios.
    \item Al darle a la ``x'' no se resetea.
\end{itemize}


\section{Funcionalidad de generar resumen}
\label{planPruebas:resumen}
A continuación se expone la tabla (Figura \ref{tab:resumen}) con los casos de prueba de la funcionalidad de generar resumen y el procedimiento para hacer dichos casos.

\begin{table}[H]
    \resizebox{1.1\textwidth}{!}{%
    \begin{tabular}{|c|c|c|c|c|}
    \hline
    \textbf{Precondición} & \textbf{Campo} & \textbf{Condición} & \textbf{Datos de entrada} & \textbf{\begin{tabular}[c]{@{}c@{}}Salida esperada\\ (Postcondición)\end{tabular}} \\ \hline
    TextoResumido=”” & Texto Original & Longitud del TextoOriginal\textless{}= 0 & TextoOriginal= “” & TextoResumido=”” \\ \hline
    TextoResumido=”” & Texto Original & Longitud del TextoOriginal\textgreater 0 & TextoOriginal= “...” & \begin{tabular}[c]{@{}c@{}}TextoResumido=”...”\\ siendo Longitud del TextoResumido \\ \textless Longitud del TextoOriginal\end{tabular} \\ \hline
    TextoResumido=”” & Texto Original & Longitud del TextoOriginal\textgreater 0 & TextoOriginal= “dgsgds” & TextoResumido=”” \\ \hline
    TextoResumido=”” & Tamaño & Tamaño \textless{}= 0 & TextoOriginal= “...” & TextoResumido=”” \\ \hline
    TextoResumido=”” & Tamaño & Tamaño \textgreater 0 & TextoOriginal= “...” y Tamaño=10 & \begin{tabular}[c]{@{}c@{}}TextoResumido=”...” \\ con longitud de TextoResumido  = 10\end{tabular} \\ \hline
    \end{tabular}%
    }
    \caption{Casos de prueba de la funcionalidad de generar resumen.}
    \label{tab:resumen}
\end{table}



\subsection{Procedimiento de prueba}
\label{procedimientoPruebas:resumen}
\begin{enumerate}
    \item Abrir el modal de Generar Resumen.
    \item Cerrar el modal.
    \item Abrir el modal de Generar Resumen.
    \item No introducir texto original y dar click a resumir.
    \item Comprobar que no ha producido ningún cambio.
    \item Introducir un texto, elegir una cantidad de palabras menor o igual que cero y darle a resumir.
    \item Cerrar el modal.
    \item Abrir el modal de Generar Resumen.
    \item Comprobar que se ha reseteado el modal.
    \item Introducir un texto, elegir una cantidad de palabras menor o igual que cero y darle a resumir.
    \item Comprobar que no ha producido ningún cambio.
    \item Introducir un texto, elegir una cantidad de palabras mayor que cero y darle a resumir.
    \item Comprobar que se ha devuelto lo especificado.
    \item Comprobar que los campos están vacíos.
    \item Introducir un texto no coherente, elegir una cantidad de palabras mayor que cero y darle a resumir.
    \item Comprobar que no ha producido ningún cambio.

\end{enumerate}

\subsection{Errores encontrados}
\label{errores:resumen}
\begin{itemize}
    \item Plantear no permitir elegir 0 palabras es una opción que nunca se podría dar.
    \item Cuando se introduce caractares sin sentido al resumirlo se produce un resultados como los siguiente: ``No se ha podido resumir'', no tiene sentido. Esto no debería poder introducirse en el editor
\end{itemize}

\section{Funcionalidad de leyenda de colores}
\label{planPruebas:leyenda}
A continuación se expone la tabla (Figura \ref{tab:leyenda}) con los casos de prueba de la funcionalidad de leyenda de colores y el procedimiento para hacer dichos casos.

\begin{table}[H]
    \resizebox{1.1\textwidth}{!}{
    \begin{tabular}{|c|c|c|c|c|}
    \hline
    \textbf{Precondición} & \textbf{Campo} & \textbf{Condición} & \textbf{Datos de entrada} & \textbf{\begin{tabular}[c]{@{}c@{}}Salida esperada\\ (Postcondición)\end{tabular}} \\ \hline
    Lista={[} {]} & Nuevo concepto & Longitud de frase \textless{}= 0 & Frase = “” & Lista = {[} {]} \\ \hline
    Lista={[} {]} & Nuevo concepto & Longitud de frase \textgreater 0 & Frase = “hola” & Lista = {[}“hola”(0,0,0) {]} \\ \hline
    Lista={[}“hola”(0,0,0){]} & Editar concepto & Longitud de frase \textless{}= 0 y la Frase existe en la lista & Frase = “” & Lista = {[}“hola”(0,0,0){]} \\ \hline
    Lista={[}“hola”(0,0,0){]} & Editar concepto & \begin{tabular}[c]{@{}c@{}}Longitud de frase \textgreater 0\\  y la Frase existe en la lista\end{tabular} & Frase = “adios” & Lista = {[}“adios”(0,0,0,){]} \\ \hline
    Lista={[}“hola”(0,0,0){]} & Editar color & \begin{tabular}[c]{@{}c@{}}Color válido\\  y la Frase existe en la lista\end{tabular} & Color = (255,0,0) & Lista = {[}“hola”(255,0,0,){]} \\ \hline
    Lista={[}“hola”(0,0,0){]} & Borrar concepto & Existe la frase a borrar en la lista & Frase = “hola” & Lista = {[} {]} \\ \hline
    \end{tabular}
    }
    \caption{Casos de prueba de la funcionalidad de leyenda de colores.}
    \label{tab:leyenda}
\end{table}

\subsection{Procedimiento de prueba}
\label{procedimientoPruebas:leyenda}
\begin{enumerate}
    \item Abrir el modal.
    \item Cerrar el modal.
    \item Abrir el modal otra vez.
    \item Intentar darle al botón de ``Ok'' y comprobar que no está habilitado.
    \item No introducir datos al insertar un concepto y dar click a añadir.
    \item Comprobar que no se ha producido ningún cambio.
    \item Introducir datos, ejemplo ``hola'' y dar click a añadir.
    \item Comprobar que se ha introducido en la lista de conceptos.
    \item Editar palabra introducida escribiendo una cadena vacía.
    \item Comprobar que no se ha cambiado el valor.
    \item Editar la palabra introducida escribiendo ``adios''.
    \item Comprobar que se ha cambiado el valor al esperado.
    \item Editar el color seleccionado un color diferente.
    \item Comprobar que se ha cambiado el color.
    \item Darle a borrar el concepto.
    \item Comprobar que se ha borrado.
    \item Introducir datos, ejemplo ``pepe'' y dar click a añadir.
    \item Comprobar que se ha introducido en la lista de definiciones.
    \item Darle a ``Ok''.
    \item Comprobar que se ha cerrado el modal y se ha añadido la leyenda al editable.
    \item Abrir el modal.
    \item Comprobar que los campos están vacíos.
\end{enumerate}

\subsection{Errores encontrados}
\label{errores:leyenda}
\begin{itemize}
    \item  Se puede editar y añadir solamente espacios.
    \item  Al darle a la ``x'' no se resetea.
    \item Si se añade una palabra demasiado larga, no queda espacio para el cuadro del color.
\end{itemize}

\section{Funcionalidad de espacios para dibujar}
\label{planPruebas:dibujar}
A continuación se expone la tabla (Figura \ref{tab:dibujar}) con los casos de prueba de la funcionalidad de espacios para dibujar y el procedimiento para hacer dichos casos.

\begin{table}[H]
    \resizebox{1.1\textwidth}{!}{
    \begin{tabular}{|c|c|c|c|c|}
    \hline
    \textbf{Precondición} & \textbf{Campo} & \textbf{Condición} & \textbf{Datos de entrada} & \textbf{\begin{tabular}[c]{@{}c@{}}Salida esperada\\ (Postcondición)\end{tabular}} \\ \hline
    Enunciado=”” & Insertar enunciado & Longitud de enunciado\textless{}= 0 & Enunciado= “” & Enunciado=”” \\ \hline
    Enunciado=”” & Insertar enunciado & Longitud de enunciado\textgreater 0 & Enunciado= “hola” & Enunciado=“hola” \\ \hline
    Enunciado=“hola” & Insertar enunciado & Longitud de enunciado\textless{}= 0 & Enunciado= “” & Enunciado=“” \\ \hline
    Enunciado=“hola” & Insertar enunciado & Longitud de enunciado\textgreater 0 & Enunciado= “adios” & Enunciado= “adios” \\ \hline
    espacio=1 & Insertar espacio & espacio\textless{}= 0 y existe un enunciado & espacio= -1 & Enunciado=”” \\ \hline
    espacio=1 & Insertar espacio & espacio \textgreater 0 y existe un enunciado & espacio= 2 & Enunciado != “” \\ \hline
    \end{tabular}
    }
    \caption{Casos de prueba de la funcionalidad de espacios para dibujar.}
    \label{tab:dibujar}
\end{table}

\subsection{Procedimiento de prueba}
\label{procedimientoPruebas:dibujar}
\begin{enumerate}
    \item Abrir el modal de espacios para dibujar.
    \item Cerrar el modal.
    \item Abrir el modal de espacios para dibujar.
    \item Comprobar que no está habilitado el botón de ``Ok''.
    \item No introducir datos al insertar enunciado y comprobar que no hay nada en la vista previa.
    \item Introducir espacio mayor que cero y comprobar que no hay nada en la vista previa.
    \item Introducir espacio menor que cero y comprobar que no hay nada en la vista previa.
    \item Introducir texto en el insertar enunciado, introducir espacio  mayor que cero y comprobar que se ve en la vista previa.
    \item Probar a cambiar enunciado y comprobar que el cambio se refleja en la vista previa.
    \item Desmarcar recuadro y ver en vista previa que no hay recuadro.
    \item Marcar recuadro y ver que aparece en vista previa.
    \item Darle a ``Ok''.
    \item Comprobar que se ha cerrado el modal.
    \item Abrir el modal de  espacios para dibujar.
    \item Comprobar que los campos están vacíos.
\end{enumerate}

\subsection{Errores encontrados}
\label{errores:espacio}
\begin{itemize}
    \item  Como nombre del modal tienes puesto Ejercicios de desarollo, sería Ejercicios para dibujar.
    \item No funciona el recuadrar, si lo marcas y luego lo desmarcas en la vista previa no se refleja.
\end{itemize}

\section{Funcionalidad de pictotraductor}
\label{planPruebas:pictotraductor}
A continuación se expone la tabla (Figura \ref{tab:pictotraductor}) con los casos de prueba de la funcionalidad de pictotraductor y el procedimiento para hacer dichos casos.

\begin{table}[H]
    \resizebox{1.1\textwidth}{!}{
    \begin{tabular}{|l|l|l|l|l|}
    \hline
    \textbf{Precondición} & \textbf{Campo} & \textbf{Condición} & \textbf{Datos de entrada} & \textbf{\begin{tabular}[c]{@{}l@{}}Salida esperada\\ (Postcondición)\end{tabular}} \\ \hline
    TextoOriginal=”” & Insertar texto & Longitud de texto\textless{}= 0 & texto= “” & TextoOriginal=”” \\ \hline
    TextoOriginal=”” & Insertar texto & Longitud de texto\textgreater 0 & texto= “La libélula tiene cuatro alas” & \begin{tabular}[c]{@{}l@{}}TextoOriginal=\\ “La libélula tiene cuatro alas””\end{tabular} \\ \hline
    \begin{tabular}[c]{@{}l@{}}TextoOriginal=\\ “La libélula tiene cuatro alas”\end{tabular} & Insertar texto & Longitud de texto\textless{}= 0 & texto= “” & TextoOriginal=“” \\ \hline
    \begin{tabular}[c]{@{}l@{}}TextoOriginal=\\ “La libélula tiene cuatro alas”\end{tabular} & Insertar texto & Longitud de texto\textgreater 0 & \begin{tabular}[c]{@{}l@{}}texto= \\ “descomponer los alimentos para obtener sus nutrientes”\end{tabular} & \begin{tabular}[c]{@{}l@{}}TextoOriginal= \\ “descomponer los alimentos para obtener sus nutrientes”\end{tabular} \\ \hline
    \begin{tabular}[c]{@{}l@{}}TextoOriginal=\\ “La libélula tiene cuatro alas”\end{tabular} & opciones de texto & Longitud de texto\textgreater 0 & opciones de texto=arriba & \begin{tabular}[c]{@{}l@{}}Pictogramas con texto \\ encima del pictograma\end{tabular} \\ \hline
    \begin{tabular}[c]{@{}l@{}}TextoOriginal=\\ “La libélula tiene cuatro alas”\end{tabular} & opciones de texto & Longitud de texto\textgreater 0 & opciones de texto=debajo & \begin{tabular}[c]{@{}l@{}}Pictogramas con texto \\ debajo del pictograma\end{tabular} \\ \hline
    \begin{tabular}[c]{@{}l@{}}TextoOriginal=\\ “La libélula tiene cuatro alas”\end{tabular} & opciones de texto & Longitud de texto\textgreater 0 & opciones de texto=sin texto & Pictogramas sin texto \\ \hline
    \begin{tabular}[c]{@{}l@{}}TextoOriginal=\\ “La libélula tiene cuatro alas\end{tabular} & opciones de color & Longitud de texto\textgreater 0 & opciones de color=blanco y negro & Pictograma en blanco y negro \\ \hline
    \begin{tabular}[c]{@{}l@{}}TextoOriginal=\\ “La libélula tiene cuatro alas”\end{tabular} & vista picto & Longitud de texto\textgreater 0 & visualización de un  picto oculta & Pictograma oculto \\ \hline
    \begin{tabular}[c]{@{}l@{}}TextoOriginal=\\ “La libélula tiene cuatro alas”\end{tabular} & vista picto & Longitud de texto\textgreater 0 & visualización de todos los  picto oculta & boton Ok deshabilitado \\ \hline
    \end{tabular}
    }
    \caption{Casos de prueba de la funcionalidad pictotraductor.}
    \label{tab:pictotraductor}
\end{table}

\subsection{Procedimiento de prueba}
\label{procedimientoPruebas:pictotraductor}
\begin{enumerate}
    \item Abrir el modal de pictotraductor.
    \item Cerrar el modal.
    \item Abrir el modal de pictotraductor.
    \item Comprobar que no está habilitado el botón de ``Ok''.
    \item No introducir datos al insertar enunciado y comprobar que no hay nada en la vista previa.
    \item Introducir cadena vacía y comprobar que no hay nada en la vista previa
    \item Introducir ``La libélula tiene cuatro alas'' y comprobar que aparecen los respectivos pictogramas.
    \item Introducir ``La libélula tiene cuatro alas'' y comprobar que aparecen los respectivos pictogramas y cambiar el texto por ``descomponer los alimentos''. para obtener sus nutrientes y ver que los pictogramas se han creado.
    \item Introducir ``La libélula tiene cuatro alas'', poner como opción de texto arriba y comprobar que aparecen los pictogramas con el texto arriba.
    \item Introducir ``La libélula tiene cuatro alas'', poner como opción de texto debajo y comprobar que aparecen los pictogramas con el texto debajo.
    \item Introducir ``La libélula tiene cuatro alas'', poner como opción de texto sin texto y comprobar que aparecen los pictogramas sin texto.
    \item Introducir ``La libélula tiene cuatro alas'', poner como opción de color blanco y negro  y comprobar que aparecen los pictogramas en blanco y negro.
    \item Introducir ``La libélula tiene cuatro alas'', comprobar que aparecen los pictogramas y ocultar uno.
    \item Introducir ``La libélula tiene cuatro alas'', comprobar que aparecen los pictogramas y ocultar todos, ver que el botón de ``Ok'' no está habilitado.
    \item Darle a ``Ok''.
    \item Comprobar que se ha cerrado el modal.
    \item Abrir el modal de  pictotraductor.
    \item Comprobar que los campos están vacíos.
\end{enumerate}

\subsection{Errores encontrados}
\label{errores:pictotra}
\begin{itemize}
    \item Cuando escribes una frase y una de las palabras no tiene pictograma en el editable se muestra un cuadrado donde pone la palabra y el icono que sale cuando no ha cargado la imagen..
    \item Cuando ocultas todos los pictogramas no deberia estar habilitado el botón de ``OK''.
    \item  Cuando quieres editar algo que ya esta en el editable la configuración está mal.
    \item Si pones algo que no existe (djfhwjhfb) se habilita el botón de ``Ok'' y al darle pasa lo mismo que en el primer punto.
    \item  Si pongo espacios se habilita el botón de pictotraductor.
\end{itemize}


\section{Funcionalidad de relacionar conceptos}
\label{planPruebas:conceptos}
A continuación se expone la tabla (Figura \ref{tab:conceptos}) con los casos de prueba de la funcionalidad de relacionar conceptos y el procedimiento para hacer dichos casos.

\begin{table}[H]
    \resizebox{1.1\textwidth}{!}{
    \begin{tabular}{|c|c|c|c|c|}
    \hline
    \textbf{Precondición} & \textbf{Campo} & \textbf{Condición} & \textbf{Datos de entrada} & \textbf{\begin{tabular}[c]{@{}c@{}}Salida esperada\\ (Postcondición)\end{tabular}} \\ \hline
    NumFilas=0 y NumCols=0 & Insertar filas y columnas & NumFilas\textless{}= 1 y NumCols\textless{}= 1 & NumFilas=0 y NumCols=0 & La tabla de conceptos no sale \\ \hline
    NumFilas=0 y NumCols=0 & Insertar filas y columnas & \begin{tabular}[c]{@{}c@{}}1 \textless NumFilas \textless 20 \\ y 1 \textless NumCols\textless 20\end{tabular} & NumFilas=2 y NumCols=2 & Dos columnas con dos filas \\ \hline
    NumFilas=2 y NumCols=2 & Insertar filas y columnas & \begin{tabular}[c]{@{}c@{}}1 \textless NumFilas \textless 20 \\ y 1 \textless NumCols\textless 20\end{tabular} & \begin{tabular}[c]{@{}c@{}}En la tabla añadir: \\ “animales”, “vaca”, “gato”\end{tabular} & Tabla con esos datos \\ \hline
    NumFilas=2 y NumCols=2 & Insertar filas y columnas & \begin{tabular}[c]{@{}c@{}}1 \textless NumFilas \textless 20 \\ y 1 \textless NumCols\textless 20\end{tabular} & En la tabla añadir cadenas vacías & Botón de Ok deshabilitado \\ \hline
    \begin{tabular}[c]{@{}c@{}}En una columna: “animales”, \\ en otra: “gato”, “vaca” y NumFilas=2 y NumCols=2\end{tabular} & Reordenar &  & \begin{tabular}[c]{@{}c@{}}En la tabla: \\ “animales”, “vaca”, “gato”\end{tabular} & \begin{tabular}[c]{@{}c@{}}En una columna: \\ “animales”, en otra: “gato”, “vaca”\end{tabular} \\ \hline
    Tabla vacía & Reordenar &  & Tabla vacía & Tabla vacía \\ \hline
    \end{tabular}
    }
    \caption{Casos de prueba de la funcionalidad de relacionar conceptos.}
    \label{tab:conceptos}
\end{table}



\subsection{Procedimiento de prueba}
\label{procedimientoPruebas:conceptos}
\begin{enumerate}
    \item Abrir el modal de relacionar conceptos.
    \item Cerrar el modal.
    \item Abrir el modal de relacionar conceptos.
    \item Comprobar que no está habilitado el botón de ``Ok''.
    \item Introducir número de columnas y filas  menores a 1.
    \item Comprobar que las columnas y filas no aparecen.
    \item Introducir número de columnas y filas mayores que 1.
    \item Comprobar que ese número de  las columnas y filas aparecen como tabla dividida en columnas.
    \item Introducir texto en las celdas.
    \item Ver que aparece en la vista previa.
    \item Reordenar y ver que se desordenan.
    \item Eliminar los concepto puestos en las celdas
    \item Poner en las celdas espacios en blanco
    \item Ver que el botón de ``Ok'' está deshabilitado
    \item Volver a poner conceptos en las celdas
    \item Darle a ``Ok''.
    \item Comprobar que se ha cerrado el modal.
    \item Abrir el modal de relacionar conceptos.
    \item Comprobar que los campos están vacíos.

\end{enumerate}


\subsection{Errores encontrados}
\label{errores:pictotra}
\begin{itemize}
    \item No deja borrar la palabra entera, siempre se queda la ultima letra.
    \item Cuando creas una tabla de las dimensiones que sea y escribes dos conceptos en la misma columna se activa reordenar y el botón de ``Ok''.
    \item  Si en una celda pongo espacios me lo da como bueno.

\end{itemize}

\section{Funcionalidad de ejercicios de matemáticas}\label{planPruebas:mates}
A continuación se expone la tabla (Figura \ref{tab:mate}) con los casos de prueba de la funcionalidad de ejercicios de matemáticas y el procedimiento para hacer dichos casos.

\begin{table}[H]
    \resizebox{1.1\textwidth}{!}{
    \begin{tabular}{|c|c|c|c|c|}
    \hline
    \textbf{Precondición} & \textbf{Campo} & \textbf{Condición} & \textbf{Datos de entrada} & \textbf{\begin{tabular}[c]{@{}c@{}}Salida esperada\\ (Postcondición)\end{tabular}} \\ \hline
     & Insertar hueco (tecla espacio) &  &  & \begin{tabular}[c]{@{}c@{}}Se ha insertado un nuevo hueco, \\ justo después del hueco seleccionado.\end{tabular} \\ \hline
    1 hueco y 1 fórmula & \begin{tabular}[c]{@{}c@{}}Borrar hueco\\ (tecla retroceso)\end{tabular} &  &  & \begin{tabular}[c]{@{}c@{}}Se ha mantenido el hueco, ya que como mínimo \\ debe haber un hueco y una fórmula.\end{tabular} \\ \hline
    Al menos 2 huecos & \begin{tabular}[c]{@{}c@{}}Borrar hueco\\ (tecla retroceso)\end{tabular} &  &  & Se ha borrado el hueco seleccionado. \\ \hline
     & Insertar fórmula (tecla enter) &  &  & \begin{tabular}[c]{@{}c@{}}Se ha insertado una nueva formula, \\ justo debajo de la fórmula seleccionada.\end{tabular} \\ \hline
    \begin{tabular}[c]{@{}c@{}}Al menos dos fórmulas y \\ solo un hueco en cada una\end{tabular} & \begin{tabular}[c]{@{}c@{}}Borrar hueco\\ (tecla retroceso)\end{tabular} &  &  & Se ha borrado la fórmula del hueco seleccionado. \\ \hline
     & Huecos &  & “1+ =2” & Fórmula = “1 + \_\_\_ = 2” \\ \hline
    \end{tabular}
    }
    \caption{Casos de prueba de la funcionalidad de ejercicios de matemáticas.}
    \label{tab:mate}
\end{table}

\subsection{Procedimiento de prueba}
\label{procedimientoPruebas:mate}
\begin{enumerate}
    \item Abrir el modal de Fórmula matemática.
    \item Comprobar que se ha abierto el modal.
    \item Cerrar el modal.
    \item Comprobar que se ha cerrado el modal.
    \item Abrir el modal de Fórmula matemática.
    \item Comprobar que no está habilitado el botón de ``Ok''.
    \item Introducir datos en el primer hueco, ejemplo ``123''.
    \item Comprobar que se ha habilitado el botón de ``Ok''.
    \item Pulsar el botón de ``Ok''.
    \item Comprobar que se ha cerrado el modal y se ha introducido el texto correctamente en el documento de trabajo.
    \item Abrir el modal de Fórmula matemática.
    \item Pulsar espacio varias veces (Probar manteniendo pulsado el espacio también).
    \item Comprobar que se han creado nuevos huecos y que el cursor está en la última posición.
    \item Pulsar la tecla de retroceso varias veces (Probar manteniendo pulsada la tecla de retroceso también).
    \item Comprobar que se han borrado huecos y que el cursor está en la misma posición.
    \item Escribir una fórmula con huecos, por ejemplo`` 2 + \rule{10mm}{0.1mm} = 5''.
    \item Pulsar el botón de ``Ok''.
    \item Comprobar que se ha cerrado el modal y se ha introducido correctamente el texto en el documento de trabajo.
    \item Introducir datos, ejemplo ``123''.
    \item Añadir varios huecos.
    \item Cerrar el modal.
    \item Abrir el modal de Fórmula matemática.
    \item Comprobar que los campos están vacíos.
\end{enumerate}

\subsection{Errores encontrados}
\label{errores:mates}
\begin{itemize}
    \item Cuando no se ha introducido ningún valor, el botón de Ok no se muestra como si estuviera desactivado, aunque sí que está desactivado ya que no se puede introducir una formula vacía.
    \item Si mantienes el espacio pulsado, se crean huecos más rápido que el movimiento del cursor, entonces el cursor no acaba en la última posición, aunque hubiese empezado en la última posición.
\end{itemize}

\section{Funcionalidad de exportar a PDF}
\label{planPruebas:exportar}
A continuación se expone la tabla (Figura \ref{tab:exportar}) con los casos de prueba de la funcionalidad de exportar a PDF y el procedimiento para hacer dichos casos.

\begin{table}[H]
    \resizebox{1.1\textwidth}{!}{
    \begin{tabular}{|c|c|c|c|c|}
    \hline
    \textbf{Precondición} & \textbf{Campo} & \textbf{Condición} & \textbf{Datos de entrada} & \textbf{\begin{tabular}[c]{@{}c@{}}Salida esperada\\ (Postcondición)\end{tabular}} \\ \hline
    Editor vacío & Insertar ejercicio & Editor vacío & Ejercicio de V/F & \begin{tabular}[c]{@{}c@{}}Ejercicio numerado, que salgan las frases \\ con la opción de V/F y el enunciado en negrita\end{tabular} \\ \hline
    \begin{tabular}[c]{@{}c@{}}Editor con\\ ejercicio de V/F\end{tabular} & Insertar ejercicio & Editor con ejercicio de V/F & Ejercicio de definiciones con pictogramas & \begin{tabular}[c]{@{}c@{}}Ejercicio numerado y el enunciado en negrita. \\ Cada concepto debe tener su respectivo tipo de pauta \\ y junto a cada concepto su programa\end{tabular} \\ \hline
    \begin{tabular}[c]{@{}c@{}}Ejercicios \\ anteriormente creados\end{tabular} & Insertar ejercicio & Ejercicios anteriormente creados & Ejercicio de desarrollo & \begin{tabular}[c]{@{}c@{}}Ejercicio numerado y el enunciado en negrita. \\ Debe aparecer la pauta seleccionada\end{tabular} \\ \hline
    \begin{tabular}[c]{@{}c@{}}Ejercicios \\ anteriormente creados\end{tabular} & Insertar ejercicio & Ejercicios anteriormente creados & Ejercicio de huecos & \begin{tabular}[c]{@{}c@{}}Ejercicio numerado y el enunciado en negrita. \\ Debe aparecer en el texto las líneas representando los huecos\end{tabular} \\ \hline
    \begin{tabular}[c]{@{}c@{}}Ejercicios \\ anteriormente creados\end{tabular} & Insertar ejercicio & Ejercicios anteriormente creados & Ejercicio de relacionar conceptos & \begin{tabular}[c]{@{}c@{}}Ejercicio numerado y el enunciado en negrita. \\ Debe aparecer los conceptos junto a su viñeta\end{tabular} \\ \hline
    \begin{tabular}[c]{@{}c@{}}Ejercicios\\ anteriormente creados\end{tabular} & Insertar ejercicio & Ejercicios anteriormente creados & Sopa de letras & \begin{tabular}[c]{@{}c@{}}Ejercicio numerado , el enunciado en negrita y la \\ sopa de letras recuadrada con una letra en cada recuadro\end{tabular} \\ \hline
    \begin{tabular}[c]{@{}c@{}}Ejercicios \\ anteriormente creados\end{tabular} & Insertar ejercicio & Ejercicios anteriormente creados & Fórmulas matemáticas & \begin{tabular}[c]{@{}c@{}}Ejercicio numerado , el enunciado en negrita, \\ de forma numerada cada fórmula con los huecos\end{tabular} \\ \hline
    \begin{tabular}[c]{@{}c@{}}Ejercicios \\ anteriormente creados\end{tabular} & Insertar ejercicio & Ejercicios anteriormente creados & Ejercicio espacio para dibujar & \begin{tabular}[c]{@{}c@{}}Ejercicio numerado y el enunciado en negrita. \\ Un recuadro para dibujar o el espacio sin recuadro\end{tabular} \\ \hline
    \begin{tabular}[c]{@{}c@{}}Ejercicios \\ anteriormente creados\end{tabular} & Insertar ejercicio & Ejercicios anteriormente creados & Resumen & Texto resumido \\ \hline
    \begin{tabular}[c]{@{}c@{}}Ejercicios\\ anteriormente creados\end{tabular} & Insertar ejercicio & Ejercicios anteriormente creados & Leyenda de colores & \begin{tabular}[c]{@{}c@{}}Ejercicio numerado y título de la leyenda en negrita. \\ Al lado de cada concepto un cuadrado con el color seleccionado\end{tabular} \\ \hline
    \begin{tabular}[c]{@{}c@{}}Ejercicios \\ anteriormente creados\end{tabular} & Insertar ejercicio & Ejercicios anteriormente creados & Pictotraductor & \begin{tabular}[c]{@{}c@{}}Pictogramas con el texto o sin él, con color o en blanco \\ y negro, con un borde y en el caso de no haber aparece el texto sin borde\end{tabular} \\ \hline
    \end{tabular}
    }
    \caption{Casos de prueba de la funcionalidad de exportar a PDF.}
    \label{tab:exportar}
\end{table}

\subsection{Procedimiento de prueba}
\label{procedimientoPruebas:exportar}
\begin{enumerate}
    \item Introducir ejercicio de V/F.
    \item Introducir ejercicio de definiciones.
    \item Introducir un pictograma al lado de cada concepto a definir.
    \item Introducir ejercicio de desarrollo.
    \item Introducir ejercicio de completar huecos.
    \item Introducir ejercicio de relacionar conceptos.
    \item Introducir ejercicio de sopa de letras.
    \item Introducir ejercicio de fórmulas matemáticas.
    \item Introducir ejercicio de espacios para dibujar.
    \item Introducir resumen.
    \item Introducir leyenda de colores.
    \item Introducir una frase transformada a pictogramas.
    \item Darle al botón de exportar a PDF.
    \item Comprobar que los enunciados están en negrita y los ejercicios están numerados.
    \item Comprobar en el ejercicio de verdadero y falso que las frases generadas están numeradas y tienen al final (V/F).
    \item Comprobar que en ejercicio de definiciones y desarrollo aparece el tipo de pauta.
    \item Comprobar que en el ejercicio de completar huecos y en el de fórmulas matemáticas aparecen las líneas haciendo referencia a los huecos.
    \item Comprobar que en el ejercicio de relacionar conceptos cada concepto tiene su propia viñeta.
    \item Comprobar que en la sopa de letras está recuadrada y en cada recuadro hay una letra.
    \item Comprobar que en el ejercicio de espacio para dibujar tiene el recuadro, si se ha marcado, o un espacio en blanco si no se ha marcado la opción de recuadro.
    \item Comprobar que en la leyenda de colores se muestran los colores al lado de los conceptos.
    \item Comprobar en la traducción de un texto a pictogramas si salen las palabras en la posición indicada, en el caso de que se haya seleccionado que se muestre las palabras. También comprobar que tienen color o no según se ha elegido. Y aquellas palabras que no tengan traducción pictograma aparecen como texto plano.

\end{enumerate}

\subsection{Errores encontrados}
\label{errores:mates}
\begin{itemize}
    \item Al insertar pictogramas que no entran en una hoja se rompen y se ven mal.
    \item No salen los pictogramas.
    \item Ejercicios distribuidos uno por hoja dejando un espacio demasiado amplio.
\end{itemize}
