\chapter{Plan de pruebas: Casos de prueba y guiones de prueba}\label{ape:pruebas}

En este Apéndice se exponen los distintos casos de prueba,  que son escenarios específicos para verificar si la funcionalidad cumple con los requisitos y funcionalidades establecidas, y guiones que se han definido para probar cada funcionalidad de AdaptaMaterialEscolar 2.0.

\section{Funcionalidad de verdadero/falso}
\label{planPruebas:v/f}
A continuación se expone la tabla (Figura \ref{tab:v/f}) con los casos de prueba de la funcionalidad de verdadero/falso y el procedimiento para hacer dichos casos.

\begin{table}[H]
    \centering
    \scalebox{0.85}{
    \begin{tblr}{ colspec = {|c|c|m{4cm}|c|c|}}
    \hline
    \textbf{Precondición} &  \textbf{Campo} &  \hfil\textbf{Condición} &  \textbf{Datos de entrada} &   \textbf{\begin{tabular}[c]{@{}l@{}}Salida esperada\\ (Postcondición)\end{tabular}} \\ \hline
    Lista=[ ] & Nueva frase & Longitud de frase $\leq$ 0 & Frase = ``'' & Lista = [ ]\\ \hline
    Lista=[ ] & Nueva frase & Longitud de frase > 0 & Frase = ``hola'' & Lista = [``hola'']\\ \hline
    Lista=[``hola''] & Editar frase &Longitud de frase $\leq$ 0 y la Frase existe en la lista & Frase = ``'' & Lista = [``hola'']\\ \hline
    Lista=[``hola''] & Editar frase & Longitud de frase > 0 y la Frase existe en la lista & Frase = ``adios'' & Lista = [``adios'']\\ \hline
    Lista=[``hola''] & Borrar frase & Existe la frase a borrar en la lista & Frase = ``hola'' & Lista = [ ]\\ \hline
    Lista=[``hola'', ``adios''] & Reordenar frases &  & Lista=[``hola''”, ``adios''] & Lista = [``adios'', ``hola'']\\ \hline
    Lista=[ ] & Reordenar frases &  & Lista=[ ] & Lista=[ ] \\ \hline
    Lista=[``hola''] & Reordenar frases &  & Lista=[``hola''] & Lista=[``hola'']\\ \hline
    \end{tblr}
    }
    \caption{Casos de prueba de la funcionalidad de verdadero/falso.}
    \label{tab:v/f}
\end{table}

\subsection{Procedimiento de prueba}
\label{procedimientoPruebas:v/f}
\begin{enumerate}
    \item Abrir el modal de Verdadero/Falso.
    \item Cerrar el modal.
    \item Abrir el modal de Verdadero/Falso.
    \item Comprobar que no está habilitado el botón de ``Reordenar'' ni el botón de ``O.k''.
    \item No introducir datos al insertar frase y dar click a añadir.
    \item Comprobar que no ha producido ningún cambio.
    \item Introducir datos, ejemplo ``hola''  y dar click a añadir.
    \item Comprobar que se ha introducido en la lista de frases.
    \item Editar la palabra introducida introduciendo una cadena vacía.
    \item Comprobar que no se ha cambiado el valor.
    \item Editar la palabra introducida introduciendo .“adios”.
    \item Comprobar que se ha cambiado el valor al esperado
    \item Darle a borrar la frase.
    \item Comprobar que se ha borrado.
    \item Introducir datos, ejemplo ``pepe'' dar click a añadir.
    \item Comprobar que se ha introducido en la lista de frases.
    \item Comprobar que no está habilitado el botón de reordenar.
    \item Introducir datos, ejemplo ``gato'' y dar click a añadir.
    \item Comprobar que se ha introducido en la lista de frases.
    \item Darle a reordenar .
    \item Comprobar que no tiene el orden anterior.
    \item Darle a ``Ok''.
    \item Comprobar que se ha cerrado el modal.
    \item Abrir el modal de Verdadero/Falso.
    \item Comprobar que los campos están vacíos.
  \end{enumerate}

\section{Funcionalidad de buscar pictograma}
\label{planPruebas:busPicto}
A continuación se expone la tabla (Figura \ref{tab:busPicto}) con los casos de prueba de la funcionalidad de buscar pictograma y el procedimiento para hacer dichos casos.

\begin{table}[H]
    \centering
    \scalebox{0.85}{
    \begin{tblr}{ colspec = {|c|c|m{4cm}|c|m{4.5cm}|}}
    \hline
  \textbf{Precondición} &  \textbf{Campo} & \hfil\textbf{Condición} &  \textbf{Datos de entrada} &   \hfil\textbf{\begin{tabular}[c]{@{}l@{}}Salida esperada\\ (Postcondición)\end{tabular}} \\ \hline
    ListaPicto=[ ] & Nuevo concepto & Longitud del concepto $\leq$ 0 & concepto= ``'' & \hfil ListaPicto=[ ]\\ \hline
    ListaPicto=[ ] & Nuevo concepto & Longitud del concepto > 0 & concepto= ``hola'' &  ListaPicto=[pictograma1, pictograma2…]\\ \hline
    ListaPicto=[ ] & Nuevo concepto & Longitud del concepto > 0 & concepto= ``jhfueh'' & ListaPicto=[ ], mensaje=``no se ha encontrado pictograma''\\ \hline
    \end{tblr}
    }
    \caption{Casos de prueba de la funcionalidad de buscar pictograma.}
    \label{tab:busPicto}
\end{table}

\subsection{Procedimiento de prueba}
\label{procedimientoPruebas:busPicto}
\begin{enumerate}
\item Abrir el modal de Buscar pictograma.
\item Cerrar el modal.
\item Abrir el modal de Buscar pictograma.
\item No introducir concepto y dar click a buscar.
\item Comprobar que no ha producido ningún cambio.
\item Introducir un concepto y darle a buscar.
\item Comprobar que salen los pictogramas relacionados con el concepto.
\item Darle a un pictograma y comprobar que se ha cerrado el modal.
\item Abrir el modal de Buscar pictograma.
\item Comprobar que los campos están vacíos.

\end{enumerate}


\section{Funcionalidad de desarrollo}
\label{planPruebas:desarrollo}
A continuación se expone la tabla (Figura \ref{tab:desarrollo}) con los casos de prueba de la funcionalidad de desarrollo y el procedimiento para hacer dichos casos.

\begin{table}[H]
    \centering
    \scalebox{0.85}{
    \begin{tblr}{ colspec = {|c|c|m{4cm}|c|c|}}
    \hline
  \textbf{Precondición} &  \textbf{Campo} & \hfil\textbf{Condición} &  \textbf{Datos de entrada} &  \textbf{\begin{tabular}[c]{@{}l@{}}Salida esperada\\ (Postcondición)\end{tabular}} \\ \hline
  Enunciado= ``'' & Insertar enunciado & Longitud de enunciado $\leq$  0 & Enunciado=``'' & Enunciado=``''\\ \hline
  Enunciado= ``'' & Insertar enunciado & Longitud de enunciado > 0 & Enunciado=``hola'' & Enunciado=``hola''\\ \hline
  Enunciado= ``hola'' & Insertar enunciado & Longitud de enunciado $\leq$ 0 & Enunciado=``'' & Enunciado=``''\\ \hline
  Enunciado= ``hola'' & Insertar enunciado & Longitud de enunciado > 0 & Enunciado=``adios'' &Enunciado=``adios''\\ \hline
  NumeroFilas=1 & Insertar número de filas & Número de filas $\leq$ 0 y existe un enunciado & NumeroFilas=-1 & Enunciado = ``''\\ \hline
  NumeroFilas=1 & Insertar número de filas & Número de filas > 0 y existe un enunciado & NumeroFilas=2 & Enunciado $\neq$ ``''\\ \hline
  Opción pauta=``'' & Elegir pauta & Número de filas > 0 y existe un enunciado &  Opción pauta $\neq$ ``'' & Opción pauta $\neq$ ``''\\ \hline
    \end{tblr}
    }
    \caption{Casos de prueba de la funcionalidad de desarrollo.}
    \label{tab:desarrollo}
\end{table}

\subsection{Procedimiento de prueba}
\label{procedimientoPruebas:desarrollo}
\begin{enumerate}
\item Abrir el modal de desarrollo.
\item Cerrar el modal.
\item Abrir el modal de desarrollo.
\item Comprobar que no está habilitado el botón de ``Ok''.
\item No introducir datos al insertar enunciado y comprobar que no hay nada en la vista previa. 
\item Introducir número de filas mayor que cero y comprobar que no hay nada en la vista previa.
\item Probar tipos de pauta y comprobar que no hay nada en la vista previa.
\item Introducir número de filas menor que cero y comprobar que no hay nada en la vista previa.
\item Probar tipos de pauta y comprobar que no hay nada en la vista previa.
\item Introducir texto en el insertar enunciado, introducir número de filas mayor que cero y comprobar que se ve en la vista previa .
\item Probar tipos de pauta y comprobar que se ve en la visa previa.
\item Probar a cambiar enunciado y comprobar que el cambio se refleja en la vista previa.
\item Darle a ``Ok''.
\item Comprobar que se ha cerrado el modal.
\item Abrir el modal de desarrollo.
\item Comprobar que los campos están vacíos.
\end{enumerate}

\section{Funcionalidad de sopa de letras}
\label{planPruebas:sopa}
A continuación se expone la tabla (Figura \ref{tab:sopa}) con los casos de prueba de la funcionalidad de sopa de letras y el procedimiento para hacer dichos casos.

\begin{table}[H]
    \centering
    \scalebox{0.70}{
    \begin{tblr}{ colspec = {|m{4cm}|m{4cm}|m{4cm}|c|m{4cm}|}}
    \hline
    \textbf{Precondición} &  \hfil\textbf{Campo} &  \hfil\textbf{Condición} &  \textbf{Datos de entrada} &   \textbf{\begin{tabular}[c]{@{}l@{}}Salida esperada\\ (Postcondición)\end{tabular}} \\ \hline
    Listapalabra=[ ] & \hfil Insertar palabra & Longitud de palabra  $\leq$ 0 & Palabra = ``'' & Listapalabra = [ ]\\ \hline
    Listapalabra=[ ] & \hfil Insertar palabra & Longitud de palabra  > 0 & Palabra = ``hola'' & Listapalabra = [``hola'']\\ \hline
    Listapalabra=[``hola''] & \hfil Insertar palabra & Longitud de palabra  $\leq$ 0 & Palabra = ``'' & Listapalabra = [``hola'']\\ \hline
    Listapalabra=[“hola”] & \hfil Insertar palabra & Longitud de frase > 0 & Palabra = ``adios'' & Listapalabra = [``hola'',``adios'']\\ \hline
    Listapalabra=[“hola”] & \hfil Editar palabra & Longitud de palabra $\leq$ 0 y la palabra existe en la lista & Palabra = ``'' & Listapalabra = [``hola'']\\ \hline
    Listapalabra=[``hola''] & \hfil Editar palabra & Longitud de palabra > 0 y la palabra existe en la lista
    & Palabra = ``adios'' & Listapalabra = [``adios'']\\ \hline
    Listapalabra=[``hola''] & \hfil Borrar palabra & Existe la frase a borrar en la lista & Palabra = ``hola'' & Listapalabra=[ ] \\ \hline
    NumFilas=1 y NumCols=1 & Insertar número de filas y columnas & NumFilas y numCols $\leq$ 0  & NumFilas=0 NumCols=0 & NumFilas=1 y NumCols=1\\ \hline
    NumFilas=1 y NumCols=1 & Insertar número de filas y columnas & NumFilas y numCols> 0 < 1000 & NumFilas=3 NumCols=8 & NumFilas=3 NumCols=8\\ \hline
    NumFilas=1 y NumCols=1 & Insertar número de filas y columnas & NumFilas =1000 numCols=1000 & NumFilas= 1000
    NumCols=1000 & sopa de letras correctamente creada\\ \hline
    NumFilas=1 y NumCols=1 y Listapalabra=[“Hola”,”adios”] y posicionamiento=horizontal & Insertar número de filas y columnas & NumFilas>0 numCols>0 & Número de filas=2 numCols=4 & sopa de letras correctamente creada\\ \hline
    NumFilas=1 y NumCols=1 y Listapalabra=[“Hola”,”adios”] y posicionamiento=horizontal & Insertar número de filas y columnas & NumFilas>0 numCols>0 & Número de filas=2 numCols=4 & sopa de letras correctamente creada\\ \hline
    NumFilas=1 y NumCols=1 y Listapalabra=[“Hola”,”adios”] y posicionamiento=horizontal & Insertar número de filas y columnas & NumFilas>0 y 0< numCols <longitud de la palabra más larga & Número de filas=2 numCols=3 & Error\\ \hline
    NumFilas=1 y NumCols=1 y Listapalabra=[“Hola”,”adios”] y posicionamiento=vertical & Insertar número de filas y columnas & NumFilas=número de palabras a insertar larga y numCols >0 & Número de filas=2 numCols=4 & sopa de letras correctamente creada\\ \hline
    NumFilas=1 y NumCols=1 y Listapalabra=[“Hola”,”adios”] y posicionamiento=vertical & Insertar número de filas y columnas & NumFilas<número de palabras a insertar larga y numCols >0 & Número de filas=1 numCols=4 & Error \\ \hline
    NumFilas=1 y NumCols=1 y Listapalabra=[“Hola”,”adios”] y posicionamiento=diagonal & Insertar número de filas y columnas & NumFilas=longitud de la palabra más larga
    larga y numCols=longitud de la palabra más larga
    & Número de filas=4 numCols=4 & sopa de letras correctamente creada\\ \hline
    NNumFilas=1 y NumCols=1 y Listapalabra=[“Hola”,”adios”] y posicionamiento=diagonal & Insertar número de filas y columnas & NumFilas<longitud de la palabra más larga
    larga y numCols<longitud de la palabra más larga & Número de filas=3 numCols=3 & Error\\ \hline
    \end{tblr}
    }
    \caption{Casos de prueba de la funcionalidad de sopa de letras.}
    \label{tab:sopa}
\end{table}
\subsection{Procedimiento de prueba}
\label{procedimientoPruebas:sopa}
\begin{enumerate}
    \item Abrir el modal de sopa de letras.
    \item Cerrar el modal.
    \item Abrir el modal de sopa de letras.
    \item Comprobar que no está habilitado el botón de ``Ok''. 
    \item No introducir datos al insertar las palabras y dar click a añadir.
    \item Comprobar que no ha producido ningún cambio.
    \item Introducir número de filas y columnas mayor que uno.
    \item Comprobar que no ha producido ningún cambio.
    \item Introducir uno o varios posicionamientos de la palabra en la sopa de letras.
    \item Comprobar que no ha producido ningún cambio.
    \item Introducir datos, ejemplo ``pepe''y dar click a añadir.
    \item Comprobar que se ha introducido en la lista de palabras.
    \item Editar palabra introduciendo una cadena vacía.
    \item Comprobar que no ha producido ningún cambio.
    \item Editar palabra introduciendo ``juan''.
    \item Comprobar que la palabra se ha modificado por ``juan''.
    \item Darle a borrar la palabra.
    \item Comprobar que se ha borrado.
    \item Introducir un posicionamiento horizontal e introducir tantas columnas como la longitud de la palabra.
    \item Comprobar que se ha creado la sopa de letras.
    \item Cambia a un posicionamiento vertical e introducir tantas filas como la longitud de la palabra.
    \item Comprobar que se ha creado la sopa de letras.
    \item Cambia a un posicionamiento diagonal e introducir tantas filas y columnas como la longitud de la palabra.
    \item Comprobar que se ha creado la sopa de letras.
    \item Cambia a un posicionamiento al revés e introducir tantas columnas como la longitud de la palabra.
    \item Comprobar que se ha creado la sopa de letras.
    \item Seleccionar todos los tipos de posicionamiento y adecuar el número de columnas y de filas. 
    \item Comprobar que se ha creado la sopa de letras.
    \item Darle a ``Ok''.
    \item Comprobar que se ha cerrado el modal.
    \item Abrir el modal de sopa de letras.
    \item Comprobar que los campos están vacíos.
\end{enumerate}

\section{Funcionalidad de completar huecos}
\label{planPruebas:huecos}
A continuación se expone la tabla (Figura \ref{tab:huecos}) con los casos de prueba de la funcionalidad de completar huecos y el procedimiento para hacer dichos casos.

\begin{table}[H]
    \centering
    \scalebox{0.80}{
    \begin{tblr}{ colspec = {|m{4cm}|c|m{4cm}|m{4cm}|m{4cm}|}}
    \hline
    \hfil\textbf{Precondición} &  \textbf{Campo} & \hfil\textbf{Condición} &  \hfil\textbf{Datos de entrada} &  \hfil\textbf{\begin{tabular}[c]{@{}l@{}}Salida esperada\\ (Postcondición)\end{tabular}} \\ \hline
    \hfil Texto= ``'' & Nuevo texto & Longitud de texto $\leq$  0 & \hfil Texto= ``'' & \hfil Texto=``''\\ \hline
    \hfil Texto= ``'' & Nuevo texto & Longitud de texto > 0 & \hfil Texto =``hola'' & \hfil Texto=``hola''\\ \hline
    \hfil Texto= ``'' & Introducir huecos & Longitud de texto $\leq$ 0 & Tamaño de hueco que pertenece a algún tamaño de hueco disponible & \hfil Texto=``''\\ \hline
  Texto= ``Hola esto es una prueba'' & Introducir huecos & Longitud de enunciado > 0 y el hueco no sea un espacio & Tamaño de hueco que pertenece a algún tamaño de hueco disponible &Texto =``\rule{10mm}{0.1mm} esto es una \rule{10mm}{0.1mm}''\\ \hline
  Texto= ``Hola esto es una prueba'' & Editar Texto & Longitud de texto > 0 & Texto =``Adiós esto es una prueba'' &Texto =``Adiós esto es una prueba''\\ \hline
    \end{tblr}
    }
    \caption{Casos de prueba de la funcionalidad de completar huecos.}
    \label{tab:huecos}
\end{table}

\subsection{Procedimiento de prueba}
\label{procedimientoPruebas:huecos}
\begin{enumerate}
    \item Abrir el modal de completar huecos.
    \item Cerrar el modal.
    \item Abrir el modal de completar huecos.
    \item Comprobar que sin texto introducido no se puede darle a ``Ok''. 
    \item Comprobar que sin texto introducido no se puede darle a añadir huecos.
    \item Insertar texto como por ejemplo ``Hola esto es una prueba'' comprobar que se puede dar click al botón de añadir hueco.
    \item Borrar texto y ver como no se puede darle a ok y no se puede darle a añadir huecos.
    \item Insertar texto como por ejemplo ``Hola esto es una prueba'' comprobar que se puede dar click al botón de añadir hueco.
    \item Dar click al botón de añadir hueco y probar todos los tipos de añadir hueco.
    \item Probar a dar click al boton de editar texto e introducir por ejemplo ``Adiós esto es una prueba''. 
    \item Dar click al botón de añadir hueco y probar todos los tipos de añadir hueco.
    \item Comprobar que no puedes convertir los espacios en huecos.
    \item Dar click al ``Ok''. 
    \item Comprobar que se ha cerrado el modal.
    \item Abrir el modal de completar huecos.
    \item Comprobar que los campos están vacíos.
\end{enumerate}

\section{Funcionalidad de definiciones}
\label{planPruebas:definiciones}
A continuación se expone la tabla (Figura \ref{tab:definiciones}) con los casos de prueba de la funcionalidad de definiciones y el procedimiento para hacer dichos casos.

\begin{table}[H]
    \centering
    \scalebox{0.85}{
    \begin{tblr}{ colspec = {|c|c|m{4cm}|c|c|}}
    \hline
    \textbf{Precondición} &  \textbf{Campo} &  \hfil\textbf{Condición} &  \textbf{Datos de entrada} &   \textbf{\begin{tabular}[c]{@{}l@{}}Salida esperada\\ (Postcondición)\end{tabular}} \\ \hline
    Lista=[ ] & Nueva definición & Longitud de frase $\leq$ 0 & Frase = ``'' & Lista = [ ]\\ \hline
    Lista=[ ] & Nueva definición & Longitud de frase > 0 & Frase = ``hola'' & Lista = [``hola'']\\ \hline
    Lista=[``hola''] & Editar definición &Longitud de frase $\leq$ 0 y la frase existe en la lista & Frase = ``'' & Lista = [``hola'']\\ \hline
    Lista=[``hola''] & Editar definición & Longitud de frase > 0 y la frase existe en la lista & Frase = ``adios'' & Lista = [``adios'']\\ \hline
    Lista=[``hola''] & Borrar definición & Existe la frase a borrar en la lista & Frase = ``hola'' & Lista = [ ]\\ \hline
    NumeroFilas=1 & Insertar número de filas & Número de filas $\leq$ 0 y existe un enunciado & NumeroFilas= -1 & Enunciado = ``''\\ \hline
    NumeroFilas=1 & Insertar número de filas & Número de filas > 0 y existe un enunciado & NumeroFilas= 2 & Enunciado $\neq$ ``'' \\ \hline
    Opción pauta=``'' & Elegir pauta & Número de filas > 0 y existe un enunciado & Opción pauta $\neq$ ``'' & Opción pauta $\neq$ ``''\\ \hline
    \end{tblr}
    }
    \caption{Casos de prueba de la funcionalidad de definiciones.}
    \label{tab:definiciones}
\end{table}

\subsection{Procedimiento de prueba}
\label{procedimientoPruebas:definiciones}
\begin{enumerate}
\item Abrir el modal de definiciones.
\item Cerrar el modal.
\item Abrir el modal de definiciones.
\item Intentar darle al botón de ``Ok'' y comprobar que no está habilitado.
\item No introducir datos al insertar definición y dar click a añadir.
\item Comprobar que no se ha producido ningún cambio.
\item Introducir datos, ejemplo ``hola'' y dar click a añadir.
\item Comprobar que se ha introducido en la lista de frases.
\item Editar palabra introducida escribiendo una cadena vacía.
\item Comprobar que no se ha cambiado el valor.
\item Editar la palabra introducida escribiendo ``adios''.
\item Comprobar que se ha cambiado el valor al esperado.
\item Darle a borrar la definición.
\item Comprobar que se ha borrado.
\item Introducir datos, ejemplo ``pepe'' y dar click a añadir.
\item Comprobar que se ha introducido en la lista de definiciones.
\item Introducir un número de filas menor o igual que cero y comprobar que no se dibujan en la vista previa.
\item Introducir un número de filas mayor que 100 y comprobar que no se dibujan en la vista previa.
\item Introducir un número de filas mayor que cero pero menor o igual que 100 y comprobar que se dibuja el número indicado de filas.
\item Probar todos los tipos de pautas y comprobar que se dibuja el tipo de pauta indicado en cada caso.
\item Darle a ``Ok''.
\item Comprobar que se ha cerrado el modal y se ha añadido el ejercicio al editable.
\item Abrir el modal de definiciones.
\item Comprobar que los campos están vacíos.
\end{enumerate}

\section{Funcionalidad de generar resumen}
\label{planPruebas:resumen}
A continuación se expone la tabla (Figura \ref{tab:resumen}) con los casos de prueba de la funcionalidad de generar resumen y el procedimiento para hacer dichos casos.

\begin{table}[H]
    \centering
    \scalebox{0.85}{
    \begin{tblr}{ colspec = {|c|c|m{4cm}|m{4cm}|m{4cm}|}}
    \hline
  \textbf{Precondición} &  \textbf{Campo} & \hfil\textbf{Condición} &  \hfil\textbf{Datos de entrada} &  \hfil\textbf{\begin{tabular}[c]{@{}l@{}}Salida esperada\\ (Postcondición)\end{tabular}} \\ \hline
  TextoResumido= ``'' & Texto Original & Longitud del TextoOriginal $\leq$ 0 & \hfil TextoOriginal= ``'' &  \hfil TextoResumido= ``''\\ \hline
  TextoResumido= ``'' & Texto Original & Longitud del TextoOriginal > 0 & \hfil TextoOriginal= ``...'' & TextoResumido= ``'' siendo Longitud del TextoResumido < Longitud del TextoOriginal\\ \hline
  TextoResumido= ``'' & Texto Original & Longitud del TextoOriginal > 0 & TextoOriginal= ``dgsgds'' & \hfil TextoResumido= ``''\\ \hline
  TextoResumido= ``'' & Tamaño & Tamaño $\leq$ 0 &  TextoOriginal= ``...'' & \hfil TextoResumido= ``''\\ \hline
  TextoResumido= ``'' & Tamaño & Tamaño > 0 &  TextoOriginal= ``...'' y Tamaño=10 & TextoResumido= ``...'' con longitud de TextoResumido  = 10\\ \hline
    \end{tblr}
    }
    \caption{Casos de prueba de la funcionalidad de generar resumen.}
    \label{tab:resumen}
\end{table}

\subsection{Procedimiento de prueba}
\label{procedimientoPruebas:resumen}
\begin{enumerate}
    \item Abrir el modal de Generar Resumen.
    \item Cerrar el modal.
    \item Abrir el modal de Generar Resumen.
    \item No introducir texto original y dar click a resumir.
    \item Comprobar que no ha producido ningún cambio.
    \item Introducir un texto, elegir una cantidad de palabras menor o igual que cero y darle a resumir.
    \item Cerrar el modal.
    \item Abrir el modal de Generar Resumen.
    \item Comprobar que se ha reseteado el modal.
    \item Introducir un texto, elegir una cantidad de palabras menor o igual que cero y darle a resumir.
    \item Comprobar que no ha producido ningún cambio.
    \item Introducir un texto, elegir una cantidad de palabras mayor que cero y darle a resumir.
    \item Comprobar que se ha devuelto lo especificado. 
    \item Comprobar que los campos están vacíos.
    \item Introducir un texto no coherente, elegir una cantidad de palabras mayor que cero y darle a resumir.
    \item Comprobar que no ha producido ningún cambio.
    
\end{enumerate}


\section{Funcionalidad de leyenda de colores}
\label{planPruebas:leyenda}
A continuación se expone la tabla (Figura \ref{tab:leyenda}) con los casos de prueba de la funcionalidad de leyenda de colores y el procedimiento para hacer dichos casos.

\begin{table}[H]
    \centering
    \scalebox{0.85}{
    \begin{tblr}{ colspec = {|c|c|m{4cm}|c|c|}}
    \hline
    \textbf{Precondición} &  \textbf{Campo} &  \hfil\textbf{Condición} &  \textbf{Datos de entrada} &   \textbf{\begin{tabular}[c]{@{}l@{}}Salida esperada\\ (Postcondición)\end{tabular}} \\ \hline
    Lista=[ ] & Nuevo concepto & Longitud de frase $\leq$ 0 & Frase = ``'' & Lista = [ ]\\ \hline
    Lista=[ ] & Nuevo concepto & Longitud de frase > 0 & Frase = ``hola'' & Lista = [``hola''(0,0,0)]\\ \hline
    Lista=[``hola''(0,0,0)] & Editar concepto &Longitud de frase $\leq$ 0 y la frase existe en la lista & Frase = ``'' & Lista = [``hola''(0,0,0)]\\ \hline
    Lista=[``hola''(0,0,0)] & Editar concepto & Longitud de frase > 0 y la frase existe en la lista & Frase = ``adios'' & Lista = [``adios''(0,0,0)]\\ \hline
    Lista=[``hola''(0,0,0)] & Editar color & EColor válido
    y la frase existe en la lista & Color = (255,0,0) & Lista = [``hola''(255,0,0)]\\ \hline
    Lista=[``hola''(0,0,0)] & Borrar concepto & Existe la frase a borrar en la lista & Frase = “hola” & Lista = [ ]\\ \hline
    \end{tblr}
    }
    \caption{Casos de prueba de la funcionalidad de leyenda de colores.}
    \label{tab:leyenda}
\end{table}

\subsection{Procedimiento de prueba}
\label{procedimientoPruebas:leyenda}
\begin{enumerate}
\item Abrir el modal.
\item Cerrar el modal.
\item Abrir el modal otra vez.
\item Intentar darle al botón de ``Ok'' y comprobar que no está habilitado.
\item No introducir datos al insertar un concepto y dar click a añadir.
\item Comprobar que no se ha producido ningún cambio.
\item Introducir datos, ejemplo ``hola'' y dar click a añadir.
\item Comprobar que se ha introducido en la lista de conceptos.
\item Editar palabra introducida escribiendo una cadena vacía.
\item Comprobar que no se ha cambiado el valor.
\item Editar la palabra introducida escribiendo ``adios''.
\item Comprobar que se ha cambiado el valor al esperado.
\item Editar el color seleccionado un color diferente.
\item Comprobar que se ha cambiado el color.
\item Darle a borrar el concepto.
\item Comprobar que se ha borrado.
\item Introducir datos, ejemplo ``pepe'' y dar click a añadir.
\item Comprobar que se ha introducido en la lista de definiciones.
\item Darle a ``Ok''.
\item Comprobar que se ha cerrado el modal y se ha añadido la leyenda al editable.
\item Abrir el modal.
\item Comprobar que los campos están vacíos.
\end{enumerate}

\section{Funcionalidad de espacios para dibujar}
\label{planPruebas:dibujar}
A continuación se expone la tabla (Figura \ref{tab:dibujar}) con los casos de prueba de la funcionalidad de espacios para dibujar y el procedimiento para hacer dichos casos.

\begin{table}[H]
    \centering
    \scalebox{0.85}{
    \begin{tblr}{ colspec = {|c|c|m{4cm}|c|c|}}
    \hline
  \textbf{Precondición} &  \textbf{Campo} & \hfil\textbf{Condición} &  \textbf{Datos de entrada} &  \textbf{\begin{tabular}[c]{@{}l@{}}Salida esperada\\ (Postcondición)\end{tabular}} \\ \hline
  Enunciado= ``'' & Insertar enunciado & Longitud de enunciado $\leq$  0 & Enunciado=``'' & Enunciado=``''\\ \hline
  Enunciado= ``'' & Insertar enunciado & Longitud de enunciado > 0 & Enunciado=``hola'' & Enunciado=``hola''\\ \hline
  Enunciado= ``hola'' & Insertar enunciado & Longitud de enunciado $\leq$ 0 & Enunciado=``'' & Enunciado=``''\\ \hline
  Enunciado= ``hola'' & Insertar enunciado & Longitud de enunciado > 0 & Enunciado=``adios'' &Enunciado=``adios''\\ \hline
  espacio=1 & Insertar espacio & Número de filas $\leq$ 0 y existe un enunciado & espacio=-1 & Enunciado = ``''\\ \hline
  espacio=1 & Insertar espacio & Número de filas > 0 y existe un enunciado & espacio=2 & Enunciado $\neq$ ``''\\ \hline
    \end{tblr}
    }
    \caption{Casos de prueba de la funcionalidad de espacios para dibujar.}
    \label{tab:dibujar}
\end{table}

\subsection{Procedimiento de prueba}
\label{procedimientoPruebas:dibujar}
\begin{enumerate}
    \item Abrir el modal de espacios para dibujar.
    \item Cerrar el modal.
    \item Abrir el modal de espacios para dibujar.
    \item Comprobar que no está habilitado el botón de ``Ok''.
    \item No introducir datos al insertar enunciado y comprobar que no hay nada en la vista previa. 
    \item Introducir espacio mayor que cero y comprobar que no hay nada en la vista previa.
    \item Introducir espacio menor que cero y comprobar que no hay nada en la vista previa.
    \item Introducir texto en el insertar enunciado, introducir espacio  mayor que cero y comprobar que se ve en la vista previa. 
    \item Probar a cambiar enunciado y comprobar que el cambio se refleja en la vista previa.
    \item Desmarcar recuadro y ver en vista previa que no hay recuadro.
    \item Marcar recuadro y ver que aparece en vista previa.
    \item Darle a ``Ok''.
    \item Comprobar que se ha cerrado el modal.
    \item Abrir el modal de  espacios para dibujar.
    \item Comprobar que los campos están vacíos. 
\end{enumerate}

\section{Funcionalidad de pictotraductor}
\label{planPruebas:pictotraductor}
A continuación se expone la tabla (Figura \ref{tab:pictotraductor}) con los casos de prueba de la funcionalidad de pictotraductor y el procedimiento para hacer dichos casos.

\begin{table}[H]
    \centering
    \scalebox{0.80}{
    \begin{tblr}{ colspec = {|m{4cm}|c|m{4cm}|m{4cm}|m{4cm}|}}
    \hline
  \textbf{Precondición} &  \hfil\textbf{Campo} & \hfil\textbf{Condición} &  \hfil\textbf{Datos de entrada} &  \hfil\textbf{\begin{tabular}[c]{@{}l@{}}Salida esperada\\ (Postcondición)\end{tabular}} \\ \hline
  \hfil TextoOriginal= ``'' & Insertar texto & Longitud de texto $\leq$  0 &  \hfil Texto=``'' & \hfil TextoOriginal=``''\\ \hline
  \hfil  TextoOriginal= ``'' & Insertar texto & Longitud de texto > 0 & Texto=``La libélula tiene cuatro alas'' & TextoOriginal=``La libélula tiene cuatro alas''\\ \hline
  TextoOriginal=``La libélula tiene cuatro alas'' & Insertar texto & Longitud de texto $\leq$ 0 &  \hfil Texto=``'' &  \hfil TextoOriginal=``''\\ \hline
  TextoOriginal=``La libélula tiene cuatro alas''' & Insertar texto & Longitud de texto > 0 & Texto=``descomponer los alimentos para obtener sus nutrientes''& TextoOriginal=`` descomponer los alimentos para obtener sus nutrientes''\\ \hline
  TextoOriginal=``La libélula tiene cuatro alas'' & Opcionesde texto &Longitud de texto > 0 & Opciones de texto = arriba & Pictogramas con texto encima del pictograma \\ \hline
  TextoOriginal=``La libélula tiene cuatro alas'' & Opciones de texto & Longitud de texto > 0 & Opciones de texto = debajo & Pictogramas con texto debajo del pictograma \\ \hline
  TextoOriginal=``La libélula tiene cuatro alas'' & Opciones de texto & Longitud de texto > 0 & Opciones de texto=sin texto & Pictogramas sin texto \\ \hline
  TextoOriginal=``La libélula tiene cuatro alas'' & Opciones de color  & Longitud de texto > 0 & Opciones de color=blanco y negro & Pictograma en blanco y negro\\ \hline
  TextoOriginal=``La libélula tiene cuatro alas'' & Vista pictograma & Longitud de texto > 0  & Visualización de un  pictograma oculta & Pictograma oculto \\ \hline
  TextoOriginal=``La libélula tiene cuatro alas'' & Vista pictograma & Longitud de texto > 0  & Visualización de todos los pictograma oculta & Botón ``Ok'' deshabilitado\\ \hline
    \end{tblr}
    }
    \caption{Casos de prueba de la funcionalidad pictotraductor.}
    \label{tab:pictotraductor}
\end{table}

\subsection{Procedimiento de prueba}
\label{procedimientoPruebas:pictotraductor}
\begin{enumerate}
    \item Abrir el modal de pictotraductor.
    \item Cerrar el modal.
    \item Abrir el modal de pictotraductor.
    \item Comprobar que no está habilitado el botón de ``Ok''.
    \item No introducir datos al insertar enunciado y comprobar que no hay nada en la vista previa. 
    \item Introducir cadena vacía y comprobar que no hay nada en la vista previa
    \item Introducir ``La libélula tiene cuatro alas'' y comprobar que aparecen los respectivos pictogramas.
    \item Introducir ``La libélula tiene cuatro alas'' y comprobar que aparecen los respectivos pictogramas y cambiar el texto por ``descomponer los alimentos''. para obtener sus nutrientes y ver que los pictogramas se han creado. 
    \item Introducir ``La libélula tiene cuatro alas'', poner como opción de texto arriba y comprobar que aparecen los pictogramas con el texto arriba. 
    \item Introducir ``La libélula tiene cuatro alas'', poner como opción de texto debajo y comprobar que aparecen los pictogramas con el texto debajo.
    \item Introducir ``La libélula tiene cuatro alas'', poner como opción de texto sin texto y comprobar que aparecen los pictogramas sin texto.
    \item Introducir ``La libélula tiene cuatro alas'', poner como opción de color blanco y negro  y comprobar que aparecen los pictogramas en blanco y negro.
    \item Introducir ``La libélula tiene cuatro alas'', comprobar que aparecen los pictogramas y ocultar uno.
    \item Introducir ``La libélula tiene cuatro alas'', comprobar que aparecen los pictogramas y ocultar todos, ver que el botón de ``Ok'' no está habilitado.
    \item Darle a ``Ok''.
    \item Comprobar que se ha cerrado el modal.
    \item Abrir el modal de  pictotraductor.
    \item Comprobar que los campos están vacíos.
\end{enumerate}

\section{Funcionalidad de relacionar conceptos}
\label{planPruebas:conceptos}
A continuación se expone la tabla (Figura \ref{tab:conceptos}) con los casos de prueba de la funcionalidad de relacionar conceptos y el procedimiento para hacer dichos casos.

\begin{table}[H]
    \centering
    \scalebox{0.76}{
    \begin{tblr}{ colspec = {|m{4cm}|c|m{4cm}|m{4cm}|m{4cm}|}}
    \hline
    \hfil \textbf{Precondición} &  \hfil\textbf{Campo} & \hfil\textbf{Condición} &  \hfil\textbf{Datos de entrada} &  \hfil\textbf{\begin{tabular}[c]{@{}l@{}}Salida esperada\\ (Postcondición)\end{tabular}} \\ \hline
  NumFilas=0 y NumCols=0  & Insertar filas y columnas & NumFilas $\leq$ 1 y NumCols $\leq$ 1 0 & NumFilas=0 y NumCols=0 & La tabla de conceptos no sale\\ \hline
  NumFilas=0 y NumCols=0  & Insertar filas y columnas & 1 < NumFilas < 20 y 1 < NumCols< 2 & NumFilas=2 y NumCols=2 & Dos columnas con dos filas\\ \hline
  NumFilas=2 y NumCols=2 & Insertar filas y columnas & 1 < NumFilas < 20 y 1 < NumCols< 2 & En la tabla añadir: ``animales'', ``vaca'', ``gato'' & Tabla con esos datos\\ \hline
  NumFilas=2 y NumCols=2 & Insertar filas y columnas & 1 < NumFilas < 20 y 1 < NumCols< 2 & En la tabla añadir cadenas vacías & Botón de ``Ok'' deshabilitado \\ \hline
  En una columna: ``animales'', en otra: ``gato'', ``vaca'' y NumFilas=2 y NumCols=2 & Reordenar &  & En la tabla: ``animales'', ``vaca'', ``gato'' & En una columna: ``animales'' en otra: ``gato'', ``vaca''\\ \hline
  Tabla vacía  & Reordenar &  & Tabla vacía & Tabla vacía\\ \hline
    \end{tblr}
    }
    \caption{Casos de prueba de la funcionalidad de relacionar conceptos.}
    \label{tab:conceptos}
\end{table}

\subsection{Procedimiento de prueba}
\label{procedimientoPruebas:conceptos}
\begin{enumerate}
\item Abrir el modal de relacionar conceptos.
\item Cerrar el modal.
\item Abrir el modal de relacionar conceptos.
\item Comprobar que no está habilitado el botón de ``Ok''.
\item Introducir número de columnas y filas  menores a 1.
\item Comprobar que las columnas y filas no aparecen.
\item Introducir número de columnas y filas mayores que 1.
\item Comprobar que ese número de  las columnas y filas aparecen como tabla dividida en columnas.
\item Introducir texto en las celdas. 
\item Ver que aparece en la vista previa.
\item Reordenar y ver que se desordenan.
\item Eliminar los concepto puestos en las celdas 
\item Poner en las celdas espacios en blanco
\item Ver que el botón de ``Ok'' está deshabilitado
\item Volver a poner conceptos en las celdas
\item Darle a ``Ok''.
\item Comprobar que se ha cerrado el modal.
\item Abrir el modal de relacionar conceptos.
\item Comprobar que los campos están vacíos.
    
\end{enumerate}


\section{Funcionalidad de ejercicios de matemáticas}
\label{planPruebas:mate}
A continuación se expone la tabla (Figura \ref{tab:mate}) con los casos de prueba de la funcionalidad de ejercicios de matemáticas y el procedimiento para hacer dichos casos.



\subsection{Procedimiento de prueba}
\label{procedimientoPruebas:mate}
\begin{enumerate}
    \item Abrir el modal de Fórmula matemática.
    \item Comprobar que se ha abierto el modal.
    \item Cerrar el modal.
    \item Comprobar que se ha cerrado el modal.
    \item Abrir el modal de Fórmula matemática.
    \item Comprobar que no está habilitado el botón de ``Ok''.
    \item Introducir datos en el primer hueco, ejemplo ``123''.
    \item Comprobar que se ha habilitado el botón de ``Ok''.
    \item Pulsar el botón de ``Ok''.
    \item Comprobar que se ha cerrado el modal y se ha introducido el texto correctamente en el documento de trabajo.
    \item Abrir el modal de Fórmula matemática.
    \item Pulsar espacio varias veces (Probar manteniendo pulsado el espacio también).
    \item Comprobar que se han creado nuevos huecos y que el cursor está en la última posición.
    \item Pulsar la tecla de retroceso varias veces (Probar manteniendo pulsada la tecla de retroceso también).
    \item Comprobar que se han borrado huecos y que el cursor está en la misma posición.
    \item Escribir una fórmula con huecos, por ejemplo`` 2 + \rule{10mm}{0.1mm} = 5''.
    \item Pulsar el botón de ``Ok''.
    \item Comprobar que se ha cerrado el modal y se ha introducido correctamente el texto en el documento de trabajo.
    \item Introducir datos, ejemplo ``123''.
    \item Añadir varios huecos.
    \item Cerrar el modal.
    \item Abrir el modal de Fórmula matemática.
    \item Comprobar que los campos están vacíos.
\end{enumerate}


\section{Funcionalidad de exportar a PDF}
\label{planPruebas:exportar}
A continuación se expone la tabla (Figura \ref{tab:exportar}) con los casos de prueba de la funcionalidad de exportar a PDF y el procedimiento para hacer dichos casos.

\begin{table}[H]
    \centering
    \scalebox{0.76}{
    \begin{tblr}{ colspec = {|m{4cm}|c|m{4cm}|m{4cm}|m{4cm}|}}
    \hline
    \hfil \textbf{Precondición} &  \hfil\textbf{Campo} & \hfil\textbf{Condición} &  \hfil\textbf{Datos de entrada} &  \hfil\textbf{\begin{tabular}[c]{@{}l@{}}Salida esperada\\ (Postcondición)\end{tabular}} \\ \hline
  \hfil Editor vacío  & Insertar ejercicio & Editor vacío & Ejercicio de V/F & Ejercicio numerado, que salgan las frases con la opción de V/F y el enunciado en negrita\\ \hline
  Editor con ejercicio de V/F  & Insertar ejercicio & Editor con ejercicio de V/F & Ejercicio de definiciones con pictogramas & Ejercicio numerado y el enunciado en negrita. Cada concepto debe tener su respectivo tipo de pauta y junto a cada concepto su programa\\ \hline
  Ejercicios anteriormente creados & Insertar ejercicio & Ejercicios anteriormente creados  &Ejercicio de desarrollo & Ejercicio numerado y el enunciado en negrita. Debe aparecer la pauta seleccionada\\ \hline
  Ejercicios anteriormente creados  & Insertar ejercicio & Ejercicios anteriormente creados  & Ejercicio de huecos  & Ejercicio numerado y el enunciado en negrita. Debe aparecer en el texto las líneas representando los huecos \\ \hline
  Ejercicios anteriormente creados & Insertar ejercicio & Ejercicios anteriormente creados & Ejercicio de relacionar conceptos & Ejercicio numerado y el enunciado en negrita. Debe aparecer los conceptos junto a su viñeta \\ \hline
  Ejercicios anteriormente creados  & Insertar ejercicio & Ejercicios anteriormente creados & Sopa de letras & Ejercicio numerado , el enunciado en negrita y la sopa de letras recuadrada con una letra en cada recuadro\\ \hline
  Ejercicios anteriormente creados  & Insertar ejercicio & Ejercicios anteriormente creados & Fórmulas matemáticas & Ejercicio numerado , el enunciado en negrita, de forma numerada cada fórmula con los huecos\\ \hline
  Ejercicios anteriormente creados  & Insertar ejercicio & Ejercicios anteriormente creados & Ejercicio espacio para dibujar & Ejercicio numerado y el enunciado en negrita. Un recuadro para dibujar o el espacio sin recuadro\\ \hline
  Ejercicios anteriormente creados  & Insertar ejercicio & Ejercicios anteriormente creados & Resumen & Texto resumido\\ \hline
  Ejercicios anteriormente creados  & Insertar ejercicio & Ejercicios anteriormente creados & Leyenda de colores  & Ejercicio numerado y título de la leyenda en negrita. Al lado de cada concepto un cuadrado con el color seleccionado\\ \hline Ejercicios anteriormente creados  & Insertar ejercicio & Ejercicios anteriormente creados & Pictotraductor & Pictogramas con el texto o sin él, con color o en blanco y negro, con un borde y en el caso de no haber aparece el texto sin borde\\ \hline
    \end{tblr}
    }
    \caption{Casos de prueba de la funcionalidad de exportar a PDF.}
    \label{tab:exportar}
\end{table}

\subsection{Procedimiento de prueba}
\label{procedimientoPruebas:exportar}
\begin{enumerate}
    \item Introducir ejercicio de V/F.
    \item Introducir ejercicio de definiciones.
    \item Introducir un pictograma al lado de cada concepto a definir. 
    \item Introducir ejercicio de desarrollo.
    \item Introducir ejercicio de completar huecos.
    \item Introducir ejercicio de relacionar conceptos.
    \item Introducir ejercicio de sopa de letras.
    \item Introducir ejercicio de fórmulas matemáticas.
    \item Introducir ejercicio de espacios para dibujar.
    \item Introducir resumen.
    \item Introducir leyenda de colores.
    \item Introducir una frase transformada a pictogramas.
    \item Darle al botón de exportar a PDF.
    \item Comprobar que los enunciados están en negrita y los ejercicios están numerados.
    \item Comprobar en el ejercicio de verdadero y falso que las frases generadas están numeradas y tienen al final (V/F).
    \item Comprobar que en ejercicio de definiciones y desarrollo aparece el tipo de pauta.
    \item Comprobar que en el ejercicio de completar huecos y en el de fórmulas matemáticas aparecen las líneas haciendo referencia a los huecos.
    \item Comprobar que en el ejercicio de relacionar conceptos cada concepto tiene su propia viñeta.
     \item Comprobar que en la sopa de letras está recuadrada y en cada recuadro hay una letra.
    \item Comprobar que en el ejercicio de espacio para dibujar tiene el recuadro, si se ha marcado, o un espacio en blanco si no se ha marcado la opción de recuadro.
    \item Comprobar que en la leyenda de colores se muestran los colores al lado de los conceptos. 
    \item Comprobar en la traducción de un texto a pictogramas si salen las palabras en la posición indicada, en el caso de que se haya seleccionado que se muestre las palabras. También comprobar que tienen color o no según se ha elegido. Y aquellas palabras que no tengan traducción pictograma aparecen como texto plano. 

\end{enumerate}
