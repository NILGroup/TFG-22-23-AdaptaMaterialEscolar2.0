\chapter{Diseños individuales para la iteración competitiva de Alberto Alejandro Rivas Fernandez}
\label{ape:disenyoAlberto}

En este apéndice se muestran los diseños individuales realizados por Alberto Alejandro Rivas Fernandez para la iteración competitiva.

En primer lugar, en la figura \ref{AlbertoPaginaPrincipal1} se muestra la página principal antes de haber seleccionado un documento PDF, por lo que hay un elemento en el que podemos subir un fichero. Luego en la figura \ref{AlbertoPaginaPrincipal2}  tenemos la página principal después de haber seleccionado un documento, en la parte izquierda se muestra dicho documento, y en la parte izquierda tenemos el editor de texto con todos los botones para acceder a las funcionalidades en la parte superior. 

En la figura \ref{Alberto13} se muestra el ejercicio de huecos, en el que el usuario puede escribir un texto y seleccionar qué palabras quiere sustituir por un hueco. En la figura \ref{Alberto14} se observa el ejercicio de definiciones, en el que el usuario puede añadir una serie de definiciones e indicar la cantidad de líneas que quiere insertar para cada definición, junto con otras opciones. En la figura \ref{Alberto15} está el ejercicio de desarrollo, que consta de un área de texto en la que se puede escribir el enunciado del ejercicio, también se puede indicar la cantidad de líneas cómo en la figura anterior, junto con otras opciones. En la figura \ref{Alberto16} se presenta el ejercicio de sopa de letras, en el que se puede indicar el número de filas, columnas y palabras, para luego insertar las palabras que se quieren mostrar en la sopa de letras. En la figura \ref{Alberto17} se encuentra el ejercicio de verdadero y falso, que consta de un input con el que el usuario puede añadir las frases que quiera insertar en el ejercicio. En la figura \ref{Alberto1} se muestra la leyenda de colores, que permite añadir una serie de frases acompañadas de un color que puede ser seleccionado por el usuario. En la figura \ref{Alberto2} se observa la leyenda de colores por asignatura, en este caso hay un solo selector de color, este color se usará para añadir un borde a la página, el cual indica la asignatura a la que pertenece. En la figura \ref{Alberto3} está el modal para crear una cuadrícula, este simplemente consta de un input en el que se puede indicar el número de filas que se quiere tener en la cuadrícula. En la figura \ref{Alberto4} se presenta el modal para crear insertar líneas doble pauta en el documento, este simplemente consta de un input en el que se puede indicar el número de líneas que se quieren insertar. En la figura \ref{Alberto5} se encuentra el ejercicio de flechas, el cual tiene dos columnas con varios inputs donde el usuario puede escribir los conceptos a relacionar, también tiene un botón en la parte inferior para añadir nuevas filas. En la figura \ref{Alberto6} se muestra el ejercicio de matemática con huecos, este modal consta de un campo de texto en el que se puede escribir una fórmula con LaTeX. En la figura \ref{Alberto8} se observa el modal de generar resumen, en este hay un campo en el que el usuario puede insertar un texto, y al dar click al botón de resumir, será resumido automáticamente. En la figura \ref{Alberto10} está el modal de pictotraductor, en este hay un campo en el que el usuario puede insertar un texto, y al dar click al botón de aceptar, será traducido a pictogramas. En la figura \ref{Alberto7} se presenta el modal de espacio para dibujar, en el que hay dos campos en los cuales se puede indicar la altura y anchura que tendrá el espacio. Por último, en la figura \ref{Alberto12} se encuentra el buscador de pictogramas, que consta de una barra de búsqueda y los resultados mostrados en forma de cuadrícula.


\begin{figure}[ht!]
  \centering
  \includegraphics[width=0.6\textwidth]{Diseño/Alberto/PaginaPrincipal1.PNG}
  \caption{Diseño página principal de Alberto.}
  \label{AlbertoPaginaPrincipal1}
\end{figure}

\begin{figure}[ht!]
  \centering
  \includegraphics[width=0.6\textwidth]{Diseño/Alberto/PaginaPrincipal2.PNG}
  \caption{Diseño página principal de Alberto.}
  \label{AlbertoPaginaPrincipal2}
\end{figure}


\begin{figure}[ht!]
  \centering
  \includegraphics[width=0.6\textwidth]{Diseño/Alberto/Capture13.PNG}
  \caption{Diseño de ejercicios con huecos de Alberto.}
  \label{Alberto13}
\end{figure}

\begin{figure}[ht!]
  \centering
  \includegraphics[width=0.6\textwidth]{Diseño/Alberto/Capture14.PNG}
  \caption{Diseño de ejercicios de definiciones de Alberto.}
  \label{Alberto14}
\end{figure}

\begin{figure}[ht!]
  \centering
  \includegraphics[width=0.6\textwidth]{Diseño/Alberto/Capture15.PNG}
  \caption{Diseño de ejercicios de desarrollo de Alberto.}
  \label{Alberto15}
\end{figure}

\begin{figure}[ht!]
  \centering
  \includegraphics[width=0.6\textwidth]{Diseño/Alberto/Capture16.PNG}
  \caption{Diseño de ejercicios de sopa de letras de Alberto.}
  \label{Alberto16}
\end{figure}

\begin{figure}[ht!]
  \centering
  \includegraphics[width=0.6\textwidth]{Diseño/Alberto/Capture17.PNG}
  \caption{Diseño de ejercicios de verdadero y falso de Alberto.}
  \label{Alberto17}
\end{figure}


\begin{figure}[ht!]
  \centering
  \includegraphics[width=0.6\textwidth]{Diseño/Alberto/Capture01.PNG}
  \caption{Diseño de leyenda de colores de Alberto.}
  \label{Alberto1}
\end{figure}

\begin{figure}[ht!]
  \centering
  \includegraphics[width=0.6\textwidth]{Diseño/Alberto/Capture02.PNG}
  \caption{Diseño de leyenda de colores por asignatura de Alberto.}
  \label{Alberto2}
\end{figure}

\begin{figure}[ht!]
  \centering
  \includegraphics[width=0.6\textwidth]{Diseño/Alberto/Capture03.PNG}
  \caption{Diseño de cuadrícula de Alberto.}
  \label{Alberto3}
\end{figure}

\begin{figure}[ht!]
  \centering
  \includegraphics[width=0.6\textwidth]{Diseño/Alberto/Capture04.PNG}
  \caption{Diseño de doble pauta de Alberto.}
  \label{Alberto4}
\end{figure}

\begin{figure}[ht!]
  \centering
  \includegraphics[width=0.6\textwidth]{Diseño/Alberto/Capture05.PNG}
  \caption{Diseño de ejercicio de flechas de Alberto.}
  \label{Alberto5}
\end{figure}

\begin{figure}[ht!]
  \centering
  \includegraphics[width=0.6\textwidth]{Diseño/Alberto/Capture06.PNG}
  \caption{Diseño de ejercicios de matemáticas con huecos de Alberto.}
  \label{Alberto6}
\end{figure}

\begin{figure}[ht!]
  \centering
  \includegraphics[width=0.6\textwidth]{Diseño/Alberto/Capture08.PNG}
  \caption{Diseño de añadir resumen de Alberto.}
  \label{Alberto8}
\end{figure}

\begin{figure}[ht!]
  \centering
  \includegraphics[width=0.6\textwidth]{Diseño/Alberto/Capture10.PNG}
  \caption{Diseño de pictotraductor de Alberto.}
  \label{Alberto10}
\end{figure}

\begin{figure}[ht!]
  \centering
  \includegraphics[width=0.6\textwidth]{Diseño/Alberto/Capture07.PNG}
  \caption{Diseño de ejercicios de espacios para dibujar de Alberto.}
  \label{Alberto7}
\end{figure}

\begin{figure}[ht!]
  \centering
  \includegraphics[width=0.6\textwidth]{Diseño/Alberto/Capture12.PNG}
  \caption{Diseño de buscar pictogramas de Alberto.}
  \label{Alberto12}
\end{figure}