\chapter{Trabajo Individual}
\label{cap:TrabajoIndividual}

En este capítulo se habla del trabajo que ha realizado cada miembro del equipo en el proyecto.

\section{Álvaro Gómez Sittima}
Durante el desarrollo de este TFG, he colaborado en la creación del código y la redacción de la memoria. Me he involucrado tanto de manera individual como en grupo para alcanzar los objetivos establecidos.

En cuanto al código, he realizado diversas tareas de forma individual, tales como investigar alternativas a CKEditor, proponiendo Quill.js, aunque finalmente nos decantamos por la opción que propuso Johan (Slate) ya que se integraba mejor con React. También he propuesto e integrado Tailwind CSS como framework CSS para la aplicación. Además, he realizado la refactorización y limpieza del código para facilitar la escalabilidad y la implementación de nuevas funcionalidades en el futuro. Asimismo, he creado distintos componentes reutilizables para facilitar la creación de nuevas ventanas modales, como por ejemplo el propio componente de la ventana modal, botones o la vista previa. En cuanto a las funcionalidades, me he encargado de desarrollar y arreglar los bugs de las funcionalidades de crear ejercicios de completar huecos, sopa de letras y generar resúmenes. También he realizado las pruebas de las funcionalidades de verdadero/falso y ejercicios de matemáticas con huecos. Además, he integrado herramientas destinadas a mejorar la calidad del código, tales como Prettier, con el fin de definir un estilo común en todo el proyecto, y ESLINT, para llevar a cabo análisis estáticos del código. En colaboración con el grupo, hemos desarrollado la estructura principal de la aplicación, incluyendo la página de inicio, la barra de navegación y el editor. También hemos desarrollado la función de buscar pictograma.

Con respecto a la memoria, he contribuido individualmente en la redacción de diferentes secciones de la memoria, tales como la Sección \ref{sec:herramientasexistentes}, en la cual se realiza una comparativa con otras aplicaciones similares a AdaptaMaterialEscolar, la Sección \ref{cap:pruebas} donde se explica el plan de pruebas seguido en el proyecto y la Sección \ref{sec:tailwind}, en la cual se explica el framework Tailwind CSS. Además, he redactado la implementación de las funcionalidades que he desarrollado: completar huecos (Sección \ref{sec:impcompletarhuecos}), sopa de letras (Sección \ref{sec:impsopaletras}) y generar resumen (Sección \ref{sec:impresumen}). También, he realizado diseños para la iteración competitiva, Sección \ref{sec:iterAlvaro}. Junto a mis compañeros, he participado en la definición de los objetivos (Sección \ref{cap:objetivos}), la definición de requisitos (Sección \ref{cap:requisitos}), el diseño final de la aplicación (Sección \ref{subsec:DisenyoFinal}), la metodología empleada (Sección \ref{cap:metodologia}), la arquitectura de la aplicación (Sección \ref{sub:Arquitectura}) y la implementación de la funcionalidad de buscar pictogramas (Sección \ref{sec:impbuscarpicto}). En cuanto al diseño de las funcionalidades, he colaborado con Johan en la realización de los diseños en Figma.

\section{Dunia Namour Doughani}
En mi contribución a la memoria del proyecto, me encargué de redactar la Sección \ref{cap:adaptacion} donde se explica la adaptación curricular y sus tipos, además de escribir la estructura del proyecto, Sección \ref{cap:estructura}. También expliqué la herramienta React en el capítulo de herramientas empleadas, Sección \ref{sec:React}, los diseños que realicé para la iteración competitiva, Sección \ref{sec:duniaIter}, y la implementación de la funcionalidad de verdadero y falso \ref{sec:funcioVF}. Además, incluí las imágenes de las funcionalidades diseñadas en Figma y diseñé cómo se verían los ejercicios en el documento de trabajo.

En la parte de la memoria junto a mis compañeros establecimos los objetivos del proyecto, Sección \ref{cap:objetivos} y redactamos de la sección de metodología de desarrollo todo lo relacionado con el tablero Kanban y las políticas explícitas. También colaboramos en la sección de Adaptamaterial 2.0, donde definimos los requisitos, el diseño final de las funcionalidades y la arquitectura del proyecto. Y juntos escribimos el capítulo de introducción en inglés.

En mi trabajo individual en el código, implementé el modal de la funcionalidad de verdadero y falso y el de la leyenda de colores. Además, realicé pruebas y creé el plan de pruebas para las funcionalidades de sopa de letras y buscar pictograma. También refactoricé la funcionalidad de verdadero y falso para usar Tailwind en lugar de CSS, corregí los errores de leyenda de colores y de la funcionalidad de verdadero y falso, y realicé un segundo test a las funcionalidades de desarrollo y definiciones.

Como trabajo en equipo, escribimos el plan de pruebas para la funcionalidad de verdadero y falso y ayudé en la refactorización del código.

\section{Alberto Alejandro Rivas Fernandez}
Al principio del proyecto, uno de los desafíos que enfrentamos fue la necesidad de implementar la funcionalidad de exportar documentos en formato Word y PDF. CKEditor tiene un plugin que nos permite hacer esto, pero necesitábamos una licencia de pago para poder utilizarlo. Me puse en contacto con el equipo de CKEditor con el fin de obtener una licencia gratuita. Sin embargo, después de varios intentos y comunicaciones con el equipo, no pudimos obtenerla, por lo que tuvimos que buscar otra librería para implementar el editor de texto.

Una de las librerías que decidimos probar fue Quill.js. En nuestra aplicaciones usamos modales para cada funcionalidad, entonces hice la prueba de implementar modales utilizando esta librería y utilizando también Redux. Sin embargo decidimos no usar Redux ya que encontramos una forma más sencilla de implementar los modales en Quill. Luego decidimos no usar Quill.js ya que encontramos otra librería llamada Slate.js que se adapta mejor a nuestras necesidades.

Una vez que decidimos que librería utilizar, empezamos a implementar las funcionalidades. La primera funcionalidad que implementé fue la de ejercicios de desarrollo. Para implementar esta funcionalidad tuve que aprender a utilizar Slate para poder insertar el ejercicio en nuestro editor de texto.

Los estilos de la funcionalidad anterior los había realizado con CSS. Sin embargo, decidimos usar Tailwind, el cual es un framework de CSS. Debido a esto, tuve que aprender a usar este framework y refactorizar el código para implementar los estilos con Tailwind en vez de CSS.

Una vez terminada mi funcionalidad, realicé el testing de la funcionalidad de ejercicios de definiciones para garantizar su correcto funcionamiento. Un compañero también hizo el testing de la funcionalidad que yo había implementado y encontró algunos errores que tuve que arreglar. Por ejemplo, se podía insertar el ejercicio en el editor sin enunciado, el modal no se reseteaba al cerrarlo, etc.

Después de esto trabajé en implementar la funcionalidad de crear ejercicios de matemática con huecos. Una vez terminada, realicé el testing de la funcionalidad de leyenda de colores. Al igual que antes, un compañero realizó el testing de la funcionalidad de crear ejercicios de matemática y encontró una serie de errores que tuve que arreglar.

Por último, también empecé la implementación de la funcionalidad de espacios para dibujar.

Con respecto a la memoria, me encargué de redactar la sección \ref{cap:adaptaMaterial} que habla sobre la primera versión de AdaptaMaterialEscolar. También escribí la sección en “Herramientas empleadas” que explica el uso de MaterialUI en nuestro proyecto. Sin embargo, hemos decidido dejar de utilizar esta librería, por lo cual esta sección ya no existe. Además expliqué la implementación de las funcionalidades de ejercicios de desarrollo y de ejercicios de matemática con huecos en las secciones \ref{sec:impdesarrollo} y \ref{sec:impmatematica}.

Realicé el diseño de la página principal para compararlo con los de mis compañeros. También hice los diseños individuales de cada funcionalidad en Figma para la iteración competitiva. Después de esto, redacté la explicación de cada uno de estos diseños en la sección \ref{sec:albertoIter} de la memoria.


\section{Johan Sebastian Salvatierra Gutierrez}
En el desarrollo de nuestro TFG en grupo, he tenido una participación activa en la creación del código y en la redacción de la memoria. En la sección de código he trabajado tanto individualmente como en grupo para lograr los objetivos propuestos.

En cuanto al trabajo individual, comencé investigando para encontrar un editor de texto adecuado para nuestras necesidades. Después de evaluar varias opciones, propuse el uso de Slate, un framework altamente personalizable que resultó ser una excelente elección. Para probar su viabilidad, creé un prototipo utilizando Slate y también trabajé en la función para crear ejercicios de definiciones, así como en la inserción de tablas y la función para crear ejercicios de relacionar conceptos.

También realicé pruebas y operaciones de testing en la función para crear ejercicios de desarrollo y, después de investigar y experimentar, logré implementar la funcionalidad adecuada para que el comportamiento de las imágenes de los ejercicios fuera el deseado. Asimismo, me encargué de investigar e implementar una forma para que la modificación de los ejercicios fuera más dinámica e intuitiva para los usuarios. Finalmente, realicé pruebas para validar todos los cambios realizados.

En el trabajo en grupo, participé en la creación de la función para buscar pictogramas, así como en el diseño del logo. También ayudé a Álvaro con la abstracción del código.

En la sección de memoria, he realizado una contribución importante tanto en el trabajo individual como en el trabajo en grupo. En particular, en mi trabajo individual, he escrito la Sección \ref{cap:motivacio}, en la cual se explica la motivación detrás de nuestro TFG, y la Sección \ref{claseDeServicio}, donde se aclara las clases de servicio que empleamos para el tablero Kanban. Además, realicé una propuesta de diseño en la sección de herramientas, donde se incluyen subsecciones sobre Figma y Slate.

En cuanto al trabajo en grupo, he colaborado junto a Álvaro en la creación de los objetivos y en el diseño de las funcionalidades. También he participado en la elaboración de la metodología empleada, los requisitos y la arquitectura del proyecto. En resumen, mi aportación ha sido significativa tanto en la creación del código como en la elaboración de la memoria.