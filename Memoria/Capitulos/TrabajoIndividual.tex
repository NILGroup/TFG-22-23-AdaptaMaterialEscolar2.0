\chapter{Trabajo Individual}
\label{cap:TrabajoIndividual}

En este capítulo se habla del trabajo que ha realizado cada miembro del equipo en el proyecto.

\section{Álvaro Gómez Sittima}
Durante el desarrollo de este TFG, he colaborado en la creación del código y la redacción de la memoria. Me he involucrado tanto de manera individual como en grupo para alcanzar los objetivos establecidos.

En cuanto al código, he realizado diversas tareas de forma individual, tales como investigar alternativas a CKEditor, proponiendo Quill.js, aunque finalmente nos decantamos por la opción que propuso Johan (Slate) ya que se integraba mejor con React. También he propuesto e integrado Tailwind CSS como framework CSS para la aplicación. Además, he realizado la refactorización y limpieza del código para facilitar la escalabilidad y la implementación de nuevas funcionalidades en el futuro. Asimismo, he creado distintos componentes reutilizables para facilitar la creación de nuevas ventanas modales, como por ejemplo el propio componente de la ventana modal, botones o la vista previa. En cuanto a las funcionalidades, me he encargado de desarrollar y arreglar los bugs de las funcionalidades de crear ejercicios de completar huecos, sopa de letras y generar resúmenes. En cuanto al editor, he implementado algunas de las funcionalidades básicas, como cambiar el tipo de fuente, el tamaño de letra, el color de la fuente, resaltar el texto con un color de fondo y distintos formatos básicos de fuente (subrayado, tachado y cursiva). También he realizado las pruebas de las funcionalidades de verdadero/falso y ejercicios de matemáticas con huecos. Además, he integrado herramientas destinadas a mejorar la calidad del código, tales como Prettier, con el fin de definir un estilo común en todo el proyecto, y ESLINT, para llevar a cabo análisis estáticos del código. En colaboración con el grupo, hemos desarrollado la estructura principal de la aplicación, incluyendo la página de inicio, la barra de navegación y el editor. También hemos desarrollado la función de buscar pictograma.

Con respecto a la memoria, he contribuido individualmente en la redacción de diferentes secciones de la memoria, tales como la Sección \ref{sec:herramientasexistentes}, en la cual se realiza una comparativa con otras aplicaciones similares a AdaptaMaterialEscolar, la Sección \ref{sec:pruebas} donde se explica el plan de pruebas seguido en el proyecto y la Sección \ref{sec:tailwind}, en la cual se explica el framework Tailwind CSS. Además, he redactado la implementación de las funcionalidades que he desarrollado: completar huecos (Sección \ref{sec:impcompletarhuecos}), sopa de letras (Sección \ref{sec:impsopaletras}) y generar resumen (Sección \ref{sec:impresumen}). También, he realizado diseños para la iteración competitiva, Sección \ref{sec:iterAlvaro}. Junto a mis compañeros, he participado en la definición de los objetivos (Sección \ref{cap:objetivos}), la definición de requisitos (Sección \ref{cap:requisitos}), el diseño final de la aplicación (Sección \ref{subsec:DisenyoFinal}), la visualización del flujo Kanban (Sección \ref{sec:flujoTrabajo}), las políticas explicitas Kanban (Sección \ref{sec:politicas}), la arquitectura de la aplicación (Sección \ref{sub:Arquitectura}), la implementación de la funcionalidad de buscar pictogramas (Sección \ref{sec:impbuscarpicto}) y el diseño del logo de ADME. En cuanto al diseño de las funcionalidades, he colaborado con Johan en la realización de los diseños en Figma.

\section{Dunia Namour Doughani}
A lo largo de este TFG he desempeñado varias tareas tanto individualmente como con los interantes de este equipo para lograr cada objetivo propuesto. En cuanto a las actividades, pueden ser clasificadas en varias categorías: estudio de la cuestión, captura de requisitos, diseño de la aplicación, implementación, evaluación, metodología, QA y memoria.
\begin{itemize}
    \item  Estudio de la cuestión: En esta categoría se presenta la investigación realizada sobre la adaptación curricular, que es el tema principal abordado. Individualmente redacté la Sección \ref{cap:adaptacion}, la cual proporciona una explicación sobre la adaptación curricular y sus diferentes tipos.
    \item Captura de requisitos: En esta categoría se presenta las tareas relacionadas con el proceso de identificar y documentar las necesidades de los usuarios. En la Sección \ref{cap:requisitos} definimos y clasificamos todos juntos los requisitos. 
    \item Diseño de la aplicación: En esta categoría se presenta las tareas relacionadas con el proceso de crear y definir la estructura, apariencia y funcionalidad de una aplicación. La primera tarea que realicé fue la creación de los diseño para la iteración competitiva, los cuales se describen en la Sección \ref{sec:duniaIter}. También creé en Figma cómo se verían los ejercicios en el documento de trabajo. Como equipo diseñamos el logo de la aplicación en Figma.
    \item Implementación: En esta categoría se presenta las tareas relacionadas con la construcción y puesta en marcha de la aplicación. Individualmente implementé la funcionalidad de verdadero y falso \ref{sec:funcioVF} y de leyenda de colores \ref{sec:leyendaColores}. También llevé a cabo una refactorización de la funcionalidad de verdadero y falso para sustituir el CSS por la librería de diseño Tailwind. En cuento al trabajo en equipo, aporté en el desarrollo de la funcionalidad de búsqueda de pictogramas y en la pantalla de inicio.
    \item Evaluación: En esta categoría se presenta las tareas relacionadas con el proceso de medir y analizar el desempeño de la aplicación. Realicé la plantilla de examen, los apuntes y el formulario de evaluacion para que los usuarios finales puedan valorar AdaptaMaterialEscolar 2.0.
    \item Metodología: En esta categoría se expone las tareas relacionadas con el conjunto de prácticas y técnicas utilizadas para planificar y llevar a cabo un proyecto. Colaboré en la redacción de la sección del tablero Kanban en la Sección \ref{sec:flujoTrabajo} y en la definición de las políticas explícitas en la Sección \ref{sec:politicas}.
    \item QA (Quality Assurance): En esta categoría se expone las tareas relacionadas con el conjunto de procesos, técnicas y actividades enfocadas en garantizar la calidad de la aplicación. Para ello realicé de forma individual la corrección de los errores relacionados con la funcionalidad de leyenda de colores y la funcionalidad de verdadero y falso. También dediqué tiempo a probar y desarrollar un plan de pruebas detallado para las funcionalidades de sopa de letras y buscar pictograma. Además, llevé a cabo el test de múltiples funcionalidades, incluyendo las de relacionar conceptos, pictotraductor, espacios para dibujar y de exportar a PDF. Finalmente, realicé un segundo test en las funcionalidades de desarrollo y definiciones, resumen y huecos de matemáticas. A cerca del trabajo en equipo, colaboré en la redacción del plan de pruebas para la funcionalidad de verdadero y falso y brindé apoyo a Álvaro en la refactorización del código.
    \item Memoria: En esta categoría se describen las tareas implicadas en la redacción de los diferentes capítulos del TFG. Individualmente me encargué de redactar la Sección \ref{cap:estructura}, donde se hace una breve descripción de cada cápitulo escrito en dicha memoria. También expliqué la herramienta React en el capítulo sobre herramientas empleadas, la Sección \ref{sec:React}. Además, incluí las imágenes de las funcionalidades diseñadas en Figma. Por último, incluí el plan de pruebas, escrito en un documento aparte, en la memoria en el Anexo \ref{ape:pruebas}. En relación al trabajo participé en la redacción de los objetivos del proyecto en la Sección \ref{cap:objetivos}. También desarrollamos el diseño final de cada funcionalidad en la Sección \ref{subsec:DisenyoFinal} y la arquitectura en la Sección \ref{sub:Arquitectura}. Asimismo, redactamos la implementación de la funcionalidad de búsqueda de pictograma en la Sección \ref{sec:impbuscarpicto} y el capítulo de introducción en inglés en el Capítulo \ref{cap:introduction}

\end{itemize}


\section{Alberto Alejandro Rivas Fernandez}
Al principio del proyecto, uno de los desafíos que enfrentamos fue la necesidad de implementar la funcionalidad de exportar documentos en formato Word y PDF. CKEditor tiene un plugin que nos permite hacer esto, pero necesitábamos una licencia de pago para poder utilizarlo. Me puse en contacto con el equipo de CKEditor con el fin de obtener una licencia gratuita. Sin embargo, después de varios intentos y comunicaciones con el equipo, no pudimos obtenerla, por lo que tuvimos que buscar otra librería para implementar el editor de texto. 

Una de las librerías que decidimos probar fue Quill.js. En nuestra aplicaciones usamos modales para cada funcionalidad, entonces hice la prueba de implementar modales utilizando esta librería y utilizando también Redux. Sin embargo decidimos no usar Redux ya que encontramos una forma más sencilla de implementar los modales en Quill. Luego decidimos no usar Quill.js ya que encontramos otra librería llamada Slate.js que se adapta mejor a nuestras necesidades.

Una vez que decidimos que librería utilizar, empezamos a implementar las funcionalidades. La primera funcionalidad que implementé fue la de ejercicios de desarrollo. Para implementar esta funcionalidad tuve que aprender a utilizar Slate para poder insertar el ejercicio en nuestro editor de texto.

Los estilos de la funcionalidad anterior los había realizado con CSS. Sin embargo, decidimos usar Tailwind, el cual es un framework de CSS. Debido a esto, tuve que aprender a usar este framework y refactorizar el código para implementar los estilos con Tailwind en vez de CSS.

Una vez terminada mi funcionalidad, realicé el testing de la funcionalidad de ejercicios de definiciones para garantizar su correcto funcionamiento. Un compañero también hizo el testing de la funcionalidad que yo había implementado y encontró algunos errores que tuve que arreglar. Por ejemplo, se podía insertar el ejercicio en el editor sin enunciado, el modal no se reseteaba al cerrarlo, etc.

Después de esto trabajé en implementar la funcionalidad de crear ejercicios de matemática con huecos. Una vez terminada, realicé el testing de la funcionalidad de leyenda de colores. Al igual que antes, un compañero realizó el testing de la funcionalidad de crear ejercicios de matemática y encontró una serie de errores que tuve que arreglar. Además de eso, se nos ocurrió añadir funciones nuevas, como poder agregar varias fórmulas al mismo tiempo, por ejemplo, por lo que también realicé esos cambios.

Por último, también realicé la implementación de la funcionalidad de espacios para dibujar.

En cuanto al trabajo en grupo en el código, desarrollamos la funcionalidad de buscar pictograma en conjunto para luego usarla como referencia al implementar el resto de funcionalidades individualmente, realizamos el plan de pruebas de la funcionalidad de ejercicios de verdadero y falso, y también ayudamos a Álvaro con la refactorización del código, con el objetivo de hacerlo más reutilizable.

Con respecto a la memoria, me encargué de redactar la sección \ref{cap:adaptaMaterial} que habla sobre la primera versión de AdaptaMaterialEscolar. También escribí la sección en “Herramientas empleadas” que explica el uso de MaterialUI en nuestro proyecto. Sin embargo, hemos decidido dejar de utilizar esta librería, por lo cual esta sección ya no existe. Además expliqué la implementación de las funcionalidades de ejercicios de desarrollo y de ejercicios de matemática con huecos en las secciones \ref{sec:impdesarrollo} y \ref{sec:impmatematica}.

Realicé el diseño de la página principal para compararlo con los de mis compañeros. También hice los diseños individuales de cada funcionalidad en Figma para la iteración competitiva. Después de esto, redacté la explicación de cada uno de estos diseños en la sección \ref{sec:albertoIter} de la memoria.

En cuanto al trabajo en grupo en la memoria, hemos redactado la Sección \ref{cap:objetivos}, en la que explicamos los objetivos de este TFG. También hemos traducido a inglés todo el capítulo de introducción (Capítulo \ref{cap:introduction}). En el capítulo de metodología (Capítulo \ref{cap:metodologia}), redactamos juntos la sección de visualizar el flujo de trabajo (Sección \ref{sec:flujoTrabajo}), en la que explicamos cómo usaremos el tablero Kanban en nuestro proyecto y también escribimos la sección de políticas explícitas (Sección \ref{sec:politicas}), en la que enumeramos las reglas que debemos seguir al usar este tablero. En el capítulo de AdaptaMaterialEscolar 2.0 (Capítulo \ref{cap:AdaptaMaterialEscolar2.0}), redactamos la sección en la que explicamos los requisitos de la aplicación (Sección \ref{cap:requisitos}), el diseño final (Sección \ref{subsec:DisenyoFinal}) y la arquitectura (Sección \ref{sub:Arquitectura}), en la sección de funcionalidades, en la que explicamos la implementación de cada una, escribimos la implementación de buscar pictograma (Sección \ref{sec:impbuscarpicto}) en grupo para después hacer las demás individualmente. Por último, diseñamos el logo de la aplicación entre todos.


\section{Johan Sebastian Salvatierra Gutierrez}
En el desarrollo de nuestro TFG en grupo, mi participación fue tanto en la creación del código y en la redacción de la memoria. En la sección de código he trabajado tanto individualmente como en grupo para lograr los objetivos propuestos.

En cuanto al trabajo individual que realicé, tuve un enfoque sistemático para encontrar soluciones adecuadas para nuestras necesidades. En primer lugar, realicé una investigación exhaustiva para encontrar un editor de texto que pudiera satisfacer los requisitos del proyecto. Tras evaluar varias opciones, me encontré con Slate, un framework altamente personalizable que parecía ser la mejor opción para nuestro proyecto. Una vez que identifiqué a Slate como el framework adecuado, me dispuse a comprobar su viabilidad creando un prototipo con él. Para ello, integré Slate con React y realicé pruebas para comprobar que podríamos realizar todas las operaciones que necesitábamos. Una vez comprobé su viabilidad, propuse su uso para el proyecto. A continuación, me dediqué a investigar la documentación de Slate con el objetivo de emplear su estructura de la forma más actualizada y correcta posible. Trabajé en la función para crear ejercicios de definiciones, cree pautas en formato SVG que pudieran reutilizarse en diferentes ejercicios. También creé el nodo Tablas, con todas las funciones necesarias para gestionarlos y modifiqué su formato para que otras funciones pudieran utilizarlo. Además, realicé la función para crear ejercicios de relacionar conceptos y cree la forma para que los ejercicios tuvieran una edición dinámica a través del modal que los creo. Esto requirió mucha investigación, ya que no encontré nada similar. Abstraje los diferentes nodos de los ejercicios en uno solo que permitiera la edición independientemente del ejercicio, y también creé la numeración automática de los ejercicios a partir de este nodo. Además, me encargué de desarrollar parte de la funcionalidad básica del editor, como formas de listar y tipos de alineación. Investigué la forma de que las imágenes e iconos se comportaran de la forma deseada y realicé pruebas para las funciones de creación de ejercicios de definiciones, resúmenes y huecos. Por último, también dediqué tiempo a investigar la forma en que todos los elementos pudieran exportarse a PDF con el menor impacto posible.

Durante el trabajo en grupo, contribuí en varias áreas importantes del proyecto. En particular, colaboré en la creación de la función de búsqueda de pictogramas, lo que permitió a los usuarios encontrar imágenes relevantes de manera más eficiente. Además, participé en el diseño del logo, asegurándome de que reflejara adecuadamente la esencia del proyecto. Además, colaboré con Álvaro en la abstracción del código, lo que nos permitió desarrollar una base de código más modular y fácil de mantener. Trabajamos juntos para identificar las áreas de código que podían ser simplificadas y refactorizadas, lo que mejoró significativamente la calidad del código. Finalmente, en la fase de pruebas, contribuí en la creación del plan de pruebas de verdadero o falso. Para tener una buena base comparativa para las demás pruebas lo que nos permitió identificar y solucionar rápidamente cualquier problema que surgiera en la aplicación.

En la sección de memoria, mi contribución individual incluyó la escritura de las secciones: Sección \ref{cap:motivacio} y la Sección \ref{claseDeServicio}. En la Sección \ref{cap:motivacio}, expliqué detalladamente la motivación detrás de nuestro TFG, lo que ayudó a establecer un contexto sólido para el lector y a comprender la importancia del proyecto. En la Sección \ref{claseDeServicio}, aclaré las clases de servicio que empleamos para el tablero Kanban, lo que fue fundamental para el desarrollo del proyecto y la gestión eficiente de los recursos. También realicé una propuesta de diseño de la aplicación. Añadí un anexo explicando el diseño que propuse. Realice en la sección de herramientas una explicación de Slate. Explique el editor empleado en la Sección \ref{Editor} donde explique los tipos de nodos y su integración con la aplicación.

En cuanto al trabajo en grupo, he participado activamente en varias tareas importantes. En primer lugar, ayudé a crear los objetivos del proyecto y en el diseño de las funcionalidades. También contribuí a la introducción en inglés, que se requería para llegar a una audiencia más amplia. Además, trabajé junto con el equipo en la elaboración de la metodología del proyecto. Esto incluyó la visualización del flujo de trabajo y la definición de las políticas explícitas, los requisitos y la arquitectura del proyecto. Otra tarea importante en la que participé fue la implementación de la función de búsqueda de pictogramas. Para ello, realicé una investigación exhaustiva para encontrar una solución efectiva y eficiente. Después de evaluar varias opciones, desarrollé una implementación que cumplía con todos los requisitos y especificaciones del proyecto.
