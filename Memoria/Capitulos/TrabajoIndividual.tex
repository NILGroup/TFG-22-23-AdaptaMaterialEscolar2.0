\chapter{Trabajo Individual}
\label{cap:TrabajoIndividual}

En este capítulo se habla del trabajo que ha realizado cada miembro del equipo en el proyecto.

\section{Dunia Namour Doughani}
Con respecto a la memoria he realizado en {\hyperref[cap:estadoDelArte]{Capítulo tres}} la Sección \ref{cap:adaptacion} en la cual se explica la adaptación curricular y sus tipos. También he buscado información sobre las funcionalidades para poder realizar una adecuada prorización. Además, he redactado la estructura del proyecto. Junto a mis compañeros he realizado los objetivos, la parte relacionada con la metodología y el diseño de la aplicación.

\section{Álvaro Gómez Sittima}
Lo primero que hice al empezar el TFG fue consultar la memoria y el código de la versión anterior de la aplicación AdaptaMaterialEscolar. De esta manera podría acudir a la primera reunión con las tutoras del TFG con una idea general del objetivo del proyecto y que funcionalidades había que añadir o mejorar. Esta investigación inicial también me permitió entender la arquitectura y las herramientas de la versión anterior, en caso de que sigamos utilizando estas en nuestra versión o considerar alguna alternativa.

Después de la primera reunión, comencé a investigar sobre algunos problemas que surgieron con CKEditor (una de las herramientas utilizadas en la versión anterior) y alternativas a este. También investigue sobre la adaptación curricular para describir la motivación y el objetivo de esta nueva versión de AdaptaMaterialEscolar, y sobre otras herramientas similares que traten de facilitar la adaptación curricular.

Para la tercera reunión, empecé a investigar herramientas de generación de resúmenes que pudieramos utilizar. También tuve que investigar sobre el intercambio de Recursos de Origen Cruzado (CORS), el cual es un mecanismo de seguridad que restringe el acceso a recursos en un origen distinto (por ejemplo una API externa) al de la petición.

\section{Johan Sebastian Salvatierra Gutierrez}
Con respecto a la memoria he realizado el {\hyperref[cap:motivacio]{Capítulo uno}} Seccion \ref{cap:motivacio} en la cual se explica la motivación de nuestro TFG y el {\hyperref[claseDeServicio]{Capítulo 4}} Sección \ref{claseDeServicio} donde se aclara las clases de servicio que empleamos para el tablero Kanban. 
Además junto a mis compañeros he realizado los objetivos, el diseño y la metodología empleada. También he realizado una investigación para la lista de priorización.