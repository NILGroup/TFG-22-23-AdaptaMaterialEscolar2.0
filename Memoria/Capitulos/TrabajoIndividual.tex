\chapter{Trabajo Individual}
\label{cap:TrabajoIndividual}

En este capítulo se habla del trabajo que ha realizado cada miembro del equipo en el proyecto.

\section{Álvaro Gómez Sittima}
Con respecto a la memoria he realizado la Sección \ref{cap:herramientas} en la cual se comparan algunas aplicaciones similares a AdaptaMaterialEscolar. También, he redactado la Sección \ref{cap:pruebas} donde se explica el plan de pruebas que ha seguido este proyecto. Además, junto a mis compañeros, he realizado los objetivos, el diseño y la metodología empleada.

\section{Dunia Namour Doughani}
Con respecto a la memoria he realizado la Sección \ref{cap:adaptacion} en la cual se explica la adaptación curricular y sus tipos. Además, he redactado la estructura del proyecto. Junto a mis compañeros he realizado los objetivos, la parte relacionada con la metodología y el diseño de la aplicación.

\section{Alberto Alejandro Rivas Fernandez}
\begin{itemize}
    \item Código
        \begin{itemize}
            \item Individual
                \begin{itemize}
                    \item Investigar cómo añadir un plugin a CKEditor para poder exportar el documento en formato Word
                    \item Contactar con CKEditor con el fin de obtener una licencia gratuita, para poder utilizar las funciones de exportar en formato Word y PDF. No fue posible obtener la licencia por lo que tuvimos que buscar otra librería para implementar el editor de texto
                    \item Intentar implementar un modal en React usando Redux y Quill.js
                    \item Implementar funcionalidad para crear ejercicios de desarrollo usando Slate
                    \item Refactorizar la funcionalidad de crear ejercicios de desarrollo para usar Tailwind en vez de CSS
                    \item Realizar testing en la funcionalidad de crear ejercicios de definición
                    \item Arreglar errores encontrados en mi funcionalidad durante el testing realizado por mi compañero Johan
                    \item Empezar a implementar la funcionalidad de crear ejercicios de matemática con huecos
                \end{itemize}
        \end{itemize}
    \item Memoria
    \begin{itemize}
        \item Individual
            \begin{itemize}
                \item He realizado la Sección \ref{cap:adaptaMaterial} que habla sobre la primera versión de AdaptaMaterialEscolar
                \item Diseño de la página principal para comprarlo con los de mis compañeros
                \item Hacer diseños de cada funcionalidad en Figma para compararlos con los de mis compañeros
                \item Escribir la explicación de mis diseños en la memoria
                \item Escribir sección en “Herramientas Empleadas” explicando el uso de MaterialUI
            \end{itemize}
        \item En grupo
            \begin{itemize}
                \item Junto a mis compañeros hemos explicado la metodología y priorizado las funcionalidades
            \end{itemize}
    \end{itemize}
\end{itemize}

\section{Johan Sebastian Salvatierra Gutierrez}
\begin{itemize}
    \item Código
        \begin{itemize}
            \item Individual
                \begin{itemize}
                    \item Investigue para obtener un editor que se ajuste a nuestras necesidades
                    \item Propuse Slate el cual es un framework de editor de texto altamente personalizable 
                    \item Cree un prototipo con Slate para poder combrobar si era una opción adecuada
                    \item Función para crear ejercicio de definiciones
                    \item Funcion para crear ejercicios de relacionar conceptos
                    \item Realice operaciones de testing en la función de crear ejercicios de desarrollo
                \end{itemize}
            \item En grupo
                \begin{itemize}
                    \item Desarrollamos la función buscar pictograma
                    \item Diseñamos un logo
                    \item Ayude a Alvaro con la abstracción del código
                \end{itemize}
        \end{itemize}
    \item Memoria
    \begin{itemize}
        \item Individual
            \begin{itemize}
                \item La Seccion \ref{cap:motivacio} en la cual se explica la motivación de nuestro TFG
                \item La Sección \ref{claseDeServicio} donde se aclara las clases de servicio que empleamos para el tablero Kanban.
                \item Cree una propuesta de diseño
                \item En el capitulo de las herramientas la subsección de figma y slate.
            \end{itemize}
        \item En grupo
            \begin{itemize}
                \item Los objetivos
                \item Junto a Alvaro hemos realizado el diseño de las funcionalidades.
                \item La metodología empleada.
                \item Requisitos
            \end{itemize}
    \end{itemize}
\end{itemize}