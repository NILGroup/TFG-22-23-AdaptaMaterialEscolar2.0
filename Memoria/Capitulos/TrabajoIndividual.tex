\chapter{Trabajo Individual}
\label{cap:TrabajoIndividual}

En este capítulo se habla del trabajo que ha realizado cada miembro del equipo en el proyecto.

\section{Álvaro Gómez Sittima}
Con respecto a la memoria he realizado la Sección \ref{cap:herramientas} en la cual se comparan algunas aplicaciones similares a AdaptaMaterialEscolar. También, he redactado la Sección \ref{cap:pruebas} donde se explica el plan de pruebas que ha seguido este proyecto. Además, junto a mis compañeros, he realizado los objetivos, el diseño y la metodología empleada.
\begin{itemize}
    \item Código
          \begin{itemize}
              \item Individual
                    \begin{itemize}
                        \item Investigue para encontrar una alternativa a CKEditor y propuse Quill.js, aunque finalmente nos decantamos por la opción que propuso Johan, ya que se ajustaba más a nuestras necesidades.
                        \item Investigue para encontrar un framework CSS que nos facilitará dar estilo a la aplicación y propuse Tailwind CSS.
                        \item Refactorización y limpieza del código, para que escalar/añadir nuevas funcionalidades sea lo más sencillo posible.
                        \item Función para crear ejercicios de completar huecos.
                        \item Función para crear ejercicios de sopa de letras.
                        \item Función para generar resúmenes.
                        \item Testing de la función de Verdadero/Falso.
                        \item He añadido Prettier para formatear el código y ESLINT para analizar estáticamente el código y encontra problemas. Ambos se pueden aplicar facilmente en el pipeline de integración continua.
                    \end{itemize}
              \item En grupo
                    \begin{itemize}
                        \item Hemos desarrollado la estructura principal de la aplicación (página de inicio, barra de navegación y editor).
                        \item Hemos desarrollado la función de buscar pictograma.
                    \end{itemize}
          \end{itemize}
    \item Memoria
          \begin{itemize}
              \item Individual
                    \begin{itemize}
                        \item La Sección \ref{sec:tailwind}, en la cual se explica el framework Tailwind CSS.
                        % \item TODO: Aclarar que hacer con esta seccion La Sección \ref{sec:arasaac}, en la cual se explica la API de ARASAAC que utilizamos para obtener los pictogramas.
                        \item Redactar los diseños que hice para la iteración competitiva
                        \item Propuestas de diseño para el rediseño de la aplicación.
                    \end{itemize}
              \item En grupo
                    \begin{itemize}
                        \item Junto a Johan, hemos realizado el diseño de las funcionalidades en Figma.
                        \item He ayudado a redactar el diseño final de las funcionalidades.
                        \item He ayudado a redactar los requisitos y la arquitectura con mis compañeros.
                    \end{itemize}
          \end{itemize}
\end{itemize}

\section{Dunia Namour Doughani}
Con respecto a la memoria he realizado la Sección \ref{cap:adaptacion} en la cual se explica la adaptación curricular y sus tipos. Además, he redactado la estructura del proyecto. Junto a mis compañeros he realizado los objetivos, la parte relacionada con la metodología de desarollo.
\begin{itemize}
    \item Código
          \begin{itemize}
              \item Individual
                    \begin{itemize}
                        \item implementar modal de la funcionalidad de V/f
                        \item He testado la funcionalidad de sopa de letras y escribirlo en el plan de pruebas
                        \item Test de buscar pictograma y escribirlo en el plan de pruebas
                        \item Escribir con mis compañeros en el plan de pruebas la funcionalidad de V/F
                        \item Correccion de los bugs de V/F
                        \item Refactorizar la funcionalidad de crear V/F para usar Tailwind en vez de CSS
                        \item Funcionalidad de leyeda
                        \item Ayudar a la refactorización del codigo
                    \end{itemize}
          \end{itemize}
    \item Memoria
          \begin{itemize}
              \item Individual
                    \begin{itemize}
                        \item Diseño en figma de como se ven los ejercicios en el editable
                        \item Sacar capturas de las fucnionalidades para insertarlas en la memoria
                        \item Redactar los diseños que hice para la iteración competitiva
                        \item Escribir React en la seccion de herramientas empeadas

                    \end{itemize}
              \item En grupo
                    \begin{itemize}
                        \item Redactar diseño final de las funcionalidades
                        \item Redactar requisitos y arquetctura con mis compañeros



                    \end{itemize}
          \end{itemize}
\end{itemize}

\section{Alberto Alejandro Rivas Fernandez}
\begin{itemize}
    \item Código
          \begin{itemize}
              \item Individual
                    \begin{itemize}
                        \item Investigar cómo añadir un plugin a CKEditor para poder exportar el documento en formato Word
                        \item Contactar con CKEditor con el fin de obtener una licencia gratuita, para poder utilizar las funciones de exportar en formato Word y PDF. No fue posible obtener la licencia por lo que tuvimos que buscar otra librería para implementar el editor de texto
                        \item Intentar implementar un modal en React usando Redux y Quill.js
                        \item Implementar funcionalidad para crear ejercicios de desarrollo usando Slate
                        \item Refactorizar la funcionalidad de crear ejercicios de desarrollo para usar Tailwind en vez de CSS
                        \item Realizar testing en la funcionalidad de crear ejercicios de definición
                        \item Arreglar errores encontrados en mi funcionalidad durante el testing realizado por mi compañero Johan
                        \item Empezar a implementar la funcionalidad de crear ejercicios de matemática con huecos
                    \end{itemize}
          \end{itemize}
    \item Memoria
          \begin{itemize}
              \item Individual
                    \begin{itemize}
                        \item He realizado la Sección \ref{cap:adaptaMaterial} que habla sobre la primera versión de AdaptaMaterialEscolar
                        \item Diseño de la página principal para comprarlo con los de mis compañeros
                        \item Hacer diseños de cada funcionalidad en Figma para compararlos con los de mis compañeros
                        \item Escribir la explicación de mis diseños en la memoria
                        \item Escribir sección en “Herramientas Empleadas” explicando el uso de MaterialUI
                    \end{itemize}
              \item En grupo
                    \begin{itemize}
                        \item Junto a mis compañeros hemos explicado la metodología y priorizado las funcionalidades
                    \end{itemize}
          \end{itemize}
\end{itemize}

\section{Johan Sebastian Salvatierra Gutierrez}
\begin{itemize}
    \item Código
          \begin{itemize}
              \item Individual
                    \begin{itemize}
                        \item Investigue para obtener un editor que se ajuste a nuestras necesidades
                        \item Propuse Slate el cual es un framework de editor de texto altamente personalizable
                        \item Cree un prototipo con Slate para poder combrobar si era una opción adecuada
                        \item Función para crear ejercicio de definiciones
                        \item Funcion para crear ejercicios de relacionar conceptos
                        \item Realice operaciones de testing en la función de crear ejercicios de desarrollo
                    \end{itemize}
              \item En grupo
                    \begin{itemize}
                        \item Desarrollamos la función buscar pictograma
                        \item Diseñamos un logo
                        \item Ayude a Alvaro con la abstracción del código
                    \end{itemize}
          \end{itemize}
    \item Memoria
          \begin{itemize}
              \item Individual
                    \begin{itemize}
                        \item La Seccion \ref{cap:motivacio} en la cual se explica la motivación de nuestro TFG
                        \item La Sección \ref{claseDeServicio} donde se aclara las clases de servicio que empleamos para el tablero Kanban.
                        \item Cree una propuesta de diseño
                        \item En el capitulo de las herramientas la subsección de figma y slate.
                    \end{itemize}
              \item En grupo
                    \begin{itemize}
                        \item Los objetivos
                        \item Junto a Alvaro hemos realizado el diseño de las funcionalidades.
                        \item La metodología empleada.
                        \item Requisitos
                    \end{itemize}
          \end{itemize}
\end{itemize}