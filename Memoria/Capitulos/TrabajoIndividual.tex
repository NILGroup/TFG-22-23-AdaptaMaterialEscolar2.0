\chapter{Trabajo Individual}
\label{cap:TrabajoIndividual}

En este capítulo se habla del trabajo que ha realizado cada miembro del equipo en el proyecto.

\section{Álvaro Gómez Sittima}
Con respecto a la memoria he realizado la Sección \ref{cap:herramientas} en la cual se comparan algunas aplicaciones similares a AdaptaMaterialEscolar. También, he redactado la Sección \ref{cap:pruebas} donde se explica el plan de pruebas que ha seguido este proyecto. Además, junto a mis compañeros, he realizado los objetivos, el diseño y la metodología empleada. También he realizado una investigación para la lista de priorización.

\section{Dunia Namour Doughani}
Con respecto a la memoria he realizado en {\hyperref[cap:estadoDelArte]{Capítulo tres}} la Sección \ref{cap:adaptacion} en la cual se explica la adaptación curricular y sus tipos. También he buscado información sobre las funcionalidades para poder realizar una adecuada priorización. Además, he redactado la estructura del proyecto. Junto a mis compañeros he realizado los objetivos, la parte relacionada con la metodología y el diseño de la aplicación.

\section{Alberto Alejandro Rivas Fernandez}
En relación a la memoria he realizado en el {\hyperref[cap:estadoDelArte]{Capítulo tres}} la Sección \ref{cap:adaptaMaterial} en la cual hablaba sobre la primera versión de AdapatMaterialEscolar. También junto a mis compañeros hemos explicado la metodología y priorizado las funcionalidades, para ello investigué acerca de ellas, sobre todo la funcionalidad de pasar de pdf a word. Asimismo realicé un diseño de la página principal de la aplicación para compararlo con los diseños de mis compañeros. Además, le escribí un email al servicio al cliente de CKEditor para preguntarles cómo obtener una licencia que podamos usar para nuestro proyecto. Luego creé una cuenta gratis en CKEditor para poder obtener la licencia y también probé comentar un trozo de código para hacer que el CKEditor funcionase al ejecutar la aplicación.

\section{Johan Sebastian Salvatierra Gutierrez}
Con respecto a la memoria he realizado el {\hyperref[cap:motivacio]{Capítulo uno}} Seccion \ref{cap:motivacio} en la cual se explica la motivación de nuestro TFG y el {\hyperref[claseDeServicio]{Capítulo 4}} Sección \ref{claseDeServicio} donde se aclara las clases de servicio que empleamos para el tablero Kanban.
Además junto a mis compañeros he realizado los objetivos, el diseño y la metodología empleada. También he realizado una investigación para la lista de priorización.