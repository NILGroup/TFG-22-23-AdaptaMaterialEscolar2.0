\chapter{Trabajo Individual}
\label{cap:TrabajoIndividual}

En este capítulo se habla del trabajo que ha realizado cada miembro del equipo en el proyecto.

\section{Álvaro Gómez Sittima}
Durante el desarrollo de este TFG, he colaborado en la creación de la aplicación y la redacción de la memoria. A continuación se exponen las distintas actividades que he realizado a lo largo de este TFG, tanto de manera individual como en grupo, para alcanzar los objetivos establecidos. Las actividades se han clasificado en las siguientes categorías: estudio de la cuestión, captura de requisitos, diseño de la aplicación, implementación, evaluación, metodologías, QA y memoria.

En cuanto al estudio de la cuestión, investigué sobre las distintas herramientas existentes que ayudan a realizar adaptaciones curriculares no significativas. Esta investigación ayudó a clarificar el objetivo del TFG, además de aportar ideas para el diseño e implementación de ciertas funcionalidades similares a las herramientas estudiadas. También me informé sobre alternativas a CKEditor, proponiendo Quill.js, aunque finalmente nos decantamos por la opción que propuso Johan (Slate) ya que se integraba mejor con React. Otras herramientas y frameworks que investigué y propusé para utilizar en la implementación son: Tailwind CSS, Flowbite, Prettier y ESLint.

En cuanto a la captura de requisitos, en grupo se definieron las distintas agrupaciones de requisitos (formato, ejercicios y auxiliar) y se decidió que requisitos se implementarían y cúales no, ya fuese porque ya estaban implementados o por falta de información.

En cuanto al diseño de la aplicación, todos los miembros del equipo realizamos diseños individuales para la iteración competitiva. Estos diseños se utilizaron como boceto inicial y tras compararlos en una reunión con las tutoras del TFG, se definió el estilo y diseño general de la aplicación. Los diseños individuales los hicé en papel y se encuentran en el Apéndice \ref{ape:disenyoAlvaro}. Tras la reunión, colaboré con Johan en la realización de los diseños en Figma y, en grupo, diseñamos el logo de ADME.

En cuanto a la implementación, he integrado Tailwind CSS como framework CSS para la aplicación. Además, he realizado la refactorización y limpieza del código para facilitar la escalabilidad y la implementación de nuevas funcionalidades en el futuro. Asimismo, he creado distintos componentes reutilizables para facilitar la creación de nuevas ventanas modales, como por ejemplo el propio componente de la ventana modal, botones o la vista previa. En cuanto a las funcionalidades, me he encargado de desarrollar y arreglar los bugs de las funcionalidades de crear ejercicios de completar huecos, sopa de letras y generar resúmenes. En cuanto al editor, he implementado algunas de las funcionalidades básicas, como cambiar el tipo de fuente, el tamaño de letra, el color de la fuente, resaltar el texto con un color de fondo y distintos formatos básicos de fuente (subrayado, tachado y cursiva). También he realizado las pruebas de las funcionalidades de verdadero/falso y ejercicios de matemáticas con huecos. Además, he integrado herramientas destinadas a mejorar la calidad del código, tales como Prettier, con el fin de definir un estilo común en todo el proyecto, y ESLINT, para llevar a cabo análisis estáticos del código. En colaboración con el grupo, hemos desarrollado la estructura principal de la aplicación, incluyendo la página de inicio, la barra de navegación y el editor. También hemos desarrollado la función de buscar pictograma.

En cuanto al control de calidad (QA), durante este TFG he sido el responsable de proponer distintos planes de prueba, herramientas y técnicas para poder garantizar la calidad del código. Además, he redactado completamente la sección de QA, Sección \ref{sec:qa}.

Con respecto a la memoria, he contribuido individualmente en la redacción de diferentes secciones de la memoria, tales como la Sección \ref{sec:herramientasexistentes}, en la cual se realiza una comparativa con otras aplicaciones similares a AdaptaMaterialEscolar, la Sección \ref{sec:qapruebas} donde se explica el plan de pruebas seguido en el proyecto y la Sección \ref{sec:tailwind}, en la cual se explica el framework Tailwind CSS. Además, he redactado la implementación de las funcionalidades que he desarrollado: completar huecos (Sección \ref{sec:impcompletarhuecos}), sopa de letras (Sección \ref{sec:impsopaletras}) y generar resumen (Sección \ref{sec:impresumen}). También, he realizado diseños para la iteración competitiva, Sección \ref{sec:iterAlvaro}. Junto a mis compañeros, he participado en la definición de los objetivos (Sección \ref{cap:objetivos}), la definición de requisitos (Sección \ref{cap:requisitos}), el diseño final de la aplicación (Sección \ref{subsec:DisenyoFinal}), la visualización del flujo Kanban (Sección \ref{sec:flujoTrabajo}), las políticas explicitas Kanban (Sección \ref{sec:politicas}), la arquitectura de la aplicación (Sección \ref{sub:Arquitectura}), la implementación de la funcionalidad de buscar pictogramas (Sección \ref{sec:impbuscarpicto}) y el diseño del logo de ADME. En cuanto al diseño de las funcionalidades, he colaborado con Johan en la realización de los diseños en Figma.

\section{Dunia Namour Doughani}
A lo largo de este TFG he desempeñado varias tareas tanto individualmente como con los interantes de este equipo para lograr cada objetivo propuesto. En cuanto a las actividades, pueden ser clasificadas en varias categorías: estudio de la cuestión, captura de requisitos, diseño de la aplicación, implementación, evaluación, metodología, QA y memoria.
\begin{itemize}
    \item  Estudio de la cuestión: En esta categoría se presenta la investigación realizada sobre la adaptación curricular, que es el tema principal abordado. Individualmente redacté la Sección \ref{cap:adaptacion}, la cual proporciona una explicación sobre la adaptación curricular y sus diferentes tipos.
    \item Captura de requisitos: En esta categoría se presenta las tareas relacionadas con el proceso de identificar y documentar las necesidades de los usuarios. En la Sección \ref{cap:requisitos} definimos y clasificamos todos juntos los requisitos.
    \item Diseño de la aplicación: En esta categoría se presenta las tareas relacionadas con el proceso de crear y definir la estructura, apariencia y funcionalidad de una aplicación. La primera tarea que realicé fue la creación de los diseño para la iteración competitiva, los cuales se describen en la Sección \ref{sec:duniaIter}. También creé en Figma cómo se verían los ejercicios en el documento de trabajo. Como equipo diseñamos el logo de la aplicación en Figma.
    \item Implementación: En esta categoría se presenta las tareas relacionadas con la construcción y puesta en marcha de la aplicación. Individualmente implementé la funcionalidad de verdadero y falso \ref{sec:funcioVF} y de leyenda de colores \ref{sec:leyendaColores}. También llevé a cabo una refactorización de la funcionalidad de verdadero y falso para sustituir el CSS por la librería de diseño Tailwind. En cuento al trabajo en equipo, aporté en el desarrollo de la funcionalidad de búsqueda de pictogramas y en la pantalla de inicio.
    \item Evaluación: En esta categoría se presenta las tareas relacionadas con el proceso de medir y analizar el desempeño de la aplicación. Realicé la plantilla de examen, los apuntes y el formulario de evaluacion para que los usuarios finales puedan valorar AdaptaMaterialEscolar 2.0.
    \item Metodología: En esta categoría se expone las tareas relacionadas con el conjunto de prácticas y técnicas utilizadas para planificar y llevar a cabo un proyecto. Colaboré en la redacción de la sección del tablero Kanban en la Sección \ref{sec:flujoTrabajo} y en la definición de las políticas explícitas en la Sección \ref{sec:politicas}.
    \item QA (Quality Assurance): En esta categoría se expone las tareas relacionadas con el conjunto de procesos, técnicas y actividades enfocadas en garantizar la calidad de la aplicación. Para ello realicé de forma individual la corrección de los errores relacionados con la funcionalidad de leyenda de colores y la funcionalidad de verdadero y falso. También dediqué tiempo a probar y desarrollar un plan de pruebas detallado para las funcionalidades de sopa de letras y buscar pictograma. Además, llevé a cabo el test de múltiples funcionalidades, incluyendo las de relacionar conceptos, pictotraductor, espacios para dibujar y de exportar a PDF. Finalmente, realicé un segundo test en las funcionalidades de desarrollo y definiciones, resumen y huecos de matemáticas. A cerca del trabajo en equipo, colaboré en la redacción del plan de pruebas para la funcionalidad de verdadero y falso y brindé apoyo a Álvaro en la refactorización del código.
    \item Memoria: En esta categoría se describen las tareas implicadas en la redacción de los diferentes capítulos del TFG. Individualmente me encargué de redactar la Sección \ref{cap:estructura}, donde se hace una breve descripción de cada cápitulo escrito en dicha memoria. También expliqué la herramienta React en el capítulo sobre herramientas empleadas, la Sección \ref{sec:React}. Además, incluí las imágenes de las funcionalidades diseñadas en Figma. Por último, incluí el plan de pruebas, escrito en un documento aparte, en la memoria en el Anexo \ref{ape:pruebas}. En relación al trabajo participé en la redacción de los objetivos del proyecto en la Sección \ref{cap:objetivos}. También desarrollamos el diseño final de cada funcionalidad en la Sección \ref{subsec:DisenyoFinal} y la arquitectura en la Sección \ref{sub:Arquitectura}. Asimismo, redactamos la implementación de la funcionalidad de búsqueda de pictograma en la Sección \ref{sec:impbuscarpicto} y el capítulo de introducción en inglés en el Capítulo \ref{cap:introduction}

\end{itemize}


\section{Alberto Alejandro Rivas Fernandez}
Durante este TFG he participado tanto en el desarrollo de la aplicación como en la redacción de la memoria, a continuación se explican más en detalle mis aportaciones individuales y de trabajo en grupo.

En cuanto a la captura de requisitos, tuvimos que realizarla en grupo para decidir cuáles eran las funcionalidades que íbamos a incluir en la versión final de la aplicación basándonos en las necesidades de los usuarios. Esta se encuentra explicada en detalle en la Sección \ref{cap:requisitos}.

Para realizar el diseño de la aplicación, todos los integrantes del grupo hicieron un diseño individual para luego compararlo con el de los demás. Debido a esto, tuve que realizar el diseño de la página principal de la aplicación y de cada una de las funcionalidades que decidimos implementar al hacer la captura de requisitos. Estos fueron hechos en Figma ya que quería mostrar exactamente cómo se vería en la aplicación final, incluyendo colores, fuente de letra, etc.

Al principio del proyecto, uno de los desafíos que enfrentamos fue la necesidad de implementar la funcionalidad de exportar documentos en formato Word y PDF. CKEditor tiene un plugin que nos permite hacer esto, pero necesitábamos una licencia de pago para poder utilizarlo. Me puse en contacto con el equipo de CKEditor con el fin de obtener una licencia gratuita. Sin embargo, después de varios intentos y comunicaciones con el equipo, no pudimos obtenerla, por lo que tuvimos que buscar otra librería para implementar el editor de texto.

Una de las librerías que decidimos probar fue Quill.js. En nuestra aplicación usamos modales para cada funcionalidad, entonces hice la prueba de implementar modales utilizando esta librería y utilizando también Redux. Sin embargo decidimos no usar Redux ya que encontramos una forma más sencilla de implementar los modales en Quill. Luego decidimos no usar Quill.js ya que encontramos otra librería llamada Slate.js que se adapta mejor a nuestras necesidades.

Una vez que decidimos que librería utilizar, empezamos a implementar las funcionalidades. La primera funcionalidad que implementé fue la de ejercicios de desarrollo. Para implementar esta funcionalidad tuve que aprender a utilizar Slate para poder insertar el ejercicio en nuestro editor de texto.

Los estilos de la funcionalidad anterior los había realizado con CSS. Sin embargo, decidimos usar Tailwind, el cual es un framework de CSS. Debido a esto, tuve que aprender a usar este framework y refactorizar el código para implementar los estilos con Tailwind en vez de CSS.

Un compañero hizo el testing de la funcionalidad que yo había implementado y encontró algunos errores que tuve que arreglar. Por ejemplo, se podía insertar el ejercicio en el editor sin enunciado, el modal no se reseteaba al cerrarlo, etc.

Después de esto trabajé en implementar la funcionalidad de crear ejercicios de matemática con huecos. Al igual que antes, un compañero realizó el testing de la funcionalidad de crear ejercicios de matemática y encontró una serie de errores que tuve que arreglar. Además de eso, se nos ocurrió añadir funciones nuevas, como poder agregar varias fórmulas al mismo tiempo, por ejemplo, por lo que también realicé esos cambios.

Por último, también realicé la implementación de la funcionalidad de espacios para dibujar.

En cuanto al trabajo en grupo en el código, desarrollamos la funcionalidad de buscar pictograma en conjunto para luego usarla como referencia al implementar el resto de funcionalidades individualmente, realizamos el plan de pruebas de la funcionalidad de ejercicios de verdadero y falso, y también ayudamos a Álvaro con la refactorización del código, con el objetivo de hacerlo más reutilizable.

En cuanto a las metodologías utilizadas para el desarrollo en equipo de este TFG, en el capítulo de metodología (Capítulo \ref{cap:metodologia}), redactamos juntos la sección de visualizar el flujo de trabajo (Sección \ref{sec:flujoTrabajo}), en la que explicamos cómo usaremos el tablero Kanban en nuestro proyecto y también escribimos la sección de políticas explícitas (Sección \ref{sec:politicas}), en la que enumeramos las reglas que debemos seguir al usar este tablero.

En cuanto al control de calidad (QA), realicé el testing de las funcionalidades de ejercicio de definiciones y leyenda de colores para garantizar su correcto funcionamiento. También escribí su plan de pruebas correspondiente. Para hacer el testing de las funcionalidades, estuve realizando varias pruebas con diferentes inputs con el fin de encontrar algún error o algún caso de uso que no funcionase correctamente.

Con respecto a la memoria, me encargué de redactar la sección \ref{cap:adaptaMaterial} que habla sobre la primera versión de AdaptaMaterialEscolar. Además expliqué la implementación de las funcionalidades de ejercicios de desarrollo y de ejercicios de matemática con huecos en las secciones \ref{sec:impdesarrollo} y \ref{sec:impmatematica}.

También redacté la explicación detallada de cada uno de los diseños individuales en la sección \ref{sec:albertoIter} de la memoria para explicar las decisiones de diseño tomadas.

En cuanto al trabajo en grupo en la memoria, hemos redactado la Sección \ref{cap:objetivos}, en la que explicamos los objetivos de este TFG. También hemos traducido a inglés todo el capítulo de introducción (Capítulo \ref{cap:introduction}). En el capítulo de AdaptaMaterialEscolar 2.0 (Capítulo \ref{cap:AdaptaMaterialEscolar2.0}), redactamos la sección en la que explicamos los requisitos de la aplicación (Sección \ref{cap:requisitos}), el diseño final (Sección \ref{subsec:DisenyoFinal}) y la arquitectura (Sección \ref{sub:Arquitectura}), en la sección de funcionalidades, en la que explicamos la implementación de cada una, escribimos la implementación de buscar pictograma (Sección \ref{sec:impbuscarpicto}) en grupo para después hacer las demás individualmente. Por último, diseñamos el logo de la aplicación entre todos.


\section{Johan Sebastian Salvatierra Gutierrez}
Durante el desarrollo de este TFG, he llevado a cabo diversas tareas, tanto de forma individual como en equipo, con el fin de alcanzar los objetivos propuestos. Estas tareas pueden ser clasificadas en diferentes categorías, tales como: estudio de la cuestión, captura de requisitos, diseño de la aplicación, implementación, evaluación, metodología, QA y memoria. A continuación, se detallará cada una de estas categorías.

En la categoría de captura de requisitos (Sección \ref{cap:requisitos}), que se refiere al proceso de identificar y documentar las necesidades de los usuarios, participé en la definición y clasificación conjunta de los requisitos del proyecto.

En la categoría de diseño de la aplicación, se presentan las tareas relacionadas con el proceso de crear y definir la estructura, apariencia y funcionalidad de una aplicación. La primera tarea que realicé fue la creación de los diseños para la iteración competitiva, los cuales se describen en la Sección \ref{sec:johanDisenyo}. Añadí un anexo explicando el diseño que propuse. Junto a Alvaro, realicé los diseños finales en Figma. Como equipo, diseñamos el logo de la aplicación también en Figma y las ideas de cómo se vería el diseño final.
En la implementación, realicé una investigación exhaustiva para encontrar un editor de texto que pudiera satisfacer los requisitos del proyecto y después de evaluar varias opciones, seleccioné Slate como el framework adecuado. Posteriormente, realicé un prototipo con Slate integrado con React para comprobar su viabilidad y propuse su uso para el proyecto. Trabajé en la creación de ejercicios de definiciones (Sección \ref{sec:ejercicioDefiniciones}) y creé pautas en formato SVG que pudieran reutilizarse en diferentes ejercicios. También creé el nodo Tablas con todas las funciones necesarias para gestionarlos y modifiqué su formato para que otras funciones pudieran utilizarlo. Además, desarrollé la función para crear ejercicios de relacionar conceptos (Sección \ref{sec:ejercicioRelacionarConceptos}) y cree la forma para que los ejercicios tuvieran una edición dinámica a través del modal que los creo. Para ello, abstraje los diferentes nodos de los ejercicios en uno solo que permitiera la edición independientemente del ejercicio, y creé la numeración automática de los ejercicios a partir de este nodo. Asimismo, desarrollé parte de la funcionalidad básica del editor, como las formas de listar y los tipos de alineación. Investigué cómo hacer que las imágenes e iconos se comportaran de la forma deseada y dediqué tiempo a investigar la forma en que todos los elementos pudieran exportarse a PDF con el menor impacto posible. También me encargué de configurar y lanzar el host para la aplicación. En cuanto al trabajo en equipo, colaboré en el desarrollo de la funcionalidad de búsqueda de pictogramas y en la pantalla de inicio, y contribuí con Álvaro en la abstracción del código, lo que nos permitió desarrollar una base de código más modular y fácil de mantener.

En la categoría de metodologías, se abordan las tareas relacionadas con el conjunto de prácticas y técnicas empleadas para planificar y llevar a cabo un proyecto. Expliqué las clases de servicio en la Sección \ref{claseDeServicio} que empleamos para el tablero Kanban. Colaboré en la redacción de la sección del tablero Kanban en la Sección \ref{sec:flujoTrabajo} y en la definición de las políticas explícitas en la Sección \ref{sec:politicas}.

En la categoría de QA (Aseguramiento de calidad), se incluyen las tareas relacionadas con el conjunto de procesos, técnicas y actividades enfocadas en garantizar la calidad de la aplicación. En mi rol individual, me encargué de corregir los errores relacionados con la funcionalidad de definiciones, relacionar conceptos y exportar PDF. Asimismo, realicé pruebas de la función "generar resumen" y "desarrollo". En cuanto al trabajo en equipo, colaboré en la redacción del plan de pruebas para la funcionalidad de verdadero y falso, y brindé apoyo a Álvaro en la refactorización del código.

En la categoría de memoria, se describen las tareas implicadas en la redacción de los diferentes capítulos del TFG. He redactado diversas secciones, incluyendo la Sección \ref{cap:motivacio}, donde se presenta la motivación detrás del proyecto, y la Sección \ref{Editor}, donde se explica detalladamente los tipos de nodos y su integración con la aplicación. También he descrito el editor utilizado en esta sección, el cual se ha integrado con éxito en la aplicación. Además, he incluido una explicación detallada de la herramienta Slate en el capítulo de herramientas empleadas, en la Sección \ref{sec:Slate}. En cuanto a mi contribución en equipo, he colaborado en la redacción de los objetivos del proyecto en la Sección \ref{cap:objetivos}. También hemos desarrollado conjuntamente el diseño final de cada funcionalidad en la Sección \ref{subsec:DisenyoFinal} y la arquitectura en la Sección \ref{sub:Arquitectura}. Por último, he participado en la descripción de la implementación de la funcionalidad de búsqueda de pictograma en la Sección \ref{sec:impbuscarpicto} y en la redacción del capítulo de introducción en inglés en el Capítulo \ref{cap:introduction}.

