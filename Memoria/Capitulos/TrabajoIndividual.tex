\chapter{Trabajo Individual}
\label{cap:TrabajoIndividual}

En este capítulo se habla del trabajo que ha realizado cada miembro del equipo en el proyecto.

\section{Álvaro Gómez Sittima}
Durante el desarrollo de este TFG, he colaborado en la creación del código y la redacción de la memoria. Me he involucrado tanto de manera individual como en grupo para alcanzar los objetivos establecidos.

En cuanto al código, he realizado diversas tareas de forma individual, tales como investigar alternativas a CKEditor, proponiendo Quill.js, aunque finalmente nos decantamos por la opción que propuso Johan (Slate) ya que se integraba mejor con React. También he propuesto e integrado Tailwind CSS como framework CSS para la aplicación. Además, he realizado refactorización y limpieza del código para facilitar la escalabilidad y la implementación de nuevas funcionalidades en el futuro. Asimismo, he creado distintos componentes reutilizables para facilitar la creación de nuevas ventanas modales, como por ejemplo el propio componente de la ventana modal, botones o la vista previa. En cuanto a las funcionalidades, me he encargado de desarrollar y arreglar los bugs de las funcionalidades de crear ejercicios de completar huecos, sopa de letras y generar resúmenes. También he realizado las pruebas de las funcionalidades de verdadero/falso y ejercicios de matemáticas con huecos. Además, he integrado herramientas destinadas a mejorar la calidad del código, tales como Prettier, con el fin de definir un estilo común en todo el proyecto, y ESLINT, para llevar a cabo análisis estáticos del código. En colaboración con el grupo, hemos desarrollado la estructura principal de la aplicación, incluyendo la página de inicio, la barra de navegación y el editor. También hemos desarrollado la función de buscar pictograma.

Con respecto a la memoria, he contribuido individualmente en la redacción de diferentes secciones de la memoria, tales como la Sección \ref{sec:herramientasexistentes}, en la cual se realiza una comparativa con otras aplicaciones similares a AdaptaMaterialEscolar, la Sección \ref{cap:pruebas} donde se explica el plan de pruebas seguido en el proyecto y la Sección \ref{sec:tailwind}, en la cual se explica el framework Tailwind CSS. Además, he redactado la implementación de las funcionalidades que he desarrollado: completar huecos (Sección \ref{sec:impcompletarhuecos}), sopa de letras (Sección \ref{sec:impsopaletras}) y generar resumen (Sección \ref{sec:impresumen}). También, he realizado diseños para la iteración competitiva, Sección \ref{sec:iterAlvaro}. Junto a mis compañeros, he participado en la definición de los objetivos (Sección \ref{cap:objetivos}), la definición de requisitos (Sección \ref{cap:requisitos}), el diseño final de la aplicación (Sección \ref{subsec:DisenyoFinal}), la metodología empleada (Sección \ref{cap:metodologia}), la arquitectura de la aplicación (Sección \ref{sub:Arquitectura}) y la implementación de la funcionalidad de buscar pictogramas (Sección \ref{sec:impbuscarpicto}). En cuanto al diseño de las funcionalidades, he colaborado con Johan en la realización de los diseños en Figma.

\section{Dunia Namour Doughani}
Con respecto a la memoria he realizado la Sección \ref{cap:adaptacion} en la cual se explica la adaptación curricular y sus tipos. Además, he redactado la estructura del proyecto. Junto a mis compañeros he realizado los objetivos, la parte relacionada con la metodología de desarollo.

En mi contribución a la memoria del proyecto, me encargué de redactar la Sección \ref{cap:adaptacion} donde se explica la adaptación curricular y sus tipos, además de escribir la estructura del proyecto, Seccioón \ref{cap:estructura}. También expliqué la herramienta React en el capítulo de herramientas empleadas \ref{cap:herramientas}, los diseños que realicé para la iteración competitiva, Sección \ref{sec:duniaIter}, y la implementación de la funcionalidad de verdadero y falso \ref{sec:funcioVF}. Además, incluí las imágenes de las funcionalidades diseñadas en Figma y diseñé cómo se verían los ejercicios en el documento de trabajo.

En la parte de la memoria junto a mis compañeros establecimos los objetivos del proyecto, Sección \ref{cap:objetivos} y redactamos de la sección de metodología de desarrollo todo lo relacionado con el tablero Kanban y las políticas explícitas. También colaboramos en la sección de Adaptamaterial 2.0, donde definimos los requisitos, el diseño final de las funcionalidades y la arquitectura del proyecto. Y juntos escribimos el capítulo de introducción en inglés.

En mi trabajo individual en el código, implementé el modal de la funcionalidad de verdadero y falso y el de la leyenda de colores. Además, realicé pruebas y creé el plan de pruebas para las funcionalidades de sopa de letras y buscar pictograma. También refactoricé la funcionalidad de verdadero y falso para usar Tailwind en lugar de CSS, corregí los errores de leyenda de colores y de la funcionalidad de verdadero y falso, y realicé un segundo test a las funcionalidades de desarrollo y definiciones.

Como trabajo en equipo, escribimos el plan de pruebas para la funcionalidad de verdadero y falso y ayudé en la refactorización del código.

\section{Alberto Alejandro Rivas Fernandez}
\begin{itemize}
    \item Código
          \begin{itemize}
              \item Individual
                    \begin{itemize}
                        \item Investigar cómo añadir un plugin a CKEditor para poder exportar el documento en formato Word
                        \item Contactar con CKEditor con el fin de obtener una licencia gratuita, para poder utilizar las funciones de exportar en formato Word y PDF. No fue posible obtener la licencia por lo que tuvimos que buscar otra librería para implementar el editor de texto
                        \item Intentar implementar un modal en React usando Redux y Quill.js
                        \item Implementar funcionalidad para crear ejercicios de desarrollo usando Slate
                        \item Refactorizar la funcionalidad de crear ejercicios de desarrollo para usar Tailwind en vez de CSS
                        \item Realizar testing en la funcionalidad de crear ejercicios de definición
                        \item Arreglar errores encontrados en mi funcionalidad durante el testing realizado por mi compañero Johan
                        \item Empezar a implementar la funcionalidad de crear ejercicios de matemática con huecos
                    \end{itemize}
          \end{itemize}
    \item Memoria
          \begin{itemize}
              \item Individual
                    \begin{itemize}
                        \item He realizado la Sección \ref{cap:adaptaMaterial} que habla sobre la primera versión de AdaptaMaterialEscolar
                        \item Diseño de la página principal para comprarlo con los de mis compañeros
                        \item Hacer diseños de cada funcionalidad en Figma para compararlos con los de mis compañeros
                        \item Escribir la explicación de mis diseños en la memoria
                        \item Escribir sección en “Herramientas Empleadas” explicando el uso de MaterialUI
                    \end{itemize}
              \item En grupo
                    \begin{itemize}
                        \item Junto a mis compañeros hemos explicado la metodología y priorizado las funcionalidades
                    \end{itemize}
          \end{itemize}
\end{itemize}

\section{Johan Sebastian Salvatierra Gutierrez}
En el desarrollo de nuestro TFG en grupo, he tenido una participación activa en la creación del código y en la redacción de la memoria. En la sección de código, he trabajado tanto individualmente como en grupo para lograr los objetivos propuestos.

En cuanto al trabajo individual, comencé investigando para encontrar un editor de texto adecuado para nuestras necesidades. Después de evaluar varias opciones, propuse el uso de Slate, un framework altamente personalizable que resultó ser una excelente elección. Para probar su viabilidad, creé un prototipo utilizando Slate y también trabajé en la función para crear ejercicios de definiciones, así como en la inserción de tablas y la función para crear ejercicios de relacionar conceptos.

También realicé pruebas y operaciones de testing en la función para crear ejercicios de desarrollo y, después de investigar y experimentar, logré implementar la funcionalidad adecuada para que el comportamiento de las imágenes de los ejercicios fuera el deseado. Asimismo, me encargué de investigar e implementar una forma para que la modificación de los ejercicios fuera más dinámica e intuitiva para los usuarios. Finalmente, realicé pruebas para validar todos los cambios realizados.

En el trabajo en grupo, participé en la creación de la función para buscar pictogramas, así como en el diseño del logo. También ayudé a Álvaro con la abstracción del código.

En la sección de memoria, he realizado una contribución importante tanto en el trabajo individual como en el trabajo en grupo. En particular, en mi trabajo individual, he escrito la Sección \ref{cap:motivacio}, en la cual se explica la motivación detrás de nuestro TFG, y la Sección \ref{claseDeServicio}, donde se aclara las clases de servicio que empleamos para el tablero Kanban. Además, realicé una propuesta de diseño en la sección de herramientas, donde se incluyen subsecciones sobre Figma y Slate.

En cuanto al trabajo en grupo, he colaborado junto a Álvaro en la creación de los objetivos y en el diseño de las funcionalidades. También he participado en la elaboración de la metodología empleada, los requisitos y la arquitectura del proyecto. En resumen, mi aportación ha sido significativa tanto en la creación del código como en la elaboración de la memoria.