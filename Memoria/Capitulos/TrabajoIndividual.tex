\chapter{Trabajo Individual}
\label{cap:TrabajoIndividual}

En este capítulo se habla del trabajo que ha realizado cada miembro del equipo en el proyecto.

\section{Álvaro Gómez Sittima}
Con respecto a la memoria he realizado la Sección \ref{cap:herramientas} en la cual se comparan algunas aplicaciones similares a AdaptaMaterialEscolar. También, he redactado la Sección \ref{cap:pruebas} donde se explica el plan de pruebas que ha seguido este proyecto. Además, junto a mis compañeros, he realizado los objetivos, el diseño y la metodología empleada.

\section{Dunia Namour Doughani}
Con respecto a la memoria he realizado la Sección \ref{cap:adaptacion} en la cual se explica la adaptación curricular y sus tipos. Además, he redactado la estructura del proyecto. Junto a mis compañeros he realizado los objetivos, la parte relacionada con la metodología y el diseño de la aplicación.

\section{Alberto Alejandro Rivas Fernandez}
En relación a la memoria he realizado la Sección \ref{cap:adaptaMaterial} en la cual hablaba sobre la primera versión de AdapatMaterialEscolar. También junto a mis compañeros hemos explicado la metodología y priorizado las funcionalidades, para ello investigué acerca de ellas, sobre todo la funcionalidad de pasar de pdf a word. Asimismo realicé un diseño de la página principal de la aplicación para compararlo con los diseños de mis compañeros. Además, contacté con CKEditor con el fin de obtener una licencia para el proyecto.

\section{Johan Sebastian Salvatierra Gutierrez}
\begin{itemize}
    \item Codigo
        \begin{itemize}
            \item Individual
                \begin{itemize}
                    \item Investigue para obtener un editor que se ajuste a nuestras necesidades
                    \item Propuse Slate el cual es un framework de editor de texto altamente personalizable 
                    \item Cree un prototipo con Slate para poder combrobar si era una opción adecuada
                    \item Función para crear ejercicio de definiciones
                \end{itemize}
            \item En grupo
                \begin{itemize}
                    \item Desarrollamos la función buscar pictograma
                    \item Diseñamos un logo
                \end{itemize}
        \end{itemize}
    \item Memoria
    \begin{itemize}
        \item Individual
            \begin{itemize}
                \item La Seccion \ref{cap:motivacio} en la cual se explica la motivación de nuestro TFG
                \item La Sección \ref{claseDeServicio} donde se aclara las clases de servicio que empleamos para el tablero Kanban.
                \item Cree una propuesta de diseño
            \end{itemize}
        \item En grupo
            \begin{itemize}
                \item Los objetivos
                \item El diseño 
                \item La metodología empleada.
                \item Requisitos
            \end{itemize}
    \end{itemize}
\end{itemize}