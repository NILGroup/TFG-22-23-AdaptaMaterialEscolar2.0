\chapter{Estado del arte}
\label{cap:estadoDelArte}

\section{Adaptación curricular}
\nocite{adaptacionCurricular}
\nocite{adaptacionCurricular2}
En nuestro sistema educativo, aceptar la diversidad del alumnado y la individualidad de cada uno de ellos, constituye la base del quehacer de los docentes. Los profesores deben modificar el currículum (\textit{Regulación de los elementos que determinan los procesos de enseñanza y aprendizaje para cada una de las enseñanzas y etapas educativas.})y el programa de aula con el fin de mejorar el desarrollo del aprendizaje. Para poder realizar esta actividad, el profesorado deberá detectar, evaluar y valorar al alumno y a los elementos curriculares y del entorno.

Una vez detectado esto, el profesorado se encuentra preparado para amoldarse a las circunstancias del alumno, gracias a esto el estudiante adquiere la atención educativa que requiere, logrando una mejora en su desarrollo personal y social.

Con esto, determinamos que la adaptación curricular es un mecanismo para particularizar el currículum, ayudando a efectuar las labores docentes con el fin de apoyar al alumno a conseguir el nivel exigido por el Currículum Oficial. \nocite{adaptacionIntro}

\subsection{Concepto de la adaptación curricular}
El BOE (\citeyear{BOE}) define la adaptación curricular como \textit{la medida de modificación de los elementos del currículo a fin de dar respuesta a las necesidades del alumnado. En todo caso, la adaptación tendrá como referente los objetivos y las competencias básicas del currículo que corresponda.}

Es decir, la adaptación curricular es cualquier adaptación personal para estudiantes, cuyas necesidades no se encuentran cubiertas por el currículo ordinario y por tanto, no pueden acceder a él de la misma manera que sus compañeros.

En definitiva, son planes de acción y estrategias didácticas que incluye las modificaciones del currículo, asegurándose de que los estudiantes tengan éxito en el proceso de aprendizaje y alcancen los objetivos generales definidos.

Para poder aplicar la adaptación curricular a un alumno se debe precisar el tipo de adaptación en función de sus necesidades, para que pueda alcanzar los objetivos propuestos. Para ello, definimos los tipos de adaptaciones que más uso tienen, enfocándonos tanto en el acceso como en el currículum.

\subsection{Adaptaciones de acceso}
Las adaptaciones de acceso permiten al alumno acceder a los diferentes componentes del currículum. No implica una adaptación del currículum sino una adaptación en los recursos materiales, espaciales o de comunicaicón para que los alumnos con necesidades educativas especiales puedan desarollar el currículum.
Dicha adaptación puede tomar a su vez diferentes tipos.

\subsubsection{Acceso espacial}
Hacen referencia a las adaptaciones en relación con el espacio. Destacan las siguientes:
\begin{itemize}
    \item \textbf{Adaptación en la sonorización del aula:} Significa que las aulas deben tener un cierto nivel de volumen adecuado, sin que haya ruidos contínuos, sin eco,etc. Dichas clases son especialmente adecuadas para estudiantes que tienen impedimentos auditivos o visuales o que requieren, por su propia condición especial, un entorno con pocos estímulos auditivos que les distraiga.
    \item \textbf{Adaptación en la iluminación del aula:} Las aulas bien iluminadas requieren que estas no tengan sombras, deben poseer ventanales que suministren luz natural o en su lugar luz artificial. Estos requisitos son necesarios para los estudiantes con discapacidades sensoriales.
    \item \textbf{Adaptación en el espacio físico:} Es todo aquello relacionado con la ausencia de barreras arquitectónicas:  braille en las puertas, ascensores, pasamanos, remate de las escaleras rugoso, etc. En esta sección también se encuentran los aspectos relacionados con la ubicación del aula (sin escaleras, lugares poco ruidosos) y con la disposición del estudiante dentro del aula (al lado de un enchufe, del profesor, de la puerta,etc.)
\end{itemize}

\subsubsection{Acceso material}
Se adaptan materiales de uso frecuente o se introducen nuevos instrumentos que faciliten el desarollo del aprendizaje.
\begin{itemize}
    \item \textbf{Material adaptado:} Se refieren a materiales que se usan habitualmente, se adaptan para un uso apropiado por parte de los alumnos con necesidades especiales, ejemplo de ello es plastificar un libro o hacer dibujos con relieve.
    \item \textbf{Materiales específicos:} Los materiales específicos deben superar las dificultades de los niños, por ejemplo el mobiliario, las sillas y mesas deben de estar adaptadas, los comunicadores electrónicos con salidas de voz o escrita, etc.
\end{itemize}

\subsubsection{Acceso de comunicación}
Algunos alumnos son incapaces de comprender o expresarse por medio del lenguaje oral, o su nivel no es suficiente para poder comunicarse correctamente. Dichos alumnos requieren estudiar y usar códigos de comunicación suplementarios al lenguaje oral, o alguna alternativa al mismo. Aprender dichos códigos de comunicación facilitarán el acceso a los elementos curriculares ordinarios, les concederá una herramienta imprescindible tanto para el desarrollo de algunas competencias y aprendizajes de diferentes contenidos, como para relacionarse y comunicarse con el resto de personas. De esta manera, podemos destacar los siguientes sistemas que mejoran la  entrada a la comunicación:
\begin{itemize}
    \item \textbf{Sistemas alternativos a la comunicación:} En estos sistemas no se requiere del lenguaje oral para su utilización. Ejemplo de estos procedimiento son los símbolos pictográficos para la comunicación, Sistema de comunicación Bliss (sistema gráfico), la lengua de signos española,etc.
          los gestos, lenguaje corporal,lenguaje por señales, tableros de comunicación

    \item \textbf{Sistemas complementarios a la comunicación:} No sustituyen al lenguaje oral, solo lo acompaña añadiendo más información visual. Se destacan la palabra complementada (para que las personas con discapacidad auditiva puedan distinguir entre fonemas con la misma articulación), la comunicación bimodal (Es un sistema de comunicación que utiliza el habla y los signos al mismo tiempo)
\end{itemize}

\subsection{Adaptaciones del currículo}

Como se ha mencionado anteriormente, la adaptación curricular es un conjunto de refinamientos y cambios en los elementos del currículum para adaptar la respuesta educativa a las necesidades formativas de los estudiantes especiales.

Las adaptaciones curriculares se dividen en dos tipos, las adaptaciones curriculares significativas y las adaptaciones curriculares no significativas.
\subsubsection{Adaptaciones curriculares significativas}
Según el BOE (\citeyear{BOE}), \textit{una adaptación curricular será significativa cuando la modificación de los elementos del currículo afecten al grado de consecución de los objetivos, los contenidos y los aprendizajes imprescindibles que determinan las competencias básicas en la etapa, ciclo, grado, curso o nivel correspondiente. }

Es decir, la adaptación curricular significativa son los ajustes que se realizan en el currículum. Para su elaboración e implementación se debe seguir el criterio de menor a mayor significatividad, el enfoque sería el siguiente:
\begin{itemize}
    \item Inclusión. Se introducen elementos curriculares no presentes en el currículo. Puede incorporar objetivos, contenidos, criterios de evaluación, etc conforme a las necesidades del alumno.
    \item Reformulación. Esta adaptación conlleva la amplia modificación de los elementos del currículo.
    \item Temporalización fuera de ciclo. Los alumnos con ritmo de aprendizaje lento con respecto a sus compañeros, tendrán la oportunidad de conseguir los objetivos en otro ciclo posterior, posponiendo a otras etapas algunos elementos curriculares.
    \item Eliminación. Este tipo de adaptación es la más significativa. Inicialmente se deben eliminar contenidos, a continuación, criterios de evaluación y objetivos, finalmente se propondrá quitar material.
\end{itemize}

\subsubsection{Adaptaciones curriculares no significativas}
Son adaptaciones que no modifican sustancialmente el contenido del currículo oficial, es decir, se adapta la metodología, las actividades, los tiempos, las técnicas e instrumentos de evaluación. Para su elaboración se debe  seguir el criterio de menor a mayor significatividad, los aspectos serían los siguientes:
\begin{itemize}
    \item Cómo evaluar: Se ajusta la manera de evaluar a las necesidades del alumno, ejemplo de ello es cuando un alumno con escayola no puede realizar un examen escrito por lo que se le adapta la forma de evaluar realizando un examen oral.
    \item Metodología: Aquí se hace mención a cómo se enseña, es decir, al la forma de transmitir el aprendizaje. El desarollo de la enseñaza-aprendizaje ha de ser activo, partiendo desde las necesidades del alumno. Además, ha de ser creativo, cooperativo y buscar un opción distinta al método tradicional de trabajo.
    \item Priorización de objetivos o contenidos: Dentro de la planificación se podría dar más valor a unos contenidos que a otros.
    \item Temporalización de contenidos u objetivos: Permitir más tiempo para alcanzar algunos de los contenidos pero respetando el ciclo, ejemplo de esto es trabajar elementos de segundo en tercero sin que los concetos a trabajar sean muy significativos.
\end{itemize}

\subsection{Ejemplificación de algunas adaptaciones curriculares asociadas a diferentes necesidades }

\begin{itemize}
    \item Discapacidad motora: Es un grupo de alteraciones que se producen como consecuencia de diversas anomalías en los Sistemas que forman el movimiento. Este tipo de discapacidad requiere  adaptaciones de acceso tales como rampas, pasamanos, suelos antideslizantes, etc.

          En relación a las adaptaciones significativas atañen sobre todo al área de Educación Física, Música o Plástica ya que en estas modalidades se precisa del manejo de instrumentos. Por ejemplo, en el área de Educación Física se realizarán ejercicios en los cuales el alumno utilizará los músculos que presenten una mayor funcionalidad, con el fin de mejorar la capacidad de respuesta.

          Con respecto a las adaptaciones no significativas se debe adaptar la forma de evaluar ya que se debe tener en cuenta su movilidad. Un ejemplo de ello sería encargar al alumno pequeñas tareas que fomenten la autonomía.

    \item Discapacidad auditiva: Es la pérdida parcial o completa de la audición, supone la obtención del lenguaje oral por otras vías como por la visual.

          Con respecto a las adaptaciones de acceso, el alumno se debe encontrar en una zona del aula en la que no haya muchas sombras ya que la adquisición de conocimientos se realiza por vía visual. Por otro lado, los alumnos con una pérdida parcial de la audición necesitan de un ambiente poco ruidoso.

          En relación a las adaptaciones curriculares significativas, los profesores deberán trabajar de forma conjunta con especialistas en audición y lenguaje para que el alumno logre alcanzar los objetivos conectados con el lenguaje oral.

          En cuanto a las adaptaciones curriculares no significativas, hay que tener en cuenta la manera de evaluar(se debe dar más valor al contenido que a los aspectos formales), además de la forma de hablar al alumno, esta debe ser de un modo claro, sin gesticular excesivamente, etc.

    \item Discapacidad intelectual: La discapacidad intelectual es una condición que se caracteriza por limitaciones significativas tanto en el funcionamiento intelectual como en el comportamiento adaptativo, que afectan la capacidad de una persona para funcionar en la vida diaria. En este tipo de discapacidad no es muy relevante la adaptación de acceso, pero podemos destacar el posicionamiento del alumno en el aula, de manera que se encuentre en una zona donde no tenga muchas distracciones.

          Las adaptaciones curriculares significativas se aplicarán en función de su nivel de competencia curricular.

          En relación a las adaptaciones curriculares no significativas se centrarán en la metodología, como por ejemplo se incentivará la motivación y el refuerzo positivo.

    \item Espectro autista: El espectro del autismo se refiere a una variedad de trastornos del neurodesarrollo que se caracterizan por déficits sociales y de comunicación, y conductas restringidas y repetitivas.

          En relación a las adaptaciones de acceso al espacio se precisa no realizar grandes cambios en la disposición del mobiliario. También destacamos las adaptaciones de comunicación, ya que las personas autistas se caracterizan por la ausencia de comunicación según el nivel, para ayudar a romper la barrera de la comunicación lo que realizan es asociar palabras con gestos e impulsar un refuerzo positivo.

          En cuanto a las adaptaciones curriculares significativas, se debe introducir o priorizar el contenido en lo que respecta a la  comunicación o rediseñar los objetivos o elementos que no alcancen.

          Las adaptaciones curriculares no significativas se centran en la metodología. Las actividades deben de ser consistentes, con una estructura y organización claras.

    \item Altas capacidades intelectuales: En este caso no se requiere de adaptaciones de acceso ya que  este tipo de alumnos no tienen dificultades para acceder al currículum.

          Con respecto a las adaptaciones curriculares significativas lo que deberían realizar los profesores es ampliar el currículum añadiendo objetivos y contenidos.

          Las adaptaciones curriculares no significativas hacen hincapié en la metodología por ejemplo, haciendo actividades de ampliación.

\end{itemize}
\nocite{adaptacionUNED}


\section{Herramientas existentes para adaptaciones curriculares}
Widgit Symbols\footnote{\url{https://www.widgit.com/about-symbols/widgit_symbol_set.htm}} es un conjunto de símbolos coloridos y sencillos que cubren un amplio vocabulario de palabras y admite 17 idiomas, principalmente el inglés. Los propios creadores de Widgit también ofrecen herramientas\footnote{\url{https://www.widgit.com/products/index.htm}} que permiten combinar la escritura de texto con pictogramas utilizando los Widgit Symbols. Por ejemplo, InPrint 3\footnote{\url{https://www.widgit.com/products/inprint/index.htm}} permite utilizar plantillas y editarlas para relacionar texto con pictogramas facilitando la adaptación de cualquier información, documento o recurso didáctico.

EducaPlay\footnote{\url{https://es.educaplay.com/?lang=es}} es una web que permite crear actividades interactivas o juegos didácticos. Entre las actividades que se pueden realizar en EducaPlay hay sopas de letras, pruebas, juegos de memoria, mapas interactivos, ejercicios de relacionar columnas, etc. EducaPlay está pensada para que los alumnos realicen las actividades desde un ordenador o dispositivo móvil y, por lo tanto, si se busca imprimir actividades para que los alumnos las realicen en papel, puede que no sea la mejor opción. Aun así, la herramienta te permite imprimir las actividades y descargarlas para que se puedan realizar sin acceso a internet.

\subsection{AdaptaMaterialEscolar1.0}
La finalidad de AdapataMaterialEscolar es proporcionar una herramienta para el profesorado con el fin de adaptar los contenidos de las asignaturas de forma intuitiva, rápida y simple. Actualmente permite lo siguiente:
\begin{itemize}
    \item Subir un documento fuente PDF, a partir del cual se puede realizar las adaptaciones, ejercicios, etc.
    \item Editor, en el que se pueden añadir y modificar las adaptaciones. También sirve como editor de texto, en el que se puede cambiar la fuente de letra, el color, posicionamiento del texto, etc.
    \item Buscador de pictogramas.
    \item Ejercicios de completar huecos: dado un texto se pueden seleccionar las palabras que deben ser sustituidas por espacios en blanco.
    \item Ejercicios de definiciones.
    \item Ejercicios de desarrollo: permite crear un enunciado y añadir un cierto número de líneas para la respuesta.
    \item Sopa de letras: el usuario introduce las palabras que desea que se pongan en la sopa de letras y el tamaño de la matriz.
    \item Ejercicios de verdadero o falso.
\end{itemize}

La aplicación se ha creado siguiendo un Diseño Centrado en el Usuario, para que la aplicación se adapte a las necesidades reales de los usuarios finales (los docentes). La captura de requisitos se realizó hablando directamente con el usuario final para poder conocer sus necesidades reales. Para esto se hicieron varias reuniones con 2 profesoras del Aula TEA (Trastornos del Espectro Autista) del IES Maestro Juan de Ávila de Ciudad Real.

Para el diseño de la aplicación se realizó una primera iteración de diseño competitivo y el diseño resultante se presentó a los usuarios finales para obtener su feedback.

Para desarrollar AdaptaMaterialEscolar1.0, se utilizó React. Para simplificar la gestión del estado de la aplicación, se utilizó Redux. Para el editor de texto de la aplicación se utilizó CKEditor. Por último, para el buscador de pictogramas, se empleó la API de ARASAAC. Esta API nos permite hacer una petición con un término de búsqueda y nos devuelve una serie de pictogramas relacionados.

Finalmente la aplicación fue evaluada por los usuarios finales, con el objetivo de descubrir si la aplicación es, en cuanto a la adaptación curricular no significativa, realmente útil para los profesores y si les ayudaba a resolver sus problemas. Para esto se creó un exámen de Ciencias Naturales adaptado usando AdaptaMaterialEscolar1.0. Luego, se replicó este exámen con los profesores para mostrarles cómo se usaría la herramienta en situaciones reales.

Después de esta demostración, se les hizo una encuesta a los profesores para que pudieran dar su opinión y feedback. Este cuestionario tenía preguntas sobre usabilidad, diseño, funcionalidades y utilidad real de la aplicación. Además, se paso a los evaluadores el cuestionario SUS (System Usability Scale), que sirve para medir que tan buena es la usabilidad de un sistema. Esta evaluación de la aplicación se realizó con 6 profesores.

Como resultado de la evaluación se llegó a la conclusión de que la aplicación sí resuelve problemas reales que tienen los profesores y que se debería seguir desarrollando.

En el cuestionario SUS se obtuvo un 99 sobre 100. También se obtuvo un 4,6 sobre 5 en estética y se recomendó que cada pestaña de las funcionalidades fuese de un color diferente para diferenciarlas mejor. También se observó que los pictogramas a veces eran muy pequeños y debían poder aumentarse de tamaño.

A partir del feedback de los usuarios finales que evaluaron la aplicación, se obtuvo una nueva lista de requisitos a implementar para mejorar la aplicación:

\begin{itemize}
    \item Traducir pictogramas a lenguaje natural y viceversa.
    \item Cuadrícula para ejercicios de matemática.
    \item Poder añadir doble pauta, en vez de renglones de una línea para determinar el tamaño de letra del alumno.
    \item Añadir herramienta para recortar imágenes.
    \item Añadir encabezado con el nombre del centro educativo, asignatura y nombre del alumno.
    \item Permitir añadir espacio para dibujar en los ejercicios.
    \item Ejercicios de cálculo con fórmulas con huecos que puedan ser rellenados por el alumno.
    \item Enumerar ejercicios automáticamente.
    \item Añadir una fuente de texto parecida a la que suelen aprender la mayoría de los alumnos cuando empiezan a escribir.
\end{itemize}