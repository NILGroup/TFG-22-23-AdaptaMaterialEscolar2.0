\chapter{Estado del arte}
\label{cap:estadoDelArte}

\section{Introducción}
En nuestro sistema educativo, aceptar la diversidad del alumnado y la individualidad de cada uno de ellos, constituye la base del quehacer de los docentes. Los profesores deben modificar el currículum y el programa de aula con el fin de adaptar el desarrollo del aprendizaje. Para poder realizar esta actividad, el profesorado deberá detectar, evaluar y valorar al alumno y a los elementos curriculares y del entorno.

Una vez detectado esto, el profesorado se encuentra preparado para amoldarse a las circunstancias del alumno, gracias a esto el estudiante adquiere la atención educativa que requiere, logrando una mejora en su desarrollo personal y social.

Con esto, determinamos que la adaptación curricular es un mecanismo para particularizar el currículum, ayudando a efectuar las labores docentes con el fin de apoyar al alumno a conseguir el nivel exigido por el Currículum Oficial.

\section{Concepto de la adaptación curricular}
El BOE define la adaptación curricular como "la medida de modificación de los elementos del currículo a fin de dar respuesta a las necesidades del alumnado. En todo caso, la adaptación tendrá como referente los objetivos y las competencias básicas del currículo que corresponda."

Es decir, la adaptación curricular es cualquier adaptación personal para estudiantes, cuyas necesidades no pueden satisfacer el currículo ordinario y por tanto, no pueden acceder a él de la misma manera que sus compañeros.

En definitiva, son planes de acción y estrategias didácticas que incluye las modificaciones del currículo, asegurándose de que los estudiantes tengan éxito en el proceso de aprendizaje y alcancen los objetivos generales definidos.

Para poder aplicar la adaptación curricular a un alumno se debe precisar el tipo de adaptación en función de sus necesidades, para que pueda alcanzar los objetivos propuestos. Para ello, definimos los tipos de adaptaciones que más uso tienen, enfocándonos tanto en el acceso como en el currículum. 

\section{Adaptaciones de acceso}
Las adaptaciones de acceso permiten al alumno acceder a los diferentes componentes del currículum. No implica una adaptación del currículum sino únicamente un acceso a él. 
Dicha adaptación puede tomar a su vez diferentes tipos:

\subsection{Acceso espacial}
Hacen referencia a las adaptaciones en relación con el espacio. Destacan las siguientes:
\subsubsection{Adaptación en la sonorización del aula}
Significa que las aulas deben tener un cierto nivel de volumen adecuado, sin que haya ruidos contínuos, sin eco,etc. Dichas clases son especialmente adecuadas para estudiantes que tienen impedimentos auditivos o visuales o ,que requieren, por su propia condición especial, un entorno con pocos estímulos auditivos que les distraiga.
\subsubsection{Adaptación en la iluminación del aula}
Las aulas bien iluminadas requieren que estas no tengan sombras, deben poseer ventanales que suministren luz natural o en su lugar luz artificial. Estos requisitos son necesarios para los estudiantes con discapacidades sensoriales.
\subsubsection{Adaptación en el espacio físico}
Es todo aquello relacionado con la ausencia de barreras arquitectónicas:  braille en las puertas,ascensores, pasamanos, remate de las escaleras rugoso, etc. En esta sección también se encuentran los aspectos relacionados con la ubicación del aula(sin escaleras, lugares poco ruidosos) y con la disposición del estudiante dentro del aula ( al lado de un enchufe, del profesor, de la puerta,etc.)

\subsection{Acceso material}
\subsubsection{Material adaptado}
Se refieren a materiales que se usan habitualmente, se adaptan para un uso apropiado por parte de los alumnos con necesidades especiales, ejemplo de ello es plastificar un libro o hacer dibujos con relieve.
\subsubsection{ Materiales específicos}
Los materiales específicos deben superar las dificultades de los niños, por ejemplo el mobiliario, las sillas y mesas deben de estar adaptadas, los comunicadores electrónicos con salidas de voz o escrita, etc.
\subsubsection{Acceso de comunicación}
Algunos alumnos son incapaces de comprender o expresarse por medio del lenguaje oral, o su nivel no es suficiente para poder comunicarse correctamente. Dichos alumnos requieren estudiar y usar códigos de comunicación suplementarios al lenguaje oral, o alguna alternativa al mismo. Aprender dichos códigos de comunicación facilitarán el acceso a los elementos curriculares ordinarios, les concederá una herramienta imprescindible tanto para el desarrollo de algunas competencias y aprendizajes de diferentes contenidos, como para para relacionarse y comunicarse con el resto de personas.

\section{Adaptaciones curriculares}
Como se ha mencionado anteriormente, la adaptación curricular es un conjunto de refinamientos y cambios en los elementos del currículum para adaptar la respuesta educativa a las necesidades formativas de los estudiantes especiales. 

Las adaptaciones curriculares se dividen en dos tipos, las adaptaciones curriculares significativas y las adaptaciones curriculares no significativas. 
\subsection{Adaptaciones curricualares significativas}
Según el BOE, "una adaptación curricular será significativa cuando la modificación de los elementos del currículo afecten al grado de consecución de los objetivos, los contenidos y los aprendizajes imprescindibles que determinan las competencias básicas en la etapa, ciclo, grado, curso o nivel correspondiente."

Es decir, la adaptación curricular significativa son los ajustes que se realizan en el currículum. Para su elaboración e implementación se deben seguir el criterio de menor a mayor significatividad, el enfoque sería el siguiente:
\begin{itemize}
    \item Inclusión. Se introducen elementos curriculares no presentes en el currículo. Puede incorporar objetivos, contenidos, criterios de evaluación.etc conforme a las necesidades del alumno.
    \item Reformulación. Esta adaptación conlleva la amplia modificación de los elementos del currículo.
    \item Temporalización fuera de ciclo. Los alumnos con ritmo de aprendizaje lento con respecto a sus compañeros, tendrán la oportunidad de conseguir los objetivos en otro ciclo posterior, posponiendo a otras etapas algunos elementos curriculares.
    \item Eliminación. Este tipo de adaptación es la más significativa. Inicialmente se deben eliminar contenidos, a continuación, criterios de evaluación y objetivos, finalmente se propondrá quitar material.
\end{itemize}

\subsection{Adaptaciones curricualares no significativas}
Son adaptaciones que no modifican sustancialmente el contenido del currículo oficial, es decir, se adapta la metodología, las actividades, los tiempos, las técnicas e instrumentos de evaluación. Para su elaboración se debe  seguir el criterio de menor a mayor significatividad, los aspectos serían los siguientes:
\begin{itemize}
    \item Cómo evaluar: se ajusta la manera de evaluar a las necesidades del alumno, ejemplo de ello es cuando un alumno con escayola no puede realizar un examen escrito por lo que se le adapta la forma de evaluar realizando un examen oral.
    \item Metodología: aquí se hace mención a cómo se enseña por ejemplo, la manera de explicar a algunos alumnos que a otros, es decir, se puede dar la situación de que un alumno requiera que le expliquen las cosas más lentamente que a otro estudiante.
    \item Priorización de objetivos o contenidos: Dentro de la planificación se podría dar más valor a unos contenidos que a otros.
    \item Temporalización de contenidos u objetivos: permitir más tiempo para alcanzar algunos de los contenidos pero respetando el ciclo, ejemplo de esto es trabajar elementos de segundo en tercero sin que sea muy significativo.
\end{itemize}

\section{Ejemplificación de algunas adaptaciones curriculares asociadas a diferentes necesidades }

\begin{itemize}
    \item Discapacidad motora: Es un grupo de alteraciones que se producen como consecuencia de diversas anomalías en los Sistemas que forman el movimiento. Este tipo de discapacidad requiere  adaptaciones de acceso tales como rampas, pasamanos, suelos antideslizantes, etc. 
    
    En relación a las adaptaciones significativas atañen sobre todo al área de Educación Física, Música o Plástica ya que en estas modalidades se precisa del manejo de instrumentos. 

    Con respecto a las adaptaciones no significativas se debe adaptar la forma de evaluar ya que se debe tener en cuenta su movilidad. 
    
    \item Discapacidad auditiva: Es la pérdida parcial o completa de la audición que supone la obtención del lenguaje oral por otras vías como por la visual. 
    
    Con respecto a las adaptaciones de acceso, el alumno se debe encontrar en una zona del aula en la que no haya muchas sombras ya que la adquisición de conocimientos se realiza por vía visual. Por otro lado, para los alumnos con una pérdida parcial de la audición necesitan de un ambiente poco ruidoso. 
    
    En relación a las adaptaciones curriculares significativas los profesores deberán trabajar de forma conjunta con especialistas en audición y lenguaje para que logre alcanzar los objetivos conectados con el lenguaje oral.

    En cuanto a las adaptaciones curriculares no significativas, hay que tener en cuenta la manera de evaluar, además de la forma de hablar al alumno, esta debe ser de un modo claro, sin gesticular excesivamente, etc.
    
    \item Discapacidad intelectual: El funcionamiento intelectual es significativamente más bajo que la media. En este tipo de discapacidad no es muy relevante las adaptaciones de acceso pero podemos destacar el posicionamiento del alumno en el aula de manera que se encuentre en una zona donde no tengas muchas distracciones  que inciten a dispersar su atención.
    
    Las adaptaciones curriculares significativas se aplicarán en función de su nivel de competencia curricular. 

    En relación a las adaptaciones curriculares no significativas se centrarán en la metodología como por ejemplo, se incentivará la motivación y el refuerzo positivo.
    
    \item Espectro autista: Discapacidades del desarrollo en el cerebro. 
    
    En relación a las adaptaciones de acceso al espacio para este tipo de discapacidad es preciso no realizar grandes cambios en relación a la disposición del mobiliario. También destacamos las adaptaciones de comunicación ya que las personas autistas se caracterizan por la ausencia de comunicación , para ayudar a romper la barrera de la comunicación lo que realizan es asociar palabras con gestos e impulsar un refuerzo positivo.

    En cuanto a las adaptaciones curriculares significativas se debe introducir o priorizar el contenido en lo que respecta a la  comunicación o rediseñar los objetivos o elementos que no alcancen.

   Las adaptaciones curriculares no significativas se centran en la metodología. Las actividades deben de ser consistentes, con una estructura y organización clara.

    \item Altas capacidades intelectuales: En este caso no se requiere de adaptaciones de acceso ya que  este tipo de alumnos no tienen dificultades para acceder al currículum. 
    
    Con respecto a las adaptaciones curriculares significativas lo que deberían realizar los profesores es ampliar el currículum añadiendo objetivos y contenidos.

   Las adaptaciones curriculares no significativas hacen hincapié en la metodología por ejemplo, haciendo actividades de ampliación.

\end{itemize}

\section{Conclusión}
Tras una exhaustiva documentación acerca de las adaptaciones y aportar ejemplos de ello, determinamos que las adaptaciones han de ser individualizadas ya que cada alumno tiene sus propias dificultades. El profesorado ha de ser el encargado de la adaptación, pero su realización es algo tediosa y no disponen de herramientas que faciliten la modificación del material. Por ello, en este TFG nos centraremos exclusivamente en las adaptaciones curriculares no significativas realizando una aplicación para que el profesorado pueda adaptar el material. Por otra parte, nuestro enfoque está en las adaptaciones mencionadas anteriormente, ya que son las únicas que podemos tratar puesto que no tenemos la potestad de modificar y/o acceder al currículum. 


\section{AdaptaMaterialEscolar1.0}

Una vez introducida la adaptación curricular, pasamos a describir AdaptaMateriaEscolar1.0 con el fin de exponer la aplicación anterior y de esta forma poder comprender la nueva versión de esta. 


\subsection{Captura de requisitos}
La captura de requisitos se realizó hablando directamente con el usuario final para poder conocer sus necesidades reales. Para esto se hicieron varias reuniones con 2 profesoras del Aula TEA (Trastornos del Espectro Autista) del IES Maestro Juan de Ávila de Ciudad Real.

También, se presentó la aplicación en un workshop del grupo de investigación (Natural Interaction based on Language). Los profesores pidieron a los alumnos del TFG añadir una funcionalidad que permita introducir en el papel el espacio suficiente para aquellos los alumnos con un amplio tamaño de letra.

A partir de estas reuniones se decidió que la aplicación fuera web, además de no ser necesario un sistema de iniciar sesión, y se capturó una lista de funcionalidades que se deberían implementar. Se llegó a la conclusión de que la aplicación fuera web ya que no todos los profesores tienen un ordenador personal en el que puedan instalar aplicaciones. Por otro lado, se decidió que no hacía falta iniciar sesión en la aplicación, para que los profesores no tengan que usar su información personal. Para poder priorizar las funcionalidades, se le envió una lista con todos los requisitos a las profesoras, para que ellas les dieran una importancia del 1 al 3, siendo el 1 la menor importancia y el 3 la mayor importancia. Luego, el equipo de desarrollo les dio una dificultad del 1 al 3, siendo el 1 la mayor dificultad y el 3 la menor dificultad. A continuación, se hizo una tabla con las importancias y dificultades medias de cada requisito. Estos 2 valores se multiplicaron para obtener un resultado, según el cual se ordenaron las funcionalidades de mayor prioridad a menor.


\subsection{Diseño de la aplicación}
Para realizar el diseño de la aplicación, el equipo empezó haciendo un boceto en papel entre todos. Luego, llevaron a cabo un diseño en Moqups,  una aplicación que permite crear prototipos, diagramas, etc. 

Para conseguir mejorar el diseño, decidieron que cada miembro del equipo debería realizar un prototipo por su cuenta para luego ponerlos todos en común, y poder ver las diferentes perspectivas. Se hizo así para que cada persona pudiera plasmar sus ideas sin ser influenciada por las ideas de los demás.

Finalmente, se realizó el diseño final de la aplicación después de poner en común los prototipos de todo el equipo. Una vez terminado, se le envió a las profesoras del Aula TEA para poder recibir feedback, el cual fue positivo.


\subsection{Implementación}
Las funcionalidades implementadas fueron las siguientes:

\begin{itemize}
    \item Subir un documento fuente PDF, a partir del cual se puede realizar las adaptaciones, ejercicios, etc.
    \item Editor, en el que se pueden añadir y modificar las adaptaciones. También sirve como editor de texto, en el que se puede cambiar la fuente de letra, el color, posicionamiento del texto, etc.
    \item Buscador de pictogramas que permite añadirlos al editor.
    \item Ejercicio de completar huecos, en el que a partir de un texto puedes seleccionar las palabras que deben ser completadas.
    \item Ejercicio de definiciones.
    \item Ejercicio de desarrollo, en el que se puede crear un enunciado y añadir un cierto número de líneas para la respuesta.
    \item Sopa de letras, con ciertas palabras dadas por el usuario y un cierto tamaño.
    \item Ejercicio de verdadero o falso.
\end{itemize}

Para desarrollar AdaptaMaterialEscolar1.0, se utilizó React, una biblioteca de JavaScript de código abierto mantenida por Meta para crear interfaces de usuario. React nos permite crear un front-end interactivo y complejo de una forma mucho más fácil que utilizando Javascript puro. Se basa en una arquitectura de componentes reutilizables que tienen su propio estado y que juntos forman una aplicación completa.

Para simplificar la gestión del estado de la aplicación, se utilizó Redux, un patrón de arquitectura de datos que permite controlar el estado de la aplicación de manera predecible, reduciendo el número de relaciones entre componentes de la aplicación, manteniendo un flujo de información sencillo.

Para las pruebas de unidad se utilizó Jest, una librería de JavaScript para testeo mantenida por Meta.

En vez de implementar un editor de texto, se decidió utilizar una librería externa, ya que desarrollar un editor desde cero llevaría demasiado trabajo y sería ineficiente porque ya existen editores que se pueden reutilizar. En este caso se utilizó CKEditor.

Por último, para el buscador de pictogramas, se empleó la API de ARASAAC, la cual fue creada por el Centro Aragonés para la comunicación Aumentativa y Alternativa. Esta API nos permite hacer una petición con un término de búsqueda y nos devuelve una serie de pictogramas relacionados.


\subsection{Evaluación}

El objetivo de la evaluación fue descubrir si la aplicación es realmente útil para los profesores y si les ayuda a resolver sus problemas. Para esto se creó un exámen de Ciencias Naturales adaptado usando AdaptaMaterialEscolar1.0. Luego, se replicó este exámen con los profesores para mostrarles cómo se usaría la herramienta en situaciones reales.

Después de esta demostración, se les hizo una encuesta a los profesores para que pudieran dar su opinión y feedback. Este cuestionario tenía preguntas sobre usabilidad, diseño, funcionalidades y utilidad real de la aplicación. En el cuestionario se usó una serie de preguntas llamadas Escala de Usabilidad de un Sistema o SUS (System Usability Scale), que sirven para medir que tan buena es la usabilidad de un sistema. 

Esta evaluación de la aplicación se realizó con varios profesores en diferentes días. Primero se hizo con las 2 profesoras del aula TEA, la orientadora y otra profesora del IES Maestro Juan de Ávila, de Ciudad Real. Luego 3 docentes de Biología, Historia y Geografía se pusieron en contacto con el equipo ya que también querían probar la aplicación. Por último, la jefa de estudios y profesora del IES Pedro Álvarez de Sotomayor, de Manzanares, también se vio interesada ya que en su instituto había un alumno con sordera total.

Como resultado de la evaluación se llegó a la conclusión de que la aplicación sí resuelve problemas reales que tienen los profesores y que se debería seguir desarrollando.

En la encuesta se obtuvo un 99 sobre 100 en la escala SUS. También se obtuvo un 4,6 sobre 5 en estética y se recomendó que cada pestaña de las funcionalidades fuese de un color diferente para diferenciarlas mejor. También se observó que los pictogramas a veces eran muy pequeños y debían poder aumentarse de tamaño.

Se obtuvo una nueva lista de requisitos a implementar para mejorar la aplicación:

\begin{itemize}
    \item Traducir pictogramas a lenguaje natural y viceversa.
    \item Cuadrícula para ejercicios de matemática.
    \item Poder añadir doble pauta, en vez de renglones de una línea para determinar el tamaño de letra del alumno.
    \item Poder recortar imágenes.
    \item Añadir encabezado con el nombre del centro educativo, asignatura y nombre del alumno.
    \item Añadir espacio para dibujar.
    \item Fórmulas con huecos que puedan ser rellenados por el alumno.
    \item Enumerar ejercicios automáticamente.
    \item Fuente de letra escolar.
\end{itemize}
