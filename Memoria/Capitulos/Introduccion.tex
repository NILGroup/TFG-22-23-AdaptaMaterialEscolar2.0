\chapter{Introducción}
\label{ch:introduccion}

En este capítulo se hace una introducción al Trabajo de Fin de Grado que va a ser presentado en este documento. Primero en la Sección \ref{cap:motivacio} se explicará la motivación que ha dado lugar al trabajo. A continuación, en la Sección \ref{cap:objetivos} se presentan los objetivos que se pretende alcanzar. Por último, la estructura del documento final se detalla en la Sección \ref{cap:estructura}.

\section{Motivación}\label{cap:motivacio}
La educación escolar tiene como objetivo promover el desarrollo de ciertas habilidades y el aprendizaje de varios contenidos necesarios para que los estudiantes se conviertan en miembros activos de la sociedad. Para ello, la escuela debe dar respuestas educativas que eviten la discriminación y promuevan la igualdad de oportunidades. Los docentes permiten lograr estos objetivos empleando recursos pedagógicos como el currículo educativo, en el cual se incluyen los planes de estudio, los fundamentos, la metodología y los programas para facilitar a los alumnos una formación integral y completa.

Además, en el curriculum escolar todos los alumnos tienen necesidades educativas comunes. Sin embargo, no todos los estudiantes se enfrentan al mismo con las mismas capacidades de aprendizaje, sino que cada alumno tiene necesidades individuales. La mayoría de estas se abordan a través de acciones simples: dar a los alumnos más tiempo para aprender determinados contenidos, diseñar actividades complementarias, etc.  Sin embargo, también existen necesidades individuales que no se pueden atacar por estos medios, y que precisan una serie de medidas didácticas especiales, diferentes de las normalmente requeridas para la mayoría de los estudiantes. Dichas necesidades se pueden satisfacer con las adaptaciones curriculares. Existen dos tipos de adaptaciones curriculares:
\begin{itemize}
    \item Adaptación no significativa: Adaptaciones en la metodología, las actividades, los tiempos, las técnicas e instrumentos de evaluación. No modifican los contenidos del curriculum.  
    \item Adaptación significativa: Ajustes significativos en el curriculum, es decir se eliminan apartados del curriculum oficial. 
\end{itemize}

Las adaptaciones curriculares no significativas para los Alumnos con Necesidades Educativas Especiales (ACNEE) deben ser realizadas por los profesores. Sin embargo, en general no se les facilita una herramienta para ello a pesar de ser un trabajo muy costoso, ya que requiere la adaptación personalizada de materiales, pruebas de evaluación, etc.

\section{Objetivos}\label{cap:objetivos}
La finalidad de este TFG es proporcionar una herramienta al profesorado que permita la adaptación curricular no significativa de los contenidos de las asignaturas de forma intuitiva, rápida y simple, con el objetivo de hacer materiales que se trabajen en el aula, adaptados a las distintas necesidades de los alumnos.

Para crear nuestra herramienta partiremos de la aplicación AdaptaMaterialEscolar 1.0  que permite la creación de diferentes tipos de ejercicios (por ejemplo, sopas de letras, rellenar espacios en blanco, etc). Analizaremos en detalle la herramienta creada, los requisitos cubiertos por ella y los que quedaron por incorporar, rediseñaremos tanto la interfaz de la aplicación como la arquitectura, seguiremos una metodología de Diseño Centrado en el Usuario (DCU) para encontrar cuales son las necesidades que quedan por cubrir y finalmente evaluaremos el resultado de nuestro TFG con usuarios finales.
 
En relación con los objetivos académicos, aspiramos a emplear los conocimientos adquiridos durante el Grado de Ingeniería del Software en un proyecto real y adquirir nuevos conocimientos.



\section{Estructura del documento}\label{cap:estructura}
La memoria se encuentra organizada en ocho capítulos. A continuación, se realiza un pequeño resumen de cada uno, exceptuando el actual.
\begin{itemize}
    \item \textbf{\hyperref[cap:introduction]{Capítulo 2}}: Se presenta la introducción en inglés.
    \item \textbf{\hyperref[cap:estadoDelArte]{Capítulo 3}}: Se presenta el estado del arte, en el que se define qué es la adaptación curricular y los tipos posibles. Además se incluyen las herramientas existentes y una mención a la aplicación de AdaptaMaterialEscolar 1.0.
    \item \textbf{\hyperref[cap:metodologia]{Capítulo 4}}: Se presenta la metodología usada junto a sus reglas, políticas y el tablero. También se encuentra explicado el plan de pruebas.
    \item \textbf{\hyperref[cap:AdaptaMaterialEscolar2.0]{Capítulo 5}}: En este capítulo se explica todo lo relacionado con la segunda versión de AdaptaMaterialEscolar, implementada de manera íntegra en este TFG.
    \item \textbf{\hyperref[cap:conclusiones]{Capítulo 6}}: En este capítulo se presentan las conclusiones y el trabajo futuro a realizar.
    \item \textbf{\hyperref[cap:conclusions]{Capítulo 7}}: Se presenta las conclusiones y el trabajo futuro a realizar en inglés.
    \item \textbf{\hyperref[cap:TrabajoIndividual]{Capítulo 8}}: Se presenta el trabajo individual realizado por cada integrante del grupo.
\end{itemize}