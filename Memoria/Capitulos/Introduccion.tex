\chapter{Introducción}
\label{ch:introduccion}

En este capítulo se explica la introducción del Trabajo de Fin de Grado que va ser presentado en este documento. Primero la Sección \ref{cap:motivacio}, se explicará la motivación que ha dado lugar al trabajo. A continuacción, en la Sección \ref{cap:objetivos}, el objetivo que se pretende alcanzar. Por último, la estuctura del proyecto final, en la Sección \ref{cap:estructura}.

\section{Motivación}\label{cap:motivacio}
La educación escolar tiene como objetivo promover el desarrollo de ciertas habilidades, y el aprendizaje de ciertos contenidos necesarios para que los estudiantes se conviertan en miembros activos de la sociedad. Para ello, la escuela debe dar respuestas educativas que eviten la discriminación y promuevan la igualdad de oportunidades.

En el currículo escolar, todos los alumnos tienen necesidades educativas comunes. Sin embargo, no todos los estudiantes se enfrentan con las mismas capacidades de aprendizaje, sino que cada alumno tiene necesidades individuales. La mayoría de estas se abordan a través de acciones simples: dar a los alumnos más tiempo para aprender determinados contenidos, diseñar actividades complementarias, etc.  Sin embargo, también existen necesidades individuales que no se pueden atacar por estos medios, lo que precisa una serie de medidas didácticas especiales, diferentes de las normalmente requeridas para la mayoría de los estudiantes. Dichas necesidades se pueden satisfacer con las adaptaciones curriculares. Existen dos tipos de adaptaciones curriculares:
\begin{itemize}
    \item Adaptación no significativa: Adaptaciones en la metodología, las actividades, los tiempos,
    las técnicas e instrumentos de evaluación. No modifican los contenidos del currículo.  
    \item Adaptación significativa: Ajustes significativos en el currículo. 
\end{itemize}
Las adaptaciones curriculares no significativas para los ACNEE (alumnos con necesidades educativas especiales) son realizados por los profesores, a diferencia de los ajustes curriculares significativos, pero no se les facilita una herramienta para ello. Por esa razón nació AdaptaMaterialEscolar1.0, una primera aproximación de la herramienta que se evaluó, de la cual hubo requisitos que no se llegaron a implementar, de los que se implementaron, se propusieron mejoras.

\section{Objetivos}\label{cap:objetivos}
La finalidad de este TFG es proporcionar una herramienta para el profesorado con el fin de adaptar los contenidos de las asignaturas de forma intuitiva, rápida y simple, con el objetivo de hacer unidades didácticas particulares que se adapten a las distintas necesidades de los alumnos.

Para poder ayudar al profesorado, partimos de la versión actual de AdaptaMaterialEscolar que permite la creación de diferentes tipos de ejercicios(por ejemplo, sopas de letras, rellenar espacios en blanco, etc). Tras la evaluación de la herramienta surgieron varias mejoras que vamos a afrontar en este TFG como objetivo específico. 


\section{Estructura del proyecto}\label{cap:estructura}
La momoria se encuentra organizada en X capítulos, a continuación se relaliza un pequeño resumen de cada uno:
\begin{itemize}
    \item \textbf{\hyperref[ch:introduccion]{Capítulo uno}}: se presenta la introducción incluyendo la motivación, los objetivos y la sección de estructura del proyecto.
    \item \textbf{\hyperref[cap:estadoDelArte]{Capítulo tres}}: se presenta el estado del arte, en el que se define qué es la adaptación curricular y los tipos que hay, además se incluye las herramientas exsistentes y una mención a la herrmienta AdaptaMaterialEscolar1.0.
\end{itemize}