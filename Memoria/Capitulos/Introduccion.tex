\chapter{Introducción}
\label{ch:introduccion}

En este capítulo se explica la introducción del Trabajo de Fin de Grado que va a ser presentado en este documento. Primero en la Sección \ref{cap:motivacio} se explicará la motivación que ha dado lugar al trabajo. A continuacción, en la Sección \ref{cap:objetivos} los objetivos que se pretende alcanzar. Por último, la estuctura del proyecto final se detalla en la Sección \ref{cap:estructura}.

\section{Motivación}\label{cap:motivacio}
La educación escolar tiene como objetivo promover el desarrollo de ciertas habilidades, y el aprendizaje de varios contenidos necesarios para que los estudiantes se conviertan en miembros activos de la sociedad. Para ello, la escuela debe dar respuestas educativas que eviten la discriminación y promuevan la igualdad de oportunidades.

En el currículo escolar, todos los alumnos tienen necesidades educativas comunes. Sin embargo, no todos los estudiantes se enfrentan con las mismas capacidades de aprendizaje, sino que cada alumno tiene necesidades individuales. La mayoría de estas se abordan a través de acciones simples: dar a los alumnos más tiempo para aprender determinados contenidos, diseñar actividades complementarias, etc.  Sin embargo, también existen necesidades individuales que no se pueden atacar por estos medios, lo que precisa una serie de medidas didácticas especiales, diferentes de las normalmente requeridas para la mayoría de los estudiantes. Dichas necesidades se pueden satisfacer con las adaptaciones curriculares. Existen dos tipos de adaptaciones curriculares:
\begin{itemize}
    \item Adaptación no significativa: Adaptaciones en la metodología, las actividades, los tiempos,
    las técnicas e instrumentos de evaluación. No modifican los contenidos del currículo.  
    \item Adaptación significativa: Ajustes significativos en el currículo, es decir se eliminan apartados del curriculum oficial. 
\end{itemize}
Las adaptaciones curriculares no significativas para los ACNEE (Alumnos con Necesidades Educativas Especiales) deben ser realizadas por los profesores, sin embargo, no se les facilita una herramienta para ello a pesar de ser un trabajo muy costoso, ya que requiere la adaptación personalizada de materiales, pruebas de evaluación, etc. Por esa razón nació AdaptaMaterialEscolar 1.0, una herramienta para facilitar la adaptación curricular no significativa y ahorrar tiempo a los docentes.

\section{Objetivos}\label{cap:objetivos}
La finalidad de este TFG es proporcionar una herramienta al profesorado que permite la adaptación curricular no significativa de los contenidos de las asignaturas de forma intuitiva, rápida y simple, con el objetivo de hacer materiales, que se trabajen en el aula, adaptados a las distintas necesidades de los alumnos.

Partimos de la aplicación AdaptaMaterialEscolar 1.0  que permite la creación de diferentes tipos de ejercicios (por ejemplo, sopas de letras, rellenar espacios en blanco, etc). Para poder ayudar al profesorado, estudiaremos las aplicaciones similares y las funcionalidades, rediseñaremos tanto la interfaz de la aplicación como la arquitectura, implementaremos las funcionalidades que les podrían ser útiles y por último, se evaluará la aplicación.

Además, seguiremos un Diseño Centrado en el
Usuario (DCU) con el fin de ajustar la aplicación a las necesidades de los usuarios finales, el profesorado.
 
En relación con los objetivos académicos, aspiramos a emplear los conocimientos adquiridos durante el Grado de Ingeniería del Software en un proyecto real.



\section{Estructura del proyecto}\label{cap:estructura}
La memoria se encuentra organizada en 8 capítulos. a continuación, se relaliza un pequeño resumen de cada uno:
\begin{itemize}
    \item \textbf{\hyperref[ch:introduccion]{Capítulo uno}}: Se presenta la introducción incluyendo la motivación, los objetivos y la sección de estructura del proyecto.
    \item \textbf{\hyperref[cap:introduction]{Capítulo dos}}: Se presenta la introducción en inglés. 
    \item \textbf{\hyperref[cap:estadoDelArte]{Capítulo tres}}: Se presenta el estado del arte, en el que se define qué es la adaptación curricular y los tipos posibles, además se incluye las herramientas existentes y una mención a la aplicación actual de  AdaptaMaterialEscolar1.0.
    \item \textbf{\hyperref[cap:metodologia]{Capítulo cuatro}}: Se presenta la metodología usada junto a sus reglas, políticas y el tablero. También se encuentra explicado el plan de pruebas.
    \item \textbf{\hyperref[cap:AdaptaMaterialEscolar2.0]{Capítulo cinco}}: En este capítulo se explica todo lo relacionado con la segunda versión de AdaptaMaterialEscolar.
    \item \textbf{\hyperref[cap:conclusiones]{Capítulo seis}}: En este capítulo se presentan las conclusiones y el trabajo futuro a realizar.
    \item \textbf{\hyperref[cap:conclusions]{Capítulo siete}}: Se presenta las conclusiones y el trabajo futuro a realizar en inglés.
    \item \textbf{\hyperref[cap:trabajo_individual]{Capítulo ocho}}: Se presenta el trabajo individual realizado por cada integrante del grupo.
\end{itemize}