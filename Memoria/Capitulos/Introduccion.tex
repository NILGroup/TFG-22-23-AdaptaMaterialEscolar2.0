\chapter{Introducción}
\label{cap:introduccion}

\section{Motivación}
La educación escolar tiene como objetivo promover el desarrollo de ciertas habilidades, y el aprendizaje de ciertos contenidos necesarios para que los estudiantes se conviertan en miembros activos de la sociedad. Para ello, la escuela debe dar respuestas educativas que eviten la discriminación y promuevan la igualdad de oportunidades.

En el currículo escolar, todos los alumnos tienen necesidades educativas comunes. Sin embargo, no todos los estudiantes se enfrentan con las mismas capacidades de aprendizaje, sino que cada alumno tiene necesidades individuales. La mayoría de estas se abordan a través de acciones simples: dar a los alumnos más tiempo para aprender determinados contenidos, diseñar actividades, etc.  Sin embargo, también existen necesidades individuales que no se pueden atacar por estos medios, lo que precisa una serie de medidas didácticas especiales, diferentes de las normalmente requeridas para la mayoría de los estudiantes. Dichas necesidades se pueden satisfacer con las adaptaciones curriculares. Estos ajustes curriculares no significativos para ACNEE (alumnos con necesidades educativas especiales) son realizadas por los profesores, pero no se les facilita una herramienta para ello, por esa razón nació AdaptaMaterialEscolar1.0.

\section{Objetivos}
La finalidad de este TFG es proporcionar una herramienta para el profesorado con el fin de adaptar los contenidos de las asignaturas de forma intuitiva, rápida y simple, con el objetivo de hacer unidades didácticas que se adapten a las necesidades de los alumnos.

Para poder ayudar al profesorado, partimos de la versión anterior, AdaptaMaterialEscolar1.0 que incluye la creación de diferentes tipos de actividades para adaptar el material (por ejemplo, sopas de letras, rellenar espacios en blanco, etc). Después de estudiar dicha herramienta nos encontramos con que faltan algunas funcionalidades tales como un generador de resúmenes, un manual de usuario, una opción para poder cambiar el PDF, crear esquemas, realizar ejercicios de flechas, etc. Tras la evaluación por parte de los profesores, enunciaron una serie de propuestas de mejora como exportar el documento a formato Word para hacer modificaciones, añadir ejercicios con espacio para dibujar, enumerar ejercicios de forma automática, etc. Con relación a los problemas encontramos que el diseño de la aplicación no se adapta a diferentes tamaños de pantalla. 

En conclusión, nos proponemos solventar los fallos, además de añadir las funcionalidades que faltan para proporcionar una herramienta más completa, con el fin de ayudar al profesorado pero también al alumno. 

