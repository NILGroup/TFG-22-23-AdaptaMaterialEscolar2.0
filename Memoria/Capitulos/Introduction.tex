\chapter{Introduction}
\label{cap:introduction}

In this chapter, an introduction to the Degree Final Project, which will be presented in this document, is provided. First, in Section \ref{cap:motivacio}, the motivation behind the work will be explained. The objectives to be achieved are presented in the Section \ref{cap:objetivos}. Finally, the structure of the final project is detailed in Section \ref{cap:estructura}.


\section{Motivation}\label{cap:motivation}
The objective of school education is to promote the development of certain skills and the learning of various contents necessary for students to become active members of society. To achieve this, schools must provide educational responses that avoid discrimination and promote equal opportunities. Teachers allow these objectives to be achieved by using pedagogical resources such as the educational curriculum, which includes study plans, foundations, methodology, and programs to provide students with a comprehensive and complete education.

Furthermore, all students have common educational needs in the school curriculum. However, not all students face the same learning capacities, and each student has individual needs. Most of these are addressed through simple actions such as giving students more time to learn certain contents, designing complementary activities, etc. However, there are also individual needs that cannot be addressed by these means and require a series of special didactic measures that are different from those normally required for the majority of students. These needs can be met with curricular adaptations. There are two types of curricular adaptations:
\begin{itemize}
\item Non-significant adaptation: Adaptations in methodology, activities, timing, evaluation techniques, and instruments. They do not modify the contents of the curriculum.
\item Significant adaptation: Significant adjustments to the curriculum, that is, official curriculum sections are eliminated.
\end{itemize}
Non-significant curricular adaptations for students with Special Educational Needs (SEN) should be made by teachers. However, they are not provided with a tool for this, despite it being a very costly job, since it requires the personalized adaptation of materials, evaluation tests, etc.

\section{Objectives}\label{cap:objetives}
The purpose of this DFP is to provide teachers with a tool that allows non-significant curricular adaptation of the contents of the subjects in an intuitive, fast, and simple way, with the aim of making materials that are worked on in the classroom, adapted to the different needs of the students.

To create our tool, we will start from the AdaptaMaterialEscolar 1.0 application, which allows the creation of different types of exercises (e.g., word search, fill in the blanks, etc.). We will analyze in detail the created tool, the requirements covered by it and those that remained to be included. We will redesign both the application interface and the architecture, follow a  User-Centered Design (UCD) methodology to find out which needs are still to be covered and finally evaluate the result of our DFP with end-users.

In relation to academic objectives, we aspire to use the knowledge acquired during the Software Engineering Degree in a real project and acquire new knowledge.

\section{Project structure}\label{cap:Projectstructure}
The report is organized into eight chapters. Below is a brief summary of each chapter, excluding the previous and current one.
\begin{itemize}
\item \textbf{\hyperref[cap:estadoDelArte]{Chapter 3}}: The state of the art is presented, in which curricular adaptation and its possible types are defined. In addition, existing tools are included along with a mention of the application AdaptaMaterialEscolar 1.0.
\item \textbf{\hyperref[cap:metodologia]{Chapter 4}}: The methodology used along with its rules, policies, and board are presented. The Test Plan is also explained.
\item \textbf{\hyperref[cap:AdaptaMaterialEscolar2.0]{Chapter 5}}: This chapter explains everything related to the second version of AdaptaMaterialEscolar.
\item \textbf{\hyperref[cap:conclusiones]{Chapter 6}}: The conclusions and future work to be carried out are presented in this chapter.
\item \textbf{\hyperref[cap:conclusions]{Chapter 7}}: The conclusions and future work to be carried out are presented in English.
\item \textbf{\hyperref[cap:TrabajoIndividual]{Chapter 8}}: The individual work carried out by each member of the group is presented.
\end{itemize}