\chapter{Conclusions and Future Work}
\label{cap:conclusions}


In Section \ref{sec:conclusiones}, the conclusions reached after completing the project are explained, and in Section \ref{sec:TrabajoFuturo}, possible improvements that could be implemented in the application are described.

\section{Conclusions}
\label{sec:conclusiones}
The main objective of this Degree Final Project (TFG) was to develop a web application that facilitates non-significant curricular adaptation for teachers. To achieve this general objective, we set several specific objectives. Next we will see if we have achieved each of these specific objectives:

\begin{itemize}
    \item Analyze AdaptaMaterial 1.0 to identify the requirements that were not covered and the improvements that could be made. As shown in Section \ref{cap:requisitos}, we conducted this analysis and identified the requirements for our application.
    \item Redesign the application. This objective has also been achieved as we completely redesigned the interface, as presented in Section \ref{disenyoDeLaAplicacion}. Additionally, we refactored the application: we transitioned from a serverless architecture to a client-server architecture, migrated from class-based components to functional components, updated the React router to the latest available version, decided not to use Redux or Sass, and started using Tailwind CSS. Lastly, due to the expiration of the CKEditor license used in AdaptaMaterialEscolar 1.0 and the impossibility of renewing it, we implemented the text editor using Slate.
    \item Improvement of existing features and addition of new features. Regarding the development and definitions features present in the previous version (AdaptaMaterialEscolar 1.0), we have incorporated the ability to choose the type of ruling (double, single, grid), and we have also added the option to select a school font type. Additionally, we have introduced several new features, including: generating a summary, exporting the working document to Word and PDF, importing to Word, pictotranslator, content matching exercises, color legend, exercises with space for drawing, and mathematical formulas. All of these features have been implemented except for the import and export to Word.
    \item We conducted an evaluation with end users who provided feedback through a Google Form survey. The survey consisted of three parts: creating an exam model, preparing adapted notes, and a general evaluation of the application. Each task was accompanied by an explanation and an image, followed by three evaluation questions. After the first two parts, a ten-question questionnaire on ease of use was conducted. Open-ended responses were also requested to gather opinions about the tool.
    
    The results obtained demonstrate that we have exceeded expectations, achieving a notable score of 78.5 on the System Usability Scale (SUS). Furthermore, analyzing Figure \ref{fig:graficaComparativaEjerciciosApuntes}, we can observe that users were more satisfied and comfortable using the features for generating exercises compared to the features for creating notes.
\end{itemize}



In addition, at the beginning of the Degree Final Project (TFG), we set ourselves two academic objectives that we have also achieved:
\begin{itemize}
    \item Apply the knowledge acquired during our degree to this project.
    \begin{itemize}
        \item \textbf{Software Project Management}: This subject focuses on the planning, organization, monitoring, and control of all aspects of a software project, from conception to the final product delivery. In this course, we have learned how to manage resources and time to ensure that projects are completed within established deadlines. Specifically, in this project, we have used the Kanban methodology, which we learned in Software Project Management. This allowed us to create a solid project plan and regularly track progress to ensure the project stays on track.
        \item \textbf{Web Applications}: Teaches how to design and develop efficient and scalable web applications. It covers a wide variety of technologies, from user interface design and creation to database management and application security. Specifically, in this project, we have utilized the knowledge acquired in HTML, CSS, JavaScript, and Node.js.
        \item \textbf{Software Engineering, Software Modeling}: Software Engineering and Software Modeling subjects are important for the development of high-quality software. The former focuses on the principles, practices, and techniques necessary to create efficient and functional software, while the latter focuses on creating accurate and detailed models before implementation to reduce errors and enable faster and more effective implementation. Both subjects cover essential techniques and skills for the software development process. Specifically, in this project, we have applied the acquired knowledge regarding error management and design patterns.
        \item \textbf{Data Structures}: Understanding data structures and data manipulation techniques is essential for the development of high-quality and efficient software. This subject provides a solid foundation for building more effective and scalable software solutions in the future. Specifically, we have applied the acquired knowledge to better understand the structure of Slate.
        \item \textbf{Ethics, Legislation, and Profession}: This subject focuses on the ethical and legal aspects of software engineering, such as data privacy, intellectual property, social and professional responsibility, and software security. It also covers relevant laws and regulations, such as the Personal Data Protection Law and the Intellectual Property Law. Specifically, we have applied the acquired knowledge to understand how to use and manage third-party code, as well as managing the license for our project.
        \item \textbf{Systems and Networks Administration}: This subject focuses on the administration of operating systems, including server and client installation, configuration, and maintenance. It also covers network administration, including router, switch, and firewall configuration, IP address management, and network traffic monitoring. Specifically, we have applied the acquired knowledge to set up the server where the application has been hosted.
    \end{itemize}

    \item Acquire new knowledge.

    During the development of the AdaptaMaterialEscolar 2.0 application, we have acquired new knowledge about the use of technologies such as React, widely used in the professional world. Additionally, we have effectively learned and applied Tailwind CSS and Slate, tools that have contributed to a more efficient and flexible development experience. We have also worked with end-users, identifying and meeting their needs.
\end{itemize}


