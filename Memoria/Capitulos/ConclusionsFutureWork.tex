\chapter{Conclusions and Future Work}
\label{cap:conclusions}


In Section \ref{sec:conclusions}, the conclusions reached after completing the project are explained, and in Section \ref{sec:FutureWork}, possible improvements that could be implemented in the application are described.

\section{Conclusions}
\label{sec:conclusions}
The main objective of this Degree Final Project (TFG) was to develop a web application that facilitates non-significant curricular adaptation for teachers. To achieve this general objective, we set several specific objectives. Next we will see if we have achieved each of these specific objectives:

\begin{itemize}
    \item Analyze AdaptaMaterial 1.0 to identify the requirements that were not covered and the improvements that could be made. As shown in Section \ref{cap:requisitos}, we conducted this analysis and identified the requirements for our application.
    \item Redesigning the application. This objective has also been achieved as we have done a complete redesign of the interface as presented in Section \ref{disenyoDeLaAplicacion}. Additionally, we have refactored it by transitioning from a serverless architecture to a client-server one, migrating from class-based components to functional components, updating the React router to the latest available version, deciding not to use Redux or Sass, and starting to use Tailwind CSS. Finally, due to the expiration of the CKEditor license used in AdaptaMaterialEscolar 1.0 and the inability to renew it, Slate was used to implement the editor.
    \item Improvement of existing features and addition of new features. In relation to the development and definitions features present in the previous version, we have incorporated the ability to choose the type of ruling (double, single, grid) and also added the option to select a school font type. Furthermore, we have introduced several new features, including generating a summary, exporting the working document to PDF, adding a pictotranslator, matching content exercises, color legend, drawing space exercises, and mathematical formulas.
    \item Additionally, as presented in Chapter 7, to comply with User-Centered Design, we conducted an evaluation of the tool with end users.
\end{itemize}



In addition, at the beginning of the Degree Final Project (TFG), we set ourselves two academic objectives that we have also achieved:
\begin{itemize}
    \item Applying the knowledge acquired during the degree. Regarding this objective, we can affirm that in this project, we have applied the knowledge acquired in the following subjects of the degree:
    \begin{itemize}
        \item \textbf{Software Project Management}: Specifically in this project, we have used the Kanban methodology, which we learned in Software Project Management. This allowed us to create a solid project plan and regularly track progress to ensure that the project progresses as planned.
        \item \textbf{Web Applications}: It teaches how to design and develop efficient and scalable web applications. It also covers a wide variety of technologies, from designing and creating user interfaces to database management and application security. Specifically in this project, we have used the knowledge acquired in \textit{HTML}, \textit{CSS}, \textit{JavaScript}, and \textit{Node.js}.
        \item \textbf{Software Engineering, Software Modeling}: Both subjects cover essential techniques and skills for the software development process. Specifically in this project, we have applied the knowledge acquired in error handling and design patterns.
        \item \textbf{Data Structures}: This subject provides a solid foundation for building more effective and scalable software solutions in the future. Specifically, we have applied the knowledge acquired to better understand the structure of Slate.
        \item \textbf{Ethics, Legislation, and Profession}: It focuses on the ethical and legal aspects of software engineering, such as data privacy, intellectual property, social and professional responsibility, and software security. Relevant laws and regulations, such as the Personal Data Protection Law and the Intellectual Property Law, are also taught. Specifically, we have applied the knowledge acquired to know how to use and manage third-party code, as well as manage the license of our project.
        \item \textbf{Systems and Network Administration}: It focuses on the administration of operating systems, including the installation, configuration, and maintenance of servers and clients. Network administration is also taught, including router, switch, and firewall configuration, IP address management, and network traffic monitoring. Specifically, we have applied the knowledge acquired to set up the server on which the application has been hosted.
    \end{itemize}
        
    \item  Acquire new knowledge. During the development of the AdaptaMaterialEscolar 2.0 application, we have acquired new knowledge about the use of technologies such as React, widely used in the professional world. We have also effectively learned and applied Tailwind CSS and Slate, tools that have contributed to a more efficient and flexible development experience. Furthermore, we have worked with end users, identifying and meeting their needs.
\end{itemize}


\section{Future Work}
\label{sec:FutureWork}
After developing the project and accomplishing most of the objectives set initially, it is inevitable that some tasks remain pending for possible future work.

From a requirements perspective, in Section \ref{cap:requisitos}, we already discussed the functionalities that were discarded due to lack of information.

\begin{itemize}
    \item Add image search by keyword.
    \item Substitute a word with an image.
    \item Create an image cropping tool for the original text.
    \item Create tables to organize the syllabus and/or activities, selecting content.
    \item Create diagrams.
    \item Exercises to fill in the blanks in tables and diagrams.
\end{itemize}

On the other hand, there have been certain requirements that have not been carried out due to the priority established among them. Specifically, these requirements are:

\begin{itemize}
    \item Import to Word: We have not implemented this because it would require changing the behavior of the elements handled by Slate. We considered it too complex, and there were other functionalities with equal or higher priority.
    \item Export to Word: The implementation of this functionality has not been carried out due to its similarity to the import to Word function. Just like in the case of importing to Word, we considered that this implementation would be too complex, and there were other functionalities with equal or higher priority.
    \item Help section: Although we consider it important, it has not been implemented as we prioritized the quality of the editor and the adaptations, as well as the quantity of adaptations.
    \item General configuration: To implement the general configuration, it would be necessary to use browser cookies to store the information, significantly increasing the complexity of the application. For each adaptation, information about all its possible configuration options would need to be stored. Additionally, it would involve a significant amount of data management since each adaptation can have multiple configuration options, requiring the storage and management of a large amount of related data. Therefore, we believed it would take a long time to make it work correctly, and we preferred to implement more adaptations and improve the functionality of both the editor and the adaptations.
\end{itemize}

The evaluation of the application has generated improvement ideas and new requirements:

\begin{itemize}
    \item Ability to rearrange exercises: The evaluators missed being able to change the order of the exercises.
    \item Allow undo and redo actions for all operations to make the user feel in control of the application and explore it without fear of making mistakes. This new functionality would also enable the user to easily recover from errors.
    \item Add help to all the features.
    \item Allow the selection of specific words to appear in the summary.
    \item Improve wording regarding punctuation and connectors to provide cohesion in the text.
    \item When translating text into pictograms, articles should be omitted as they are complex to represent through drawings.
    \item Improve exercise editing in the working document.
    \item Allow the generation of non-numbered exercises.
    \item Enhance advanced text editing features.
    \item Automatically assign colors to text based on selected categories in the color legend.
    \item Improve the functionality of math exercises because evaluators found it confusing.
\end{itemize}