\chapter{Evaluación}
\label{cap:evaluacion}
% TODO: Cambiar
En este capítulo explicaremos la obtención de requisitos y cómo se han clasificado en la Sección \ref{cap:requisitos}. También se describirá el diseño realizado por cada integrante de la aplicación para la iteración competitiva y el diseño final de las funcionalidades en la Sección \ref{disenyoDeLaAplicacion}. Por último, en la Sección \ref{sec:implmentaction} se expondrá cómo se ha implementado la aplicación.

\section{Introduccion}\label{sec:introEvaluacion}
Una vez terminado el desarrollo de AdaptaMaterialEscolar 2.0, llevamos a cabo una evaluación para cumplir con nuestro objetivo principal: confirmar si nuestro proyecto resultaba útil para los docentes que necesitaban realizar adaptaciones curriculares para sus estudiantes en cualquier etapa académica.

Con el propósito de realizar una valoración detallada, nos enfocamos en evaluar varios aspectos de nuestro proyecto. Estos incluían el diseño de la interfaz, la usabilidad del sistema en su totalidad, cada funcionalidad de manera independiente y las posibles mejoras que podrían implementarse en el futuro.

Nuestro objetivo era analizar y obtener retroalimentación precisa sobre la calidad del diseño de la interfaz, asegurándonos de que fuera intuitiva y fácil de navegar para los usuarios. Asimismo, examinamos cada funcionalidad por separado, buscando identificar cualquier problema o área de mejora.

Con el propósito de garantizar un enfoque continuo en el crecimiento del proyecto, también consideramos las posibles mejoras que podrían implementarse en futuras versiones.

\section{Diseño de la evaluación}\label{sec:disenyoEvaluacion}
Con el fin de permitir a los usuarios finales evaluar nuestra aplicación, creamos un examen de prueba y una hoja de apuntes sobre las asignaturas de conocimiento del medio y matemáticas.
Para facilitar la evaluación y el análisis de los resultados de nuestra aplicación web, creamos una encuesta\footnote{\url{https://docs.google.com/forms/d/e/1FAIpQLSdg7mGUfGBW5LoGIWulEUq-vhloL5rlkU_aIZCUzpqiJp164A/viewform?usp=sf_link}} en Google Forms dirigida a los usuarios finales. Durante la evaluación, los usuarios experimentaban directamente con las diversas funcionalidades, tratando de replicar el examen y la hoja de apuntes proporcionados.

La evaluación constaba de 3 partes y estaba diseñada para que se realizase en 45 minutos aproximadamente. Las partes de la evaluación eran las siguientes: la creación de un modelo de examen, la elaboración de unos apuntes adaptados y una evaluación general de la aplicación.

En las dos primeras partes tenían que realizar distintas tareas. Para realizar cada tarea, se proporcionaba una explicación detallada de cómo realizarla junto con una imagen de cómo quedaría el documento tras realizarla. Al terminar cada tarea se hacían 3 preguntas para evaluar dicha tarea.

Una vez completadas las dos primeras partes de la evaluación, se presentó un cuestionario que constaba de 10 preguntas y que nos permitió evaluar la facilidad de uso de la aplicación. Para terminar se realizaron una serie de preguntas de respuesta abierta para conocer la opinión de los usuarios sobre la herramienta.

Tras la creación de cada parte de la evaluación (examen y apuntes), el usuario debía exportar a PDF el documento y subirlo a una carpeta de Google Drive.

Estructuramos las preguntas del cuestionario de acuerdo con diferentes aspectos y fueron las siguientes:

\begin{itemize}
    \item \textbf{Preguntas generales}. Datos demográficos del usuario.
          \begin{itemize}
              \item ¿Cuántos años tienes?
              \item ¿Eres docente?
          \end{itemize}
    \item \textbf{Preguntas para no docentes}.
          \begin{itemize}
              \item ¿Eres estudiante de magisterio?
          \end{itemize}
    \item \textbf{Preguntas para docentes}. Recogen el nivel académico en el que imparten clase, si han realizado alguna adaptación curricular y, en caso afirmativo, cuantas veces.
          \begin{itemize}
              \item ¿Podrías decirnos en qué nivel del sistema educativo eres docente?
              \item ¿Has tenido que hacer alguna adaptación curricular no significativa?
              \item ¿Cuántas veces?
          \end{itemize}
    \item \textbf{Preguntas sobre funcionalidades}. Obtienen información sobre la
          dificultad de uso de cada funcionalidad y recopilar nuevas adaptaciones
          o mejoras que se desean para la aplicación.
          \begin{itemize}
              \item En general, estoy  satisfecho o satisfecha con la facilidad de completar esta tarea.
              \item En general estoy  satisfecho o satisfecha con la cantidad de tiempo que me ha llevado completar esta tarea.
              \item Estoy  satisfecho o satisfecha con la respuesta de la aplicación al realizar las acciones, sé lo que pasa en todo momento.
              \item En caso de la funcionalidad de pictotraductor también se han realizado las siguientes preguntas: ¿La traducción resultante te parece correcta? y ¿Qué cuestiones crees que son mejorables en la traducción o que no son correctas?
              \item En caso de la funcionalidad de generar resumen también se han realizado las siguientes preguntas: ¿El resumen resultante te parece correcto? y ¿Qué cuestiones crees que son mejorables en el resumen o que no son correctas?
          \end{itemize}
    \item \textbf{Preguntas sobre usabilidad}. El cuestionario utilizado para evaluar la usabilidad de un sistema es conocido como Escala de Usabilidad de un Sistema\footnote{\url{https://uxpanol.com/teoria/sistema-de-escalas-de-usabilidad-que-es-y-para-que-sirve/}}, (SUS, por sus siglas en inglés) y es ampliamente utilizado en este campo. El cuestionario consta de las siguientes preguntas:
          \begin{itemize}
              \item Creo que usaría esta aplicación frecuentemente.
              \item Encontré la aplicación innecesariamente compleja.
              \item Creo que la aplicación es fácil de usar.
              \item Creo que necesitaría la ayuda de una persona con conocimientos técnicos para usar la aplicación.
              \item Las funciones de la aplicación están bien integradas.
              \item Creo que la aplicación es muy confusa.
              \item Creo que la mayoría de la gente aprendería a usar la aplicación muy rápidamente.
              \item Encuentro la aplicación muy complicada de utilizar.
              \item Me siento confiado o confiada al utilizar la aplicación.
              \item Necesito aprender muchas cosas antes de poder utilizar la aplicación.
          \end{itemize}
    \item \textbf{Preguntas generales sobre la aplicación}. Recopilan la opinión de los usuarios sobre la aplicación:
          \begin{itemize}
              \item ¿Qué te ha parecido la aplicación?
              \item ¿Qué es lo que más te ha gustado?
              \item ¿Qué es lo que menos te ha gustado?
              \item ¿Echas de menos alguna funcionalidad?
              \item ¿Te sobra alguna funcionalidad?
              \item ¿Algo más que quieras añadir?
          \end{itemize}
\end{itemize}

En las preguntas en las que se requería asignar una puntuación como respuesta, utilizamos una escala Likert de 5 puntos. Esta escala iba del 1 al 5, donde el valor más bajo, 1, indicaba que el usuario estaba "Muy en desacuerdo" con la afirmación de la pregunta, y el valor más alto, 5, indicaba que el usuario estaba "Muy de acuerdo". Todos los aspectos de la encuesta, excepto la sección de usabilidad, fueron diseñados específicamente por nosotros para obtener información que consideramos relevante para el proyecto.