\chapter{Conclusiones y Trabajo Futuro}
\label{cap:conclusiones}

En la Sección \ref{sec:conclusiones} se explican las conclusiones a las que se ha llegado tras realizar el proyecto y en la Sección \ref{sec:TrabajoFuturo} se describen las posibles mejoras que se podrían llevar a cabo en la aplicación.

\section{Conclusiones}
\label{sec:conclusiones}
En este Trabajo de Fin de Grado (TFG), se planteó como objetivo principal el desarrollo de una aplicación web que facilite la adaptación curricular no significativa para los docentes. Para lograr este objetivo general nos propusimos varios objetivos más específicos. A continuación, vamos a ir viendo si hemos cumplido con cada uno de estos objetivos específicos:

\begin{itemize}
    \item Analizar AdaptaMaterial 1.0 para ver que requisitos se quedaron sin cubrir y que mejoras se podían realizar. Tal y como mostramos en la Sección \ref{cap:requisitos} realizamos este análisis y con el sacamos los requisitos de nuestra aplicación.
    \item Rediseñar la aplicación. Este objetivo también se ha cumplido ya que realizamos un rediseño completo de la interfaz tal y como hemos presentado en la Sección \ref{disenyoDeLaAplicacion} y, además, hemos realizado una refactorización de esta: hemos pasado de una arquitectura serverless a una cliente-servidor, hemos migrado de componentes basados en clases a componentes funcionales, hemos actualizado el router de React a la última versión disponible, hemos decidido no utilizar Redux ni Sass y se empezó a utilizar Tailwind CSS. Por último, debido a que la licencia de CKEditor que se utilizaba en AdaptaMaterialEscolar 1.0 expiró y no fue posible renovarla, se utilizó Slate para implementar el editor.
    \item Mejora de las funcionalidades existentes y adición de nuevas funcionalidades. En relación con las funcionalidades de desarrollo y definiciones presentes en la versión anterior, hemos incorporado la capacidad de elegir el tipo de pauta (doble, simple, cuadrícula) y también se ha agregado la opción de seleccionar el tipo de fuente escolar. Además, hemos introducido varias nuevas funcionalidades, que incluyen: generar un resumen, exportar el documento de trabajo a PDF, añadir un pictotraductor, ejercicios de relacionar contenido, leyenda de colores, ejercicios con espacio para dibujar y fórmulas matemáticas.
    \item Además, tal y como hemos presentado en el capítulo 7, para cumplir con el Diseño Centrado en el Usuario realizamos una evaluación de la herramienta con usuarios finales.
\end{itemize}

 Además, al comienzo del TFG nos planteamos dos objetivos académicos que también hemos alcanzado:
 \begin{itemize}
    \item Aplicar los conocimientos adquiridos durante el grado. Respecto a este objetivo, podemos afirmar que en este proyecto hemos aplicado los conocimientos adquiridos en las siguientes asignaturas del grado:
    \begin{itemize}
        \item \textbf{Gestión de proyectos Software}: En concreto en este proyecto hemos utilizado la metodología Kanban, la cual hemos aprendido en Gestión de proyectos Software, lo que nos ha permitido, crear un plan de proyecto sólido y hacer un seguimiento regular del progreso para garantizar que el proyecto avance según lo previsto.
        \item \textbf{Aplicaciones Web}: Enseña a cómo diseñar y desarrollar aplicaciones web eficientes y escalables. Además, cubre una amplia variedad de tecnologías, desde el diseño y la creación de interfaces de usuario hasta la gestión de bases de datos y la seguridad de aplicaciones. En concreto en este proyecto hemos utilizado los conocimientos adquiridos sobre \textit{HTML}, \textit{CSS}, \textit{JavaScript} y \textit{Node.js}.
        \item \textbf{Ingeniería del Software, Modelado de Software}: En ambas asignaturas se cubren técnicas y habilidades esenciales para el proceso de desarrollo de software. En concreto en este proyecto hemos aplicado los conocimientos adquiridos sobre la gestión de errores y los patrones de diseño.
        \item \textbf{Estructura de Datos}: Esta asignatura proporciona una base sólida para construir soluciones de software más efectivas y escalables en el futuro. En concreto hemos aplicado los conocimientos adquiridos para entender mejor la estructura de Slate.
        \item \textbf{Ética, legislación y profesión}: Se centra sobre los aspectos éticos y legales de la ingeniería de software, como la privacidad de los datos, la propiedad intelectual, la responsabilidad social y profesional, y la seguridad del software. También se enseñan las leyes y regulaciones relevantes, como la Ley de Protección de Datos Personales y la Ley de Propiedad Intelectual. En concreto hemos aplicado los conocimientos adquiridos para saber cómo usar y gestionar código de terceros, así como la gestión de la licencia de nuestro proyecto.
        \item \textbf{Administración de Sistemas y Redes}: Se centra en la administración de sistemas operativos, incluyendo la instalación, configuración y mantenimiento de servidores y clientes. También se enseña la administración de redes, incluyendo la configuración de routers, switches y firewalls, la gestión de direcciones IP y el monitoreo del tráfico de la red. En concreto hemos aplicado los conocimientos adquiridos para montar el servidor en el que se ha alojado la aplicación.
    \end{itemize}
    \item Adquirir nuevos conocimientos. Durante el desarrollo de la aplicación AdaptaMaterialEscolar 2.0, hemos adquirido nuevos conocimientos sobre el uso de tecnologías, como React, ampliamente utilizada en el mundo laboral. Además, hemos aprendido y aplicado eficazmente Tailwind CSS y Slate, herramientas que han contribuido a una experiencia de desarrollo más eficiente y flexible. También hemos trabajado con los usuarios finales, identificando y satisfaciendo sus necesidades.
 \end{itemize}




\section{Trabajo Futuro}
\label{sec:TrabajoFuturo}
Después de desarrollar el proyecto y cumplir con la mayoría de los objetivos fijados al principio del trabajo, es inevitable que queden algunas tareas pendientes para posible trabajo futuro.

Desde el punto de vista de los requisitos, ya en la Sección \ref{cap:requisitos} se comentaron las funcionalidades descartadas por falta de información.

\begin{itemize}
    \item Añadir imágenes buscando una palabra.
    \item Sustituir una palabra por una imagen.
    \item Crear una herramienta de recorte de imágenes para el texto original.
    \item Crear tablas que organicen el temario y/o las actividades, seleccionando contenido.
    \item Crear esquemas.
    \item Ejercicios de completar los espacios en blanco en tablas y esquemas.
\end{itemize}

Por otro lado, ha habido ciertos requisitos que no se han llevado a cabo por la prioridad establecida entre ellos. Específicamente, estos requisitos son:
\begin{itemize}
    \item Importar a Word: No lo hemos realizado porque requeriría cambiar el comportamiento de los elementos tratados por Slate. Consideramos que era muy complejo y había otras funcionalidades con igual o mayor prioridad.
    \item Exportar a Word: La implementación de esta funcionalidad no se ha llevado a cabo debido a su similitud con la función de importar a Word. Al igual que en el caso de importar a Word, consideramos que esta implementación sería demasiado compleja, y además había otras funcionalidades con igual o mayor prioridad.
    \item Sección de ayuda: Aunque consideramos que es importante, no se ha realizado ya que hemos priorizado la calidad del editor y de las adaptaciones, además de la cantidad de adaptaciones.
    \item Configuración general: Para implementar la configuración general se necesitaría utilizar cookies de navegador para guardar la información aumentando considerablemente la complejidad de la aplicación. Para cada adaptación habría que guardar información sobre todas sus posibles opciones de configuración. Además, que implicaría un gran manejo de la información, dado que cada adaptación puede tener múltiples opciones de configuración, se requeriría almacenar y gestionar una gran cantidad de datos relacionados. Por lo que consideramos que nos iba a llevar mucho tiempo conseguir que funcionara correctamente y preferimos implementar más adaptaciones y mejorar el funcionamiento tanto del editor como de las adaptaciones.
\end{itemize}

Aunque el proyecto despertó más interés del esperado entre el profesorado, el número de evaluaciones obtenidas fue limitado. Para obtener conclusiones más significativas, se debería contactar con más docentes para recabar sus evaluaciones y propuestas de mejora futura para la aplicación. Las ideas de mejora y nuevos requisitos obtenidos como resultado de la evaluación son los siguientes:

\begin{itemize}
    \item Posibilidad de reorganizar los ejercicios. Los evaluadores echaron de menos poder cambiar el orden de los ejercicios.
    \item Permitir deshacer y rehacer cualquier acción para de este modo hacer sentir al usuario en control de la aplicación y que pueda explorarla sin miedo a equivocarse. Esta nueva funcionalidad además permitiría al usuario recuperarse fácilmente de los errores.
    \item Añadir ayuda en todas las funcionalidades de la aplicación.
    \item Permitir la selección de palabras específicas para que aparezcan obligatoriamente en el resumen.
    \item Mejorar la redacción en cuanto a signos de puntuación y conectores para dar cohesión al texto.
    \item Al traducir un texto a pictogramas se debería omitir los artículos ya que son complejos de entender mediante dibujos.
    \item Mejorar la edición de los ejercicios en el documento de trabajo.
    \item Posibilidad de generar ejercicios no numerados.
    \item Mejorar las funcionales de edición de texto avanzado.
    \item Permitir asignar colores automáticamente al texto según las categorías seleccionadas en la leyenda de colores.
    \item Mejorar la funcionalidad de ejercicios de matemáticas ya que los evaluadores la consideran confusa.
\end{itemize}