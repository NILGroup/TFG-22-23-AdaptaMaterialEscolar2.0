\chapter{AdaptaMaterialEscolar 2.0}
\label{cap:AdaptaMaterialEscolar2.0}
En este capítulo explicaremos la obtención de requisitos y su priorización en la Sección \ref{cap:requisitos}. También se describirá la iteración competitiva para el diseño de la aplicación en la Sección \ref{disenyoDeLaAplicacion}.

\section{Requisitos}
\label{cap:requisitos}

Lo primero que hicimos fue analizar la memoria de AdaptaMaterialEscolar 1.0 extrayendo las funcionalidades que faltaban por implementar y los resultados de la evaluación que se realizó. Tras este análisis surgieron una seria de cambios y nuevas funcionalidades. Agrupamos dichas funcionalidades en tres grupos: formato, ejericios y auxiliar. Quedando las funcionalidades agrupadas de la siguiente manera:
\\

Formato: 
\begin{itemize}
  \item Añadir encabezado al texto: el usuario elijará un encabezado y se le añadirá al CKEditor.
  \item Añadir un tipo de fuente escolar: incluir en los tipos de fuentes la escolar.
  \item Añadir una leyenda de colores con la categoría de cada tipo: crear leyenda de colores según unas características.
  \item Añadir leyenda de colores para el tema de cada asignatura: crear una leyenda según el color del borde de la hoja asignada a cada asignatura.
  \item Añadir ejercicios de matemáticas con cuadrícula para escribir los números: crear una hoja de cuadrículas para los ejercicios de matemáticas.
  \item Añadir la alternativa de añadir doble pauta: en vez de renglones de una única línea, se podrá crear una hoja con doble pauta, para determinar el tamaño de la letra del alumno.
  \item Estandarizar formato para títulos e índices del temario: escoger un formato para el editable.
  \item Enumerar ejercicios de forma automática: establecer un orden numérico para los ejercicios.
\end{itemize}
Ejercicios:
\begin{itemize}
  \item Ejercicios de relacionar contenido mediante flechas: generar un ejercicio para relacionar conceptos.
  \item Añadir ejercicios de cálculo con huecos a rellenar por el alumno: espacios en blanco para que el alumno pueda rellenarlos con el contenido adecuado.
  \item Añadir ejercicios con espacio para dibujar: amplio hueco en blanco con el fin de que el alumno pueda dibujar.
  \item Ejercicios de completar los espacios en blanco en tablas y esquemas: dado una tabla o esquema se establen espacios en blanco para que el alumno los rellene con el contenido adecuado.
\end{itemize}
Auxiliar:
\begin{itemize}
  \item Generar un resumen a partir de un texto: se crea un resumen a partir de un texto.
  \item Exportar el documento a formato Word para hacer modificaciones: exportar un docuemnto PDF a Word.
  \item Añadir un pictotraductor como funcionalidad: dado una frase genera sus respectivos pictogramas.
  \item Añadir imágenes buscando una palabra: a partir de una palabra se busca su respectiva imagen en las bases de datos de imágenes libres.
  \item Sustituir una palabra por una imagen: una palabra se reemplazará por una imagen.
  \item Crear una herramienta de recorte de imágenes para el texto original: añadir una herramienta que recorte imágenes del texto original.
  \item Crear tablas que organicen el temario y/o las actividades, seleccionando contenido: seleccionando el contenido crea una tabla.
  \item Crear esquemas que faciliten la visualización: añadir un esquema para visualizar fácilmente los contenidos.
\end{itemize}
Tras haber analizado en detalle las funcionalidades anteriores hemos encontrado que varias funciones ya están realizadas y otras no se van a implementar por falta de información. A continuación, especificamos las funciones realizadas, las funciones que no aportan información suficiente y las que realizaremos.

Funciones realizadas:
  \begin{itemize}
    \item Añadir encabezado al texto.
    \item Enumerar ejercicios de forma automática.
  \end{itemize}
Funciones sin información suficiente:
\begin{itemize}
  \item Añadir imágenes buscando una palabra en bases de datos de imágenes libres
  \item Sustituir una palabra por una imagen.
  \item Crear una herramienta de recorte de imágenes para el texto original.
  \item Crear tablas que organicen el temario y/o las actividades, seleccionando contenido.
  \item Crear esquemas que faciliten la visualización.
  \item Ejercicios de completar los espacios en blanco en tablas y esquemas.
\end{itemize}


Por lo tanto las fuciones a implementar son las que se muestran en la tabla \ref{Funciones}.
\begin{table}[]
  \begin{tabular}{|l|l|l|l|}
  \hline
  Funciones                                                                                                                                                                            & Coste & Importancia & Prioridad \\ \hline
  Generar un resumen a partir de un texto.                                                                                                                                             & 5     & 5           & 25        \\ \hline
  \begin{tabular}[c]{@{}l@{}}Exportar el documento a formato Word para \\ hacer modificaciones.\end{tabular}                                                                           & 5     & 5           & 25        \\ \hline
  Añadir un pictotraductor como funcionalidad.                                                                                                                                         & 5     & 4           & 20        \\ \hline
  \begin{tabular}[c]{@{}l@{}}Ejercicios de relacionar contenido mediante \\ flechas.\end{tabular}                                                                                      & 5     & 3           & 15        \\ \hline
  Añadir un tipo de fuente escolar.                                                                                                                                                    & 5     & 2           & 10        \\ \hline
  \begin{tabular}[c]{@{}l@{}}Añadir una leyenda de colores con la categoría \\ de cada tipo.\end{tabular}                                                                              & 4     & 2           & 8         \\ \hline
  \begin{tabular}[c]{@{}l@{}}Añadir ejercicios para ejercicios de cálculo con \\ fórmulas con huecos arellenar por el alumno.\end{tabular}                                             & 4     & 2           & 8         \\ \hline
  Añadir ejercicios con espacio para dibujar.                                                                                                                                          & 5     & 1           & 5         \\ \hline
  \begin{tabular}[c]{@{}l@{}}Añadir leyenda de colores para el tema de cada \\ asignatura(color borde personalizar colores).\end{tabular}                                              & 4     & 1           & 4         \\ \hline
  \begin{tabular}[c]{@{}l@{}}Añadir ejercicios de matemáticas con cuadrícula \\ para escribir los números.\end{tabular}                                                                & 3     & 1           & 3         \\ \hline
  \begin{tabular}[c]{@{}l@{}}Añadir la alternativa de añadir doble pauta, en vez \\ de renglones de una única línea, para determinar el \\ tamaño de la letra del alumno.\end{tabular} & 3     & 1           & 3         \\ \hline
  \begin{tabular}[c]{@{}l@{}}Estandarizar formato para títulos e índices del \\ temario.\end{tabular}                                                                                  & 1     & 1           & 1         \\ \hline
  \end{tabular}
  \caption{Funciones a implementar}
  \label{Funciones}
  \end{table}

\section{Diseño de la apliación}
\label{disenyoDeLaAplicacion}
Para el diseño de la aplicación web hemos realizado una iteración competitiva. Cada integrante del grupo ha proporcionado un diseño de la apliación como se muestra en las figuras \ref{IteracionCompetitiva1}, \ref{IteracionCompetitiva2}, \ref{IteracionCompetitiva3} y \ref{IteracionCompetitiva4}.
Una vez que cada integrante ha explicado su diseño, hemos cogido lo mejor de cada uno quedando la aplicación de la siguiente manera \ref{diseño_final}.


\begin{figure}[ht!]
    \centering
    \includegraphics[scale=0.3]{Diseño/IteracionCompetitiva1}
    \caption{Diseño aplicación iteración competitiva 1.}
    \label{IteracionCompetitiva1}
\end{figure}
\begin{figure}[ht!]
    \centering
    \includegraphics[scale=0.5]{Diseño/IteracionCompetitiva2}
    \caption{Diseño aplicación iteración competitiva 2.}
    \label{IteracionCompetitiva2}
\end{figure}
\begin{figure}[ht!]
    \centering
    \includegraphics[scale=0.5]{Diseño/IteracionCompetitiva3}
    \caption{Diseño aplicación iteración competitiva 3.}
    \label{IteracionCompetitiva3}
\end{figure}
\begin{figure}[ht!]
    \centering
    \includegraphics[scale=0.5]{Diseño/IteracionCompetitiva4}
    \caption{Diseño aplicación iteración competitiva 4.}
    \label{IteracionCompetitiva4}
\end{figure}
\begin{figure}[ht!]
  \centering
  \includegraphics[scale=0.3]{Diseño/Diseño Final}
  \caption{Diseño final aplicación.}
  \label{diseño_final}
\end{figure}
