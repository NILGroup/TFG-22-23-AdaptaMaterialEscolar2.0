\chapter{Metodología}
\label{cap:metodologia}
En este capítulo se explicará la metodología de desarrollo utilizada en la Sección \ref{cap:Kanban} y se describirá el plan de pruebas en la Sección \ref{cap:pruebas}.
\section{Metodología de desarrollo}
\label{cap:Kanban}
Para el desarrollo de este trabajo hemos decidido aplicar la metodología Kanban. Esta metodología tiene cuatro reglas básicas: visualizar el flujo de trabajo, determinar y respetar el límite de trabajo en curso (WIP), gestionar el flujo y hacer políticas explícitas. Estas reglas las hemos desarrollado en las siguientes subsecciones.

\subsection{Visualizar el flujo de trabajo: Tablero Kanban}
\label{sec:flujoTrabajo}
En el proyecto hemos distinguido dos tipos de tareas, las tareas relacionadas con la memoria y las de implementación. Para el tablero Kanban hemos decidido crear cinco columnas: \textit{To Do}, \textit{Doing}, \textit{Testing}, \textit{Validate} y \textit{Done}.  Las tareas continuarán a través del flujo siguiendo las siguientes definiciones de las columnas:
\begin{itemize}
    \item \textbf{\textit{To Do}}: Listado de todas las tareas sin empezar.
    \item \textbf{\textit{Doing}}: Tareas que se encuentran en desarrollo, ya sea la implementación de código o la redacción de la memoria.
    \item \textbf{\textit{Testing}}: Una vez desarrollada la tarea, se probará que cumpla con los requisitos.
          Para las tareas de documentación el testing sera realizado por todos integrantes del equipo siguiendo los siguientes pasos:
          \begin{enumerate}
              \item Cuando haya una tarea de memoria en dicha columna, esta dispondra de una lista con \textit{checkboxes} con los nombres de los integrantes.
              \item Cuando un miembro del equipo haya terminado de revisar la tarea debe marcarlo en el \textit{checkbox} referido a él.
              \item Cuando todos los miembros del equipo hayan revisado la tarea, el ultimo revisor se encargará de mover la tarea a la columna de \textit{Validate}.
          \end{enumerate}
          Para las tareas de código, si se encuentra un error en una funcionalidad, ya sea durante esta fase o tras haberse dado por terminada, se creará una nueva tarea de tipo \textit{bug} en la columna \textit{To Do}.
    \item \textbf{\textit{Validate}}: La tarea de memoria será comprobada por las tutoras.
    \item \textbf{\textit{Done}}: Las tutoras han dado el visto bueno a la tarea de memoria o cuando un integrante del grupo haya finalizado su tarea de código.
\end{itemize}

\subsection{Políticas explícitas}
\label{sec:politicas}
A continuación se presentan las políticas explicitas que hemos ido estableciendo a lo largo del proyecto:

\begin{itemize}
    \item Límites del trabajo en curso (WIP): en las columnas de \textit{Doing} y \textit{Testing} habrá como máximo 2 tareas por persona en cada columna, es decir, no podrá haber más de 8 tareas en cada una de esas columnas.
    \item Definición de \textit{Done}:
          \begin{itemize}
              \item Tareas de memoria: Cuando hayan sido validadas por las tutoras.
              \item Tareas de implementación: Cuando hayan pasado todo el plan de pruebas.
          \end{itemize}
    \item Cuando un integrante del grupo haya terminado su tarea él será el encargado de moverla a la columna correspondiente.
    \item Cualquier integrante del grupo puede poner una tarea en el tablero tras consultarlo con el resto.
    \item Hemos llegado al acuerdo de realizar reuniones todos los domingos a las 12:00, para poner en común el trabajo realizado por cada miembro.
    \item Se realizará una reunión de todos los integrantes una semana antes de la revisión con las tutoras para comprobar el trabajo realizado por cada miembro del equipo.
\end{itemize}

\subsection{Clases de servicios}
\label{claseDeServicio}
En Kanban para priorizar las tareas del tablero en ocasiones se emplean las clases de servicio. Estas son una serie de categorías que nos son útiles para clasificar cada una de las tareas de nuestro sistema, las cuales nos permiten identificar rápidamente el nivel de prioridad que tiene la tarea sin hacer un análisis o estimación muy extensa del mismo. Además, la categoría asociada a una tarea determinará como se desplazará la tarea en el tablero. En nuestro caso las clases de servicio que empleamos son las siguientes:
\begin{itemize}
    \item Expedite: Tareas que necesitan ser gestionadas de manera acelerada o urgente. Por ejemplo, algún problema con el editor que nos impediría implementar cualquier otra tarea hasta que esta esté terminada.
    \item Fixed Delivery Date: Tareas con fecha fija que debemos cumplir. Por ejemplo, la corrección de la documentación para la siguiente reunión con las tutoras.
    \item Standard: Tareas que ya ha hecho antes el equipo que no tienen una fecha fija y que ya saben como realizar. Por ejemplo, la realización de pruebas de control de calidad de rutina en el software.
    \item Intangible: Tareas que son nuevas para el equipo, tareas que nunca antes han realizado y por tanto, se desconoce el tiempo que se le va a dedicar y el riesgo que suponen. Por ejemplo, la función de exportar a PDF.
\end{itemize}

Para su aplicación tomaremos unas medidas en base a su prioridad. Las clases \textit{Expedite} son las más prioritarias por lo que serán las primeras en ser realizadas. Las \textit{Fixed Delivery Date}, si en su debida fecha (la cual estará indicada en su descripción) no están implementadas, se convierten en \textit{Expedite}. Las \textit{Standard} son un poco menos prioritarias que las anteriores al contar con el coste y el esfuerzo que suponen pero presentan un cierto grado de incertidumbre al no tener una fecha fija. Por último, en las \textit{Intangibles} su prioridad varía al presentar un alto grado de incertidumbre, ya que inicialmente se desconoce el riesgo que suponen pero pueden convertirse en \textit{Standard} o en \textit{Expedite}.

\section{Plan de Pruebas}
\label{cap:pruebas}

Para asegurar el correcto funcionamiento del software se ha decidido realizar pruebas manuales, ya que se ha considerado que para el tamaño actual del proyecto no supone un problema realizar pruebas manuales para comprobar las distintas funcionalidades. Si, en un futuro, se decide extender la aplicación y añadir más funcionalidades es recomendable actualizar este plan de pruebas, incluyendo pruebas automatizadas e integración continua. Las pruebas de una tarea de implementación concreta las realizará algún miembro del equipo que no haya participado en el desarrollo de esta y las hará cuando la tarea se encuentre en la columna de \textit{Testing}. La ventaja de que las pruebas las realice un miembro que no se haya visto involucrado en el desarrollo de la tarea es que puede sacar más casos de prueba que aquellos miembros que han implementado la tarea y conocen el código.

Cuando un miembro del equipo se haya asignado una tarea de implementación para probar, tendrá que seguir los siguientes pasos:
\begin{enumerate}
    \item \textbf{Generación de los casos de prueba}: Se realizará una tabla con los distintos casos de prueba que se utilizarán para probar una funcionalidad concreta, teniendo en cuenta los requisitos de usuario. Los casos de prueba se entienden como las condiciones de ejecución (precondiciones), el conjunto de entradas, el objetivo de la prueba (campos y condiciones que se quieren comprobar) y los resultados esperados tras ejecutar la prueba (postcondiciones).
    \item \textbf{Definición de los procedimientos de la prueba}: Después de generar los casos de prueba, se definirá un guión en el que se explicarán los pasos a seguir para la ejecutar la prueba. En este guión se deben tener en cuenta todos los casos de prueba generados anteriormente. Cada paso debe ser claro y concreto, además de ofrecer ejemplos de datos de entrada, por ejemplo ``Escribir `hola'\, en el campo de añadir frase.'' o ``Pulsar botón de añadir frase.''.
    \item \textbf{Ejecución de la prueba}: Una vez se hayan definido los pasos necesarios para probar una funcionalidad concreta, hay que ejecutar la prueba, siguiendo estos pasos. Durante la ejecución, se registrarán los resultados obtenidos que no concuerden con los resultados esperados o comportamientos de la funcionalidad que puedan afectar negativamente a la experiencia de usuario.
    \item \textbf{Realización de un informe de la prueba}: Finalmente, se deberá crear una incidencia de tipo \textit{bug} con la clase de servicio \textit{Expedite}. En la descripción de esta incidencia se deberá informar de los resultados obtenidos, los resultados esperados y los pasos a seguir para replicar cada fallo o problema que haya ocurrido durante la ejecución de la prueba. También se puede incluir información que pueda ser relevante para encontrar una solución a cada fallo o problema.
\end{enumerate}

Los casos de prueba y los guiones de prueba de cada funcionalidad se muestran en el Apéndice \ref{ape:pruebas}.